% !TeX encoding = utf8
%This is frenchle.tex  : a light french style for LaTeX 
%                       LPPL Copyright by eFrench
%
%
% =  =  = =  =  = =  =  = =  =  = =  =  = ATTENTION ! =  =  = =  =  = =  =  = =  =  = =  =  = 
%
% Ce source devrait rester dans le dépôt subversion. Il ne faut le publier qu'en forme PDF
% ou DVI. Le source en tant que tel pourrait perturber un utilisateur 
% moyen de LaTeX et frenchle
% =  =  = =  =  = =  =  = =  =  = =  =  = =  =  = =  =  = =  =  = =  =  = =  =  = =  =  = =  =  = =
\documentclass[a4paper,12pt,openright]{article}
\usepackage[utf8]{inputenc}
\usepackage[T1]{fontenc}
\usepackage{times}
\usepackage{url}
\usepackage{moreverb}
\usepackage{relsize}
\makeatletter\def\tempa{\let\if@screen\iffalse}\makeatother% 1.5in
\tempa%
\def\href#1#2{#2}
\let\titcolor\relax
\usepackage[print]{pdfscreen}
\usepackage{mymaj, french}
\usersfrenchoptions{%
  \disallowuchyph
  \overfullhboxmark
}

  \definecolor{section0}{rgb}{.722,.525,.431}
  \definecolor{section1}{rgb}{.650,.350, 0  }
  \definecolor{section2}{rgb}{.750,.250, 0}
  \definecolor{section3}{rgb}{.950,.050,.0}
  \definecolor{section4}{rgb}{.545,.271,.750}
  \definecolor{section5}{rgb}{.627,.322,.176}
  \definecolor{titcolor}{rgb}{.922,.725,.831}
  \definecolor{bleu}{rgb}{.0,.0,.650} 

\hypersetup{
  pdfwindowui=true, pdfnewwindow=true,
  colorlinks=true, pdfmenubar=true, pdffitwindow=true,
  anchorcolor=bleu, urlcolor=bleu,
}

\makeatletter
\newcommand{\linkandfootnote}[3]{%
             \href{#3}{#2}\bgroup%
      \expandafter\ifx\csname r@tlf.#1\endcsname\relax%
       \expandafter\global%
       \expandafter\let\csname r@tlf.#1\endcsname\empty%
                  \footnote{\texttt{<}\url{#3}\texttt{>}.\label{lf.#1}}%
                  \else\refmark{lf.#1}%
                  \fi\egroup%
                           }%
\makeatother

\textwidth=168mm
\textheight=240mm
\voffset -1in
\hoffset -1in
\oddsidemargin=21mm
\evensidemargin=21mm
\topmargin=1cm
\footskip=23mm
\renewcommand{\ttdefault}{lmtt}
\hypersetup{
pdftitle={Le style frenchle },
pdfauthor={Bernard Gaulle, Raymond Juillerat},
pdfcreator={Bernard Gaulle},
pdfsubject={TeX, LaTeX et la langue française},
pdfkeywords={TeX, LaTeX, pdfLaTeX, efrench, frenchle.sty, french.sty},
}
\makeindex
\begin{document}
%\moretolerance
\sloppy
\newcommand{\fbefr}{{\relsize{-1}E}F{\relsize{-1}RENCH}}
\DeclareRobustCommand{\befr}{\fbefr}
\newcommand{\fslefr}{\textsl{\fbefr}}
\DeclareRobustCommand{\slefr}{\fslefr}
\disallowuchyph% C'est ma devise... (inclus dans myfroptn)

% Page de titre :
\makeatletter
\def\@fnsymbol#1{\ensuremath{\ifcase#1\or \star \or \diamond 
                              \or \heartsuit \else\@ctrerr\fi}}
\makeatother
\title{\hbox{}%\vspace*{-3cm}
       \Large
       L'extension \textit{frenchle}\\
       pour \LaTeX\\[0ex]
Notice d'utilisation}

\author{Bernard \fsc{Gaulle}, repris par Raymond \fsc{Juillerat}\\ 
pour le groupe \textsl{\befr} \href{mailto:efrench(at)lists.tuxfamily.org}{\texttt{<efrench.org>}}%
\thanks{Attention, l'adresse m.él. fournie via les butineurs est
piégée ; il faut la corriger à la main.}\\%
LPPL ©  \textsl{\befr} (voir le fichier 
\linkandfootnote{copyright}{\texttt{Copyright.pdf}}
{http://efrench.org/}%
% ou \texttt{copyrigh.tex}
).\\
}
\date{version \textit{le} \frenchstyleid}\label{prempage}
\index{efrench@\textsl{\befr}|(}
\maketitle  \index{Copyright}%\thispagestyle{empty}
%\vfill
\sommaire[3]
\vfill\vfill
\newcommand{\NNL}{\protect\let\\\relax}
\makeatletter\def\HyPsd@CatcodeWarning#1{}\makeatother%
%%
\let\Subsubsection\subsubsection
\def\subsubsection#1{\Subsubsection[\NNL#1]{\protect\color{section3}#1}}
\let\Subsection\subsection
\def\subsection#1{\Subsection[\NNL#1]{\protect\color{section2}#1}}
\let\Section\section
\def\section#1{\Section[\NNL#1]{\protect\color{section1}#1}}

\pagebreak

\section {Introduction}
L’extension \textit{frenchle} pour \LaTeX{} est destinée à « franciser » les documents
imprimés par \LaTeX{} selon les canons typographiques usuellement employés dans
le monde de l’édition française. Elle est aussi appelée « french allégé » car il
existe une version professionnelle , maintenue par le groupe \textsl{\befr} 
offrant plus de dispositifs actifs
et notamment plus de possibilités de personnalisation. Cette dernière est 
aussi sous licence LPPL.
\section {Objectifs}
Cette extension a été conçue d’emblée comme sous-ensemble de \textsl{FrenchPro}\index{FrenchPro@\textsl{FrenchPro}aujourd'hui {\large\textsl{\befr}}}
(aujourd'hui {\befr})
avec les objectifs suivants :
\begin{itemize}
\item être un logiciel libre, ou du moins tout autant que peut l’être \LaTeX.
Ainsi il peut être utilisé gratuitement, sans restriction ; être déposé dans
le domaine public et être rediffusé par n’importe quel moyen.
\item être le plus simple possible d’emploi ; ne pas nécessiter de connaissances
particulières autres que celles de \LaTeX{} et fournir, par défaut, une francisation
acceptable des documents de tous les jours tels que courriers,
articles, notes et rapports. Pour les livres ou les thèses d’autres extensions
peuvent s’avérer nécessaires et pour un niveau de satisfaction élevé,
la version professionnelle peut être requise.
\item permettre d’évoluer vers 
%\textsl{french} de 
la distribution \textsl{\befr} (autrefois \textsl{FrenchPro}\index{FrenchPro@\textsl{FrenchPro}!aujourd'hui {\large\textsl{\befr}}}) 
sans aucune modificiation préalable
du document source (à l’exception du changement de nom d’extension).
\item être très facile d’installation ; ne nécessiter aucune modificiation de fichier
de configuration ; utiliser très peu de fichiers et ainsi s’intégrer naturellement
dans les environnements \LaTeX{} français.
\end{itemize}
Cette notice décrit de façon exhaustive ce que fait pour vous l’extension de
façon automatique, comment vous pouvez l’utiliser et les nouvelles commandes
mises à votre disposition. En outre, vous y trouverez des exemples de l’effet
produit par \textit{frenchle} et ses commandes, sur le document résultant.
En complément, vous pouvez vous reporter à la 
\linkandfootnote{lafaq}{FAQ french}{http://www.efrench.org/distributions/faq.pdf} qui répond à
nombre de questions sur la francisation avec \TeX{} et \LaTeX.
\section {Historique} \label{histo} 
Cet historique est en deux parties.
\begin{itemize}
\item D'abord, il s'agit d'une reprise intégrale de l'historique laissé par Bernard \textsc{Gaulle} 
créateur des extensions \textsl{french} et \textit{frenchle}. Sa denière mise au point datait
du 23 mai 2007. Les seules modifications qui ont été nécessaires ont été de supprimer les
références à des sites qui n'existaient plus sur la Toile, éventuellement d'en remplacer
par un lien vers une adresse valable.
\item Ensuite quelques courtes réflexions au sujet du contexte dans lequel cette histoire
se déroule
\end{itemize}
\subsection{Historique repris de Bernard \textsc{Gaulle}}
Avant \textit{frenchle}, existait l’extension \texttt{french} (version 0.1 à pré-V5) qui avait
été développée pendant 12 ans (1988-2000) et avait traversé avec succès tous
les grands changements de version de \TeX{} (v2, v3, tds) et de \LaTeX{} (2.09, nfss,
2e, dc, ec) tout en restant toujours compatible avec la version précédente pour
assurer la stabilité si chère aux utilisateurs de \TeX.

Au début, l’installation de la francisation nécessitait parfois plusieurs jours
de travail à un gourou \TeX{} et l’utilisateur final disposait de peu de fonctions de
%1. <http://frenchpro.free.fr/FAQ.pdf>.
personnalisation ; il était enfermé dans les limites du style, d’autant que toute
modification nécessitait la connaissance de \LaTeX.

De très gros efforts ont été réalisés pour que toute fonction/décision de
\textsl{french} soit réversible et que l’utilisateur puisse y introduire ses propres améliorations.
Un nombre important de commandes a été développé à cet effet.
Dans le même temps, cette francisation s’adaptait tant bien que mal aux
différentes plates-formes d’exploitation (DOS, Window, MacOs, Unix, ...) et
aux différents moteurs \TeX{} (Personal \-\TeX, OzTEX, GTEX, Web2c, ...). La complexité 
de l’ensemble s’est alors accrue, ne serait-ce que par la multiplication
du nombre de fichiers nécessaires à l’installation.

Pour maîtriser cette expansion et en contrôler l’évolution j’ai choisi la méthode
employée pars Donald \textsc{Knuth} pour \TeX{} à savoir : écoute des demandes des
utilisateurs, discussion des améliorations proposées et intégration des fonctions
d’intérêt général qui enrichissent l’extension et ne remettent pas en cause la
compatibilité avec les versions précédentes. à cela j’avais souhaité que \textsl{french}
n’entre pas dans un circuit commercial sans mon accord préalable. Pour ce faire,
une notice de Copyright avait été élaborée et jointe à la distribution de l’ensemble
des fichiers. Dédié aux francophones, le Copyright de \textsl{french} faisait naturellement
mention de GUTenberg (association francophone des utilisateuers
de \TeX) dont j’étais le président fondateur.

Depuis l’origine, \textsl{french} était archivé sur Internet dans le domaine public.
Aussi, lorsque l’initiative CTAN s’est mise en place avec pour objectif de contenir
tout ce qui a été développé autour de \TeX, \textsl{french} y a été archivé tout
naturellement.

Plus tard, dans les années 1997-1998, le TUG ainsi que plusieurs éditeurs a
souhaité pouvoir rediffuser le contenu de CTAN sur CD; certaines de ces distributions
entrant dans un circuit commercial effectif. La distribution de \textsl{french} ne
pouvait pas s’intégrer, par défaut, à ces dernières. Alors, pour simplifier la tâche
des distributeurs, les archivistes du CTAN on créé en 1999 une zone « \textit{nonfree} »
où \textsl{french} a été déposé, \textit{ex abrupto}, pour ne pas risquer d’être redistribué en
violation du Copyright. Bien que ne ressemblant en rien à un \textit{shareware}, \textsl{french}
s’est alors retrouvé archivé aux côtés des autres \textit{sharewares} du monde \TeX. Cet
état de fait étant insupportable pour moi qui avais permis à quiconque d’utiliser
\textsl{french} gratuitement pendant une bonne dizaine d’années, j’ai demandé son
retrait de CTAN. Mes tentatives pour changer les choses au sein de CTAN et
du TUG ayant échoué, j’ai choisi de m’adapter à la situation en développant
\textit{frenchle} avec licence \LaTeX{} standard et \textsl{FrenchPro} \index{FrenchPro@\textsl{FrenchPro}!d'origine}
sous licence \textit{shareware} ; le
premier pouvant résider normalement sur CTAN et le second pouvant être mis
dans la zone « \textit{nonfree} » aux côtés des autres \textit{sharewares}.

L’extension \textit{frenchle} a été mise dans le domaine public pour la première
fois en juin 2000. Six mois plus tard, \textsl{FrenchProV5} \index{FrenchPro@\textsl{FrenchPro}!d'origine} était disponible et six mois
plus tard encore, les adhérents GUTenberg ayant participé au vote du nouveau
conseil d’administration n’ont pas voulu renouveler leur confiance au président
fondateur. C’est ainsi que l’extension \textit{frenchle} est arrivée dans le monde du
logiciel libre.

Depuis, le TUG (fin 2001) et les associations régionales d’utilisateurs de
\TeX{} (GUTenberg, Dante, etc.) distribuent des CDs contenant l’intégralité des
fichiers disponibles sur CTAN, y compris la zone « \textit{nonfree} » \index{FrenchPro@\textsl{FrenchPro}!d'origine}%
\footnote{ Déplorons juste que \textsl{FrenchPro} y ait été oublié jusqu’à l’année 2003.}. 
Pour sa part, GUTenberg, documente \texttt{frenchb} mais oublie curieusement
\texttt{{\backslash}usepackage[frenchle]\{babel\} }\index{extension!babel@\verb'babel'}
et ne connait pas \textsl{FrenchPro} comme l’atteste
le Cahier hors-série numéro 2 de février 2003, page 24%
\footnote{Quelques corrections ont toutefois été consenties dans une version en ligne
(aujourd'hui [2012] disparue) }.
Dans la liste des « \textit{malheurs} » infligés à mes outils de francisation de \LaTeX,
GUTenberg a demandé à l’auteur du paquetage Babel (de Johannes \textsc{Braams})
de réduire la francisation (option \texttt{french} de Babel) à la seule utilisation de
l’option \texttt{frenchb}\index{babel@\textit{babel}!frenchb@\texttt{frenchb}} de Daniel \textsc{Flipo}. 
Cette mesure a pris effet avec la version
de \LaTeX{} distribuée fin 2003. Ceci est la raison pour laquelle je propose l’extension
\texttt{babelfr}\index{extension!babelfr@\texttt{babelfr}} en remplacement\index{extension!babel@\verb'babel'} 
de \texttt{babel} ce qui ne retire rien à Babel mais lui
redonne les facilités antérieures.

À suivre ... ?

\subsection{Réflexions de Raymond Juillerat}
\subsubsection{Babel et \textsl{mlp}}
Il est clair que les travaux de Daniel \textsc{Flipo} (\texttt{frenchb}) et de Bernard \textsc{Gaulle} (\texttt{french})
sont tous deux remarquables. L'extension \texttt{french.sty} 
(avec ses modules de \textsl{FrenchPro}\index{FrenchPro@\textsl{FrenchPro}!aujourd'hui {\large\textsl{\befr}}})
a toujours été pensée comme
pouvant tourner sous \TeX{} et sous \LaTeX, l'extension \texttt{frenchb.ldf} est prévue comme
fonctionnant uniquement sous \LaTeX{} avec \index{babel@\textit{babel}!options de francisation}
Babel.
J'y vois une incompatibilité. Pour l'allemand, par exemple, cette incompatibilité a subsisté
et il y a un \texttt{german.ldf} et un \texttt{german.sty}.  En ce qui concerne la typographie française, on n'aura 
probablement non plus de solution unique pouvant tourner avec ou sans Babel. Je comprends une certaine 
déception de Bernard Gaulle de voir sa solution multilangage \textsl{mlp}\index{extension!mlp@\textsl{mlp}} dépassée par Babel.
Mais il faudra suivre l'exemple de nos amis germains.
\subsubsection{Les formats}\label{lesformats}
Bernard \textsc{Gaulle} a vécu l'époque où pour une césure\index{césure} correcte, il était très important de créer un 
\index{format}format. Cette initialisation n'est plus nécessaire\index{commande!frlatex@\verb'frlatex'} (depuis 1992 ou même plus tôt)
et a été remplacée par des méthodes plus simples.
Ainsi, il ne faut pas oublier d'activer la langue française dans l'installation de votre système \LaTeX. 
C'est une option dans Mik\TeX{} où l'on peut activer ou désactiver les langues prévues.
La langue française, par défaut y est active.
Avec \TeX live, il ne faut pas oublier de charger le module de langue française.
C'est par une de ces méthodes que le fichier de césure sera présent
et que le fichier \texttt{language.dat} \index{fichier!langage.dat@\texttt{language.dat}}
indiquera à \LaTeX{} où sont les fichiers de césure.

Pourrait-on mettre à disposition, en dehors de \texttt{french.sty} quelques éléments qui
devraient aussi être parfaitement compatibles avec Babel, comme les lettrines, composition en drapeau  ?
Ou d'autres pièces parmi  
celles présentées dans le \linkandfootnote{tortures}{test de torture du dépôt \textsl{\befr}}
{http://svn.tuxfamily.org/viewvc.cgi/efrench_efrenchsources/trunk/test/frenchrf.dvi}
ainsi que des macros comme versatim, vers, numero, ordinal, etc.
Et, pourquoi pas, les intégrer à \texttt{frenchb} ?

 \section{Maintenance et mises-à-jour}
Cette section est divisée en deux sous-sections.
\begin{itemize}
\item D'abord, il s'agit la reprise intégrale de la section laissés par Bernard \textsc{Gaulle} 
au 23 mai 2007. Les seules modifications qui ont été nécessaires c'était de supprimer les
références à des sites qui n'existent plus sur la Toile, éventuellement d'en remplacer
par un lien vers une adresse valable.
\item Enfin la suite de la maintenance, dès que le projet \textsl{\befr} a été mis en place pour reprendre l'œuvre
de son auteur originel afin qu'elle ne tombe pas dans l'oubli.
\end{itemize}
\subsection{Maintenance par Bernard \textsc{Gaulle}}
Bénéficiant des nombreuses années de test dans le cadre du développement
de l’ensemble des modules \textsl{FrenchPro}\index{FrenchPro@\textsl{FrenchPro}!aujourd'hui {\large\textsl{\befr}}} puis \textsl{\befr}, 
l’extension \textit{french allégée} est extrêmement
solide et résistante aux différentes conditions d’utilisation.
Des messages d’erreur ou d’avertissement sont fournis à l’utilisateur en cas
de détection d’anomalie (cf paragr. \ref{messag} page \pageref{messag} ).
à chaque fois qu’un correctif est appliqué à \textit{frenchle} il est immédiatement
publié sur \linkandfootnote{svnfrench}{le serveur web}
{http://svn.tuxfamily.org/viewvc.cgi/efrench_efrenchsources/trunk/inputs/french/frenchle.sty}
%\footnote{disparu depuis} 
 puis recopié sur les serveurs CTAN.
Avant de signaler une erreur éventuelle vous devez vous assurer de reproduire
le problème avec le plus petit fichier source possible. Ensuite, vous pouvez
envoyer votre rapport de bogue sur Internet (liste francophone gut@ens.fr ou
les newsgroup tels que fr.comp.text.tex) où les gourous \LaTeX{} auront vite fait
de trouver, sinon une solution, tout au moins un contournement. Et lorsque vous
estimerez avoir obtenu une solution satisfaisante ou, à l’inverse, aucun début
de solution, n’hésitez pas à nous contacter (voir les contacts, page \pageref{contact}).
\subsection{\textsc{eFrench} et \textit{frenchle}}
%\subsection{\slefr}
Ici quelques explications pour indiquer ce que le groupe {\befr} a apporté à frenchle.sty version 5,9993 pour en
faire une version 5,9995.
\subsubsection{De FrenchPro à \textsc{eFrench}}
Bernard Gaulle, l'auteur des modules de typographie française dénommés {\sl Frenchle} et {\sl FrenchPro} est décédé peu après
avoir sorti les versions 5,9993 de ces extensions de francisation pour \LaTeX{} .
Avec l'accord de l'épouse de l'auteur, héritière des droits sur l'œuvre, un groupe,
lancé par Laurent \textsc{Bloch}
a géré une nouvelle version qui succède à  celle de Bernard Gaulle publiée désormais sous licence LPPL.
 {\sl FrenchPro} est devenu dès lors \textsl{\befr}.
La version de \texttt{french.sty} a ainsi passé à 5,9994. La grande différence, c'est qu'elle n'est plus 
{\sl shareware}, et ne se bloque plus après un mois d'utilisation et beaucoup d'avertissements.
Il n'était alors pas nécessaire alors de modifier {\sl Frenchle} déjà sous licence LPPL.

Les responsables de la distribution \textsl{\befr} et {\sl Frenchle} ont constaté qu'il y avait un nouveau
problème au niveau de l'utilisation multilingue. Si bien qu'il a 
fallu une nouvelle correction de frenchle.sty et french.sty qui ont été portés tous deux à la version 5,9995.

\subsubsection{Utilisation multilingue}\label{multilingue}
\index{multilinguisme}\index{langues}\index{langages}%
\seealso{langages}{langues}
À chaque ou {\em langage}
 indiqué dans le fichier\index{fichier!langage.dat@\texttt{language.dat}} 
\verb|language.dat| l'extension \textsl{\befr} définit une commande du
même nom. Si nous notons ce langage sous la forme : 
\verb|<|{\em langage}\verb|>|,
la commande s'écrira alors \verb|\<|{\em langage}\verb|>|.
Cette nouvelle commande permet alors de passer du français%
\footnote{Chaque commande \texttt{\backslash\string<}{\em langage}\texttt{\string>} 
permet de passer en fait du langage actif à ce nouveau 
\texttt{\string<}{\em langage}\texttt{\string>}.} 
 à ce \verb|<|{\em langage}\verb|>|. 


Toutefois, il y a des exceptions. Un exemple, c'est le passage  à la langue arabe qui, si \textsl{\befr}
avait gardé ce schéma en toute occasion, entrerait en conflit avec le formatage de nombre
en notation arabe qui est défini par \verb|\arabic|. Dans tous les cas où la définition simple 
mènerait à un conflit, il faut choisir \verb|\<|{\em langage}\verb|>Lang| 
\index{langage@\texttt{\backslash<{}}{\em langage}{\tt>Lang{}}}. % 
Dans ces cas, un message est envoyé au journal.
On passera alors à la 
langue arabe par \verb|\arabicLang|. 

En fonctionnement normal une option de style \verb|\<|{\em langage}\verb|>| 
ou \verb|\<|{\em langage}{\tt>Lang{}}
(dont l'extension s'appelle %le code est dans le fichier 
\index{fichier!langage.sty@{\tt<\em langage\tt>.sty}}%
\verb|<|{\em lan\-gage}\verb|>|\texttt{.sty}) 
définit une commande \verb|\<|{\em langage}\verb|>TeXmods|
\index{langage@\texttt{\backslash<{}}{\em langage}{\tt>{}}}% 
\index{langageTeX@\texttt{\backslash<{}}{\em langage}{\tt>TeXmods}}% 
pour compléter cette nouvelle langue (définir des commandes 
ou faire des actions spécifiques) 
et deux autres : \verb|\begin<|\emph{langage}\verb|>| pour y passer
et \verb|\end<|\emph{langage}\verb|>| pour
en sortir. \index{efrench@\textsl{\befr}|)}
Pour l'arabe, on aura \verb|\begin{arabicLang}| et \verb|\end{arabicLang}|
en complément de \verb|\arabicLang|. 

Comme ce mécanisme avait été introduit avec succès dans {\befr}, nous l'avons aussi
introduit dans \textsl{frenchle}.

\section{ Conditions d’utilisation}
Il n’y a pas de condition particulière d’utilisation de l’extension \textit{frenchle} ;
toutefois il est conseillé de placer l’appel de \textit{frenchle} en dernière position dans
%2. Déplorons juste que \textit{eFrench} y ait été oublié jusqu’à l’année 2003.
%3. Quelques corrections ont toutefois été consenties dans une version en ligne à 
%http://math.univ-lille1.fr/~flipo/dfshort.pdf.
%4. http://daniel.flipo.free.fr/frenchb/index.html
%5. <http://frenchle.free.fr/>.
%6. <frenchlebg -chez- free.fr>
le chargement des extensions au sein du document \index{classe!de document}source :\\[.5em]
\rule{0pt}{1em}\hfill\fbox{
\begin{minipage}{90mm}
 \backslash\texttt{documentclass}\{\textit{\texttt{classe\_de\_documents}}\}\\
 \backslash\texttt{usepackage}[\textit{codage d’entrée} ]\{\texttt{inputenc}\}\\
\rule{0.4\textwidth}{0pt} \vdots \\
 \backslash\texttt{usepackage}\{\textit{extension n-2} \}\\
 \backslash\texttt{usepackage}\{\textit{extension n-1} \}\\
 \backslash\texttt{usepackage}\{\texttt{frenchle}\}\\
 \backslash\texttt{begin\{document\}}
\end{minipage}
}\hfill\rule{0pt}{1em}\\[.5em]
%\begin{decl}
%\end{decl}
en effet, il est tenu compte d’un certain nombre d’extensions afin d’en modifier
légèrement le comportement ou bien d’éviter des conflits. Quelque soit le
moteur \TeX, quelque soit l’installation (réalisée correctement) et quelque soit
l’environnement (de bonne programmation) l’extension \textit{frenchle} doit fonctionner
; n’hésitez pas à  \linkandfootnote{noustux}{nous}{mailto:efrench@lists.tuxfamily.org}
faire part d’anomalies flagrantes.
Je rappelle que la gestion des caractères accentués est faite par l’extension
\texttt{inputenc} ; il suffit juste que vous précisiez le codage de votre document (latin9,
utf8, applemac, ansinew, ...).
Nous verrons plus loin (page \pageref{peutfaire}) qu’il existe d’autres possibilités pour faire
appel à \textit{frenchle}.
\section{ Ce que fait l’extension}
Nous arrivons maintenant au côté pratique des choses : je vais vous montrer
ce que fait, par défaut, l’extension \textit{frenchle} sur le plan typographique et
comment son comportement peut éventuellement être changé.
\subsection{ La ponctuation}
La virgule, le point-virgule, les deux-points, le point d’exclamation et le
point d’interrogation sont activés afin de les espacer\index{ponctuation} correctement du texte
précédent.
\begin{minipage}{\textwidth}
\begin{center}
\textit{Texte source}\\[0.5ex]
\begin{boxedverbatim}
$\pi=3,14...$
point-virgule ; le texte
deux-points : le texte
point d’exclamation ! La phrase
point d’interrogation ? La phrase

\end{boxedverbatim}
\\[.5ex]
\begin{minipage}{60mm}
\begin{center}
\textit{Composition standard}
\fbox{\begin{minipage}{55mm}
\nofrenchtypography
$\pi=3,14...$\\
point-virgule ; le texte\\
deux-points : le texte\\
point d’exclamation ! La phrase\\
point d’interrogation ? La phrase
\end{minipage}
}
\end{center}
\end{minipage}
\begin{minipage}{60mm}
\begin{center}
\textit{Composition avec frenchle}
\fbox{\begin{minipage}{55mm}
\frenchtypography
$\pi=3,14...$\\
point-virgule ; le texte\\
deux-points : le texte\\
point d’exclamation ! La phrase\\
point d’interrogation ? La phrase
\end{minipage}
}
\end{center}
\end{minipage}
\end{center}
\end{minipage}
\rule{0pt}{1em}

L’espacement inséré par \textit{frenchle} devant la ponctuation est insécable, ce qui
permet de ne jamais retrouver cette ponctuation en tout début de ligne ou de
page. Une mesure spéciale, du même type, est appliquée aux deux-points afin
de ne pas le retrouver seul en bas de page, coupé du texte qui suit.
En standard, le traitement de la virgule par \LaTeX{} est différent en mode
mathématique et en mode texte. Dans ce dernier mode, aucun espacement n’est
inséré automatiquement par \LaTeX. En mode mathématique, grâce à \textit{frenchle}
il en est maintenant de même (contrairement au fonctionnement standard).
Ainsi, on se souviendra de la règle de base du mode mathématique : la barre
d’espacement ne produit aucun effet ; tout espacement doit être indiqué par une
commande ad hoc ( \backslash, \index{\,@\verb'\,'}
ou \backslash\texttt{\textit{space}} \index{space,@\verb'\space'} 
ou  \backslash\texttt{\textit{thinspace}} ou ...)\index{thinspace,@\verb'\thinspace'} .


\begin{center}
\textit{Texte source}\\[1ex]
\begin{boxedverbatim}
$\pi \simeq 3,14$
PI vaut environ 3,14$
$f(x,y,z)$ saisi sans espacement
$g(x, y, z)$ saisi avec espacement
$h(x,\,y,\,z)$ avec commande ad hoc
\end{boxedverbatim}
\\[0.3ex]
\begin{minipage}{70mm}
\begin{center}
\textit{Composition standard}
\fbox{\begin{minipage}{65mm}
\nofrenchtypography
$\pi \simeq 3,14$\\
$\textrm{PI vaut environ 3,14}$\\
$f(x,y,z)$ saisi sans espacement\\
$g(x, y, z)$ saisi avec espacement\\
$h(x,\,y,\,z)$ avec commande ad hoc
\end{minipage}
}
\end{center}
\end{minipage}
\begin{minipage}{65mm}
\begin{center}
\textit{Composition avec frenchle}
\fbox{\begin{minipage}{60mm}
\frenchtypography
$\pi \simeq 3,14$\\
$\textrm{PI vaut environ 3,14}$\\
$f(x,y,z)$ saisi sans espacement\\
$g(x, y, z)$ saisi avec espacement\\
$h(x,\,y,\,z)$ avec commande ad hoc
\end{minipage}
}
\end{center}
\end{minipage}
\end{center}
\rule{0pt}{1em}


Les commandes de \textit{frenchle} qui régissent l’espacement après la virgule sont\\
\verb|\frenchmathcomma| \index{frenchmathcomma@\verb'\frenchmathcomma'}
et \verb|\regularmathcomma|\index{regularmathcomma@\verb'\regularmathcomma'} :

\begin{center} %=============================
\textit{Texte source}\\[1ex]

\begin{boxedverbatim}
\regularmathcomma $3,14 \ f(x,y)$
\frenchmathcomma  $3,14 \ f(x,y)$
\end{boxedverbatim}
\\[.3ex]
\begin{minipage}{60mm}
\begin{center}
\textit{Composition standard}
\fbox{\begin{minipage}{55mm}
\nofrenchtypography
\regularmathcomma $3,14 \ f(x,y)$\\
\frenchmathcomma $3,14 \ f(x,\,y)$
\end{minipage}
}
\end{center}
\end{minipage}
\begin{minipage}{60mm}
\begin{center}
\textit{Composition avec frenchle}
\fbox{\begin{minipage}{55mm}
\frenchtypography
\regularmathcomma $3,14 \ f(x,y)$\\
\frenchmathcomma $3,14 \ f(x,\,y)$
\end{minipage}
}
\end{center}
\end{minipage}
\end{center}
\rule{0pt}{1em}
L’espacement dans un nombre\index{nombre} (décimal ou non) se fait automatiquement
avec l’ordre \verb|\nombre|\index{nombre@\verb'\nombre'} dès l’instant où un espace a été inséré avec la barre d’espacement,
comme il se doit, après chaque millier ou millième :

%\pagebreak
\begin{center} %=============================
\textit{Texte source}\\[1ex]
\begin{boxedverbatim}
$1 234,567 89$
\nombre{1 234,567 89}
\end{boxedverbatim}
\\[.3ex]
\begin{minipage}{60mm}
\begin{center}
\textit{Composition standard}
\fbox{\begin{minipage}{55mm}
\nofrenchtypography
$1 234,567 89$\\
\nombre{1 234,567 89}
\end{minipage}
}
\end{center}
\end{minipage}
\begin{minipage}{60mm}
\begin{center}
\textit{Composition avec frenchle}
\fbox{\begin{minipage}{55mm}
\frenchtypography
$1 234,567 89$\\
\nombre{1 234,567 89}
\end{minipage}
}
\end{center}
\end{minipage}
\end{center}
%6
%Composition standard
%1234, 56789
%1 234, 567 89
%Composition avec \textit{frenchle}
%1234,56789
%1 234,567 89


À l’exception de cet ordre  \backslash\texttt{nombre}, \textit{frenchle} peut rajouter automatiquement
les espaces que vous auriez oubliés à la saisie devant la double ponctuation :

\begin{center} %=============================
\textit{Texte source}\\
\fbox{\begin{minipage}{130mm}
\nofrenchtypography\tt
\begin{tabular}{ll}
&... fin; il s’ensuit: boum! quoi?\\
\{{\backslash}untypedspaces&... fin; il s’ensuit: boum! quoi?\}\\
&... fin ; il s’ensuit : boum ! quoi ?\\
\end{tabular}
\end{minipage}
}\\[.5em]
\begin{minipage}{60mm}
\begin{center}
\textit{Composition standard}
\fbox{\begin{minipage}{55mm}
\nofrenchtypography
... fin; il s’ensuit: boum! quoi?\\
{\untypedspaces... fin; il s’ensuit: boum! quoi?}\\
... fin ; il s’ensuit : boum ! quoi ?
\end{minipage}
}
\end{center}
\end{minipage}
\begin{minipage}{60mm}
\begin{center}
\textit{Composition avec frenchle}
\fbox{\begin{minipage}{55mm}
\frenchtypography
... fin; il s’ensuit: boum! quoi?\\
{\untypedspaces... fin; il s’ensuit: boum! quoi?}\\
... fin ; il s’ensuit : boum ! quoi ?
\end{minipage}
}
\end{center}
\end{minipage}
\end{center}
% ====================================
\rule{0pt}{1em}

Vous l’avez compris, c’est l’option \verb|\typedspaces|\index{typedspaces@\verb'\typedspaces'} qui est active par défaut
(c’est-à-dire que l’on saisit tous les espacements nécessaires).
Tous les effets que l’on vient de voir sont contrôlables par les ordres \verb|\typedspaces|\index{typedspaces@\verb'\typedspaces'}
\verb|\untypedspaces|\index{untypedspaces@\verb'\untypedspaces'}, 
 \backslash\texttt{frenchtypography}\index{frenchtypography@\verb'\frenchtypography'}
et  \backslash\texttt{nofrenchtypography} ; voyons, sur l’exemple suivant, ce que
\index{nofrenchtypography@\verb'\nofrenchtypography'}
cela donne avec  \backslash\texttt{nofrenchtypography}.
\begin{center}
\textit{Texte source}\\
\begin{boxedverbatim}
{\nofrenchtypography
$\pi=3,14...$\\
point-virgule ; le texte\\
deux-points : le texte\\
point d’exclamation ! La phrase\\
point d’interrogation ? La phrase\\
}
\end{boxedverbatim}
\\[.5em]
\begin{minipage}{60mm}
\begin{center}
\textit{Composition standard}
\fbox{\begin{minipage}{55mm}
\nofrenchtypography
$\pi=3,14...$\\
point-virgule ; le texte\\
deux-points : le texte\\
point d’exclamation ! La phrase\\
point d’interrogation ? La phrase
\end{minipage}
}
\end{center}
\end{minipage}
\begin{minipage}{60mm}
\begin{center}
\textit{Composition avec frenchle}
\fbox{\begin{minipage}{55mm}
\nofrenchtypography
$\pi=3,14...$\\
point-virgule ; le texte\\
deux-points : le texte\\
point d’exclamation ! La phrase\\
point d’interrogation ? La phrase
\end{minipage}
}
\end{center}
\end{minipage}
\end{center}
\rule{0pt}{1em}
\frenchtypography\frenchlayout
Vous voyez donc clairement qu’en plaçant l’ordre \index{nofrenchtypography@\verb'\nofrenchtypography'}
 \backslash\texttt{nofrenchtypography} au
bon endroit, on annule les effets micro-typographiques de \textit{frenchle}.

\subsection{Les notes}\index{notes}
Un petit exemple vaudra mieux qu’un long discours :
%\let\oldrfootcounter\thefootnote %- - - - - - - - - - - - - - - - - - - - - - - - counter thefootnote
\begin{center} %=============================
\textit{Texte source}\\\index{footnote@\verb'\footnote'}
\begin{boxedverbatim}
\noindent Ici\footnote{Essai de note.}\\
          Ici \footnote{Ceci est une note.}
et là \footnote{Une autre note.}\footnote{Et
 encore une autre note.}.
\end{boxedverbatim}
\setcounter{mpfootnote}{1} %- - - - - - - - - - - - - - - - - - - - - - - - counter mpfootnote
\renewcommand{\thempfootnote}{\arabic{mpfootnote}}
\parbox{60mm}{
\begin{center}
\textit{Composition standard} %-----------------------------------------standard
\fbox{\begin{minipage}{55mm}
\nofrenchtypography\nofrenchlayout{
\noindent 
\large Ici\footnote{Essai de note.}\\
\large Ici \footnote{Ceci est une note.}
et là\footnote{Une autre note.}\footnote{Et
encore une autre note.}.}
\vspace{5ex}
\end{minipage}
}%
\end{center}
}%
\parbox{65mm}{
\begin{center}
\textit{Composition avec frenchle} %-----------------------------------------frenchle
\fbox{\frenchtypography\frenchlayout%
\begin{tabular}{@{}lll}
%\frenchtypography{
%\renewcommand{\@makefntext}{\fr@makefntext}%
\multicolumn{2}{@{}l}{\large Ici\footnotemark[1]}\\
\multicolumn{2}{@{}l}{\large Ici\footnotemark[2]
et là $^{3,}$\,$^4$.}\\
\\
\\\cline{1-2}%}
\rule{0.5em}{0pt}&\footnotesize1. {Essai de note.}&\\[-.2em]
&\multicolumn{2}{l}{\footnotesize2. {Ceci est une note.}}\\[-.2em]
&\footnotesize3. {Une autre note.}\\[-.2em]
&\multicolumn{2}{l}{\footnotesize4. {Et encore une autre note.\rule{10mm}{0pt}}}\\[-.03em]
\end{tabular}%
}%
\end{center}
}
\end{center}
% ====================================
Comme on le voit, l’espacement devant l’ordre \texttt{{\backslash}footnote} %n’a aucune importance
est sans importance
avec \textit{frenchle}. Deux \texttt{{\backslash}footnotes} peuvent se suivre (sans espace cette fois-ci)
et être composées selon l’usage. Vous aurez bien entendu noté la différence\index{notes!de bas de page}
en bas de page.

Dans la page de titre, \LaTeX{} exige que l’on utilise l’ordre \texttt{{\backslash}thanks} à la place
de {\backslash}footnote :
\begin{center} %=============================
\textit{Texte source}\\[.3ex]
\begin{boxedverbatim}
\title{L’extension \textit{frenchle}
\thanks{Merci à tous les contributeurs.} \\
pour les documents en français
\author{B. Gaulle}
}
\maketitle
\end{boxedverbatim}
\\[.5ex]
\setcounter{mpfootnote}{1} %- - - - - - - - - - - - - - - - - - - - - - - - counter mpfootnote
\renewcommand{\thempfootnote}{\arabic{mpfootnote}}
\parbox{70mm}{
\begin{center}
\textit{Composition standard}
\fbox{\begin{minipage}{65mm}
\nofrenchtypography\nofrenchlayout{
%\let{\@mpfootnotetext}{\@footnotetex}t%
\noindent 
\hfill \large L'extension \textit{frenchle}\footnote {Merci à tous les contributeurs}\hfill\rule{0pt}{1pt}\\
\large pour les documents en français}\\
\rule{0pt}{2em}\hfill B. Gaulle\hfill\rule{0pt}{1pt}\\
\rule{0pt}{2em}
\end{minipage}
}%
\end{center}
}%
\parbox{70mm}{
\begin{center}
\textit{Composition avec frenchle}
\fbox{\frenchtypography\frenchlayout%
\begin{tabular}{@{}ll@{}l@{}}
%\frenchtypography{
%\renewcommand{\@makefntext}{\fr@makefntext}%
\multicolumn{3}{@{}c}{\large L'extension \textit{frenchle}\footnotemark[1]}\\
\multicolumn{3}{@{}c}{\large pour les documents en français}\\
&\rule{5em}{0pt}\\
\multicolumn{3}{@{}c}{\rule{0pt}{1em}B. Gaulle}\\
\multicolumn{3}{@{}c}{\rule{0pt}{2em}}\\\cline{1-2}
&\multicolumn{2}{l}{\rule{0pt}{1em}\footnotesize1. {Merci à tous les contributeurs}}\\
\end{tabular}%
}%
\end{center}
}
\end{center}

Des notes peuvent aussi être placées dans une figure\index{notes!dans les figures} mais en général le texte
de la note est perdu par \LaTeX{} et \textit{frenchle} vous en avertit, il faut alors procéder
comme suit : d'une part \index{footnotemark@\verb'\footnotemark'}
saisir {\backslash}footnotemark à la place de \verb'\footnote' pour placer simplement \index{footnotetext@\verb'\footnotetext'}
l’appel de note et ensuite, après la figure saisir\index{figures!notes dans les }
 \verb'\footnotetext'\{texte de la note\}.\label{notesfigu}
\begin{center} %=============================
\begin{minipage}{\textwidth}
\begin{center}
\textit{Texte source}\\[1ex]
\begin{boxedverbatim}
\begin{figure}
  \begin{tabular}{|c|} 
    \hline 
    \centerline{Peu importe la figure  \footnotemark.} \\ 
    \hline 
  \end{tabular} 
     \caption{Une figure explicative}
 \end{figure} 
 \footnotetext{Une explication peut en amener une autre 
 en bas de page}
\end{boxedverbatim}
\end{center}
\end{minipage}
\setcounter{mpfootnote}{1} %- - - - - - - - - - - - - - - - - - - - - - - - counter mpfootnote
\renewcommand{\thempfootnote}{\arabic{mpfootnote}}
\parbox{70mm}{
\begin{center}
\textit{Composition standard}
\fbox{\begin{minipage}{65mm}
\nofrenchtypography\nofrenchlayout{
\noindent 
\rule{0pt}{1pt}\hfill \fbox{
\large Peu importe la figure\footnote{%
Une explication peut en amener une autre en bas de page}.}\hfill\rule{0pt}{1pt}\\
}
\rule{0pt}{1pt}\hfill\rule{0pt}{1.5em} Figure 1: Une figure explicative\hfill\rule{0pt}{1pt}\\
\vspace{4em}
\end{minipage}
}%
\end{center}
}%
\parbox{70mm}{
\begin{center}
\textit{Composition avec frenchle}

\fbox{
\begin{minipage}{65mm}
\hfill\fbox{\large Peu importe la figure\footnotemark[1].}\hfill\rule{0pt}{1pt}\\[1em]
\centerline{\textsc{Fig.} 1 -- \textit{Une figure explicative}}
\begin{tabular}{lll}
&\rule{4.3em}{0pt}\\[1.9em]
\multicolumn{3}{@{}c}{\rule{0pt}{2em}}\\\cline{1-2}
&\multicolumn{2}{l}{\rule{0pt}{1em}\footnotesize1. {Une explication peut en amener une}}\\[-.3em]
\multicolumn{3}{l}{\footnotesize{}autre en bas de page}\\
\end{tabular}
\end{minipage}}
\end{center}
}
%
\end{center} % ===========================

Comme vous le savez les figures \index{figures} et tableaux sont flottants avec \LaTeX{} c’est
à dire qu’il sont placés au mieux par \LaTeX{} en fonction des options que vous
choisissez ; il est donc parfois délicat de faire résider sur la même page la figure
et le texte de la note en bas de page.

Pour les tableaux\index{tableaux}, par contre, les notes ne sont pas perdues en français car
elles sont composées juste en dessous du tableau\index{tableaux!notes dans les}
 (comme dans le cas d’une minipage
\LaTeX) :
\begin{center} %=============================
\textit{Texte source}\\[1ex]

\begin{boxedverbatim}
\begin{table} 
Il est toujours possible de placer une
note dans un tableau en français : 
\bigskip 
\begin{tabular}{|c|} 
  \hline 
   Peu importe le tableau 
   \footnote{Une explication en 
             dessous (perdue ?).}. \\
   \hline 
 \end{tabular} 
 \end{table}
\end{boxedverbatim}

\setcounter{mpfootnote}{1} %- - - - - - - - - - - - - - - - - - - - - - - - counter mpfootnote
\renewcommand{\thempfootnote}{\alph{mpfootnote}}
\parbox[t]{70mm}{
\begin{center}
\textit{Composition standard}
{\nofrenchtypography\nofrenchlayout
\fbox{
\begin{minipage}{65mm}
Il est toujours possible de placer une\\
note dans un tableau en français :\\
\bigskip
\begin{tabular}{|c|}
\hline
\rule{0pt}{1.2em}Peu importe le tableau
\footnotemark[1].\\
\hline
\end{tabular}
\end{minipage}
}}
\end{center}
}%
\parbox[t]{70mm}{
\begin{center}
\textit{Composition avec frenchle}
\fbox{
\begin{minipage}{65mm}
Il est toujours possible de placer une\\
note dans un tableau en français :\\
\bigskip
\begin{tabular}{|c|}
\hline
\rule{0pt}{1.2em}Peu importe le tableau
\footnote{Une explication en
dessous (perdue ?).}.\\
\hline
\end{tabular}
\end{minipage}
}

\end{center}
}
%
\end{center}

\frenchtypography\frenchlayout
Tous les effets que l’on vient de voir pour les notes sont contrôlables par
les ordres \texttt{{\backslash}french\-typography}\index{frenchtypography@\verb'\frenchtypography'}
 et \verb|\frenchlayout|\index{frenchlayout@\verb'\frenchlayout'}, l’un gérant l’espacement et
l’autre la mise-en-page. Pour les désactiver vous pouvez coder 
\texttt{{\backslash}nofrenchtypography}\index{nofrenchtypography@\verb'\nofrenchtypography'}
et \texttt{{\backslash}nofrenchlayout}\index{nofrenchlayout@\verb'\nofrenchlayout'} ; 
essayons-les pour voir ce que cela donne :

\begin{center} %=============================
\textit{Texte source}\\
\begin{boxedverbatim}
{ \nofrenchtypography  
 \nofrenchlayout 
 \begin{table} 
    Peu importe le tableau 
    \footnote{Une note perdue.}.\\
 \hline 
 \end{tabular} 
 \end{table}
}
\end{boxedverbatim}
\\[.5em]
\setcounter{mpfootnote}{1} %- - - - - - - - - - - - - - - - - - - - - - - - counter mpfootnote
\renewcommand{\thempfootnote}{\arabic{mpfootnote}}
\parbox[t]{70mm}{
\begin{center}
\textit{Composition standard}
{\nofrenchtypography\nofrenchlayout
\begin{minipage}{65mm}
\begin{tabular}{|c|}
\hline
\rule{0pt}{1.6em}\Large Peu importe le tableau
\footnotemark[1].\\
\hline
\end{tabular}
\end{minipage}
}
\end{center}
}%
\parbox[t]{70mm}{
\begin{center}
\textit{Composition avec frenchle}

\begin{minipage}{65mm}
\begin{tabular}{|c|}
\hline
\rule{0pt}{1.6em}\Large Peu importe le tableau
\footnotemark[1].\\
\hline
\end{tabular}
\end{minipage}

\end{center}
}
%
\end{center}

Ainsi, la composition par \textit{frenchle} est redevenue tout à fait au standard
\LaTeX.
\subsection{ Le titrage}
L’extension \textit{frenchle} traduit et compose les différents éléments servant au
titrage\index{titrage} dans le document comme vous le trouverez ci-après. Une exception
toutefois concerne l’ordre \verb|\chaptername|\index{chaptername@\verb'\chaptername'} qui, pour des raisons de conception
des hauts de pages de \LaTeX, ne peut pas disposer, pour la plupart du temps,
d’une traduction anglaise.\\[1em]

\begin{tabular}{|l|l|l|}\hline
\rule{0pt}{1.1em}\textit{Composition en anglais}&\textit{Texte source}&\textit{Composition en français}\\\hline
\rule{0pt}{1em}References         &\tt{\backslash}refname        &Références \index{refname@\verb'\refname '}         \\
Abstract           &\tt{\backslash}abstractname   &Résumé    \index{abstractname@\verb'\abstractname '}         \\
Table of Contents  &\tt{\backslash}contentsname   &Table des matières \index{contentsname@\verb'\contentsname'}\\
List of Figures    &\tt{\backslash}listfigurename &Table des figures  \index{listfigurename@\verb'\listfigurename'}\\
List of Tables     &\tt{\backslash}listtablename  &Liste des tableaux\index{listtablename@\verb'\listtablename'} \\
Index              &\tt{\backslash}indexname      &Index            \index{indexname@\verb'\indexname'}  \\
Figure             &\tt{\backslash}figurename     &Fig.            \index{figurename@\verb'\figurename'}   \\
Table              &\tt{\backslash}tablename      &Tab.          \index{tablename@\verb'\tablename'}     \\
Part               &\tt{\backslash}partname       &partie        \index{partname@\verb'\partname'}     \\
\hline
\end{tabular}

\bigskip
Il s’agit là de quelques commandes standard de \LaTeX{} mais \textit{frenchle} en
introduit d’autres dans certains environnements de classe ou d’extensions :

\bigskip
\begin{tabular}{|l|l|l|}\hline
\rule{0pt}{1.1em}\textit{Composition en anglais}&\textit{Texte source}&\textit{Composition en français}\\\hline
\rule{0pt}{1em}see       &\tt{\backslash}seename          &voir \index{seename@\verb'\seename'}         \\
see also  &\tt{\backslash}seealsoname      &voir aussi    \index{seealsoname @\verb'\seealsoname '}\\
Glossary  &\tt{\backslash}glossaryname     &Glossaire  \index{glossaryname@\verb'\glossaryname'}   \\
Keywords: &\tt{\backslash}kwname           &Mots-clé :    \index{kwname@\verb'\kwname'}\\
DRAFT     &\tt{\backslash}draftname        &- épreuve -   \index{draftname@\verb'\draftname'}\\
Preface   &\tt{\backslash}prefacename      &Préface    \index{prefacename@\verb'\prefacename'}   \\
Proof     &\tt{\backslash}proofname        &Démonstration\index{proofname@\verb'\proofname'} \\
To        &\tt{\backslash}headtoname\verb|\mbox{}|&              \index{headtoname@\verb'\headtoname'}\\
cc        &\tt{\backslash}ccname           &c.c.          \index{ccname@\verb'\ccname'}\\
Encl      &\tt{\backslash}enclname         &P.j.       \index{enclname@\verb'\enclname'}   \\
PS:       &\tt{\backslash}PSname           &P.-S. :    \index{PSname@\verb'\PSname'}   \\
Subject:  &\tt{\backslash}Objectname       &Objet :      \index{Objectname@\verb'\Objectname'} \\
Your Ref: &\tt{\backslash}YourRefname      &v/réf. :   \index{YourRefname@\verb'\YourRefname'}   \\
Our Ref:  &\tt{\backslash}OurRefname       &n/réf. :     \index{OurRefname@\verb'\OurRefname'} \\
email:    &\tt{\backslash}emailname        &m.él. :       \index{emailname@\verb'\emailname'}\\
Slide     &\tt{\backslash}slidename        &Transparent  \index{slidename@\verb'\slidename'} \\
Notes     &\tt{\backslash}notesname        &Notes         \index{notesname@\verb'\notesname'}\\
Appendix  &\tt{\backslash}appendixname     &Annexe       \index{appendixname@\verb'\appendixname'} \\
\hline %
\end{tabular}\\[1em]
(Le libellé {\tt{\backslash}draftname} est destiné à une utilisation en PostScript)

D’autres libellés peuvent être employés dans d’autres extensions ; leur traduction
dépendra de l’interaction de celle-ci avec \textit{frenchle}. Pour être garanti
d’obtenir l’interaction maximum il est souhaitable de coder l’option \textsl{french}
dans {\tt{\backslash}documentclass}\index{classe!de document}
 (voir les autres possibilités au paragr. \ref{peutfaire} page \pageref{peutfaire}) :\\[1em]
\rule{0pt}{1em}\hfill\fbox{
\begin{minipage}{90mm}
 \backslash\texttt{documentclass[french]}\{\textit{classe\_de\_documents}\}\\
 \backslash\texttt{usepackage}\{\textit{extension 1}\}\\
\rule{0.4\textwidth}{0pt} \vdots \\
 \backslash\texttt{usepackage}\{\textit{extension n-1} \}\\
 \backslash\texttt{usepackage}\{\texttt{frenchle}\}\\
 \backslash\texttt{begin\{document\}}
\end{minipage}
}\hfill\rule{0pt}{1em}\\[.5em]

Cette simple option sera alors passée à toutes les extensions qui ont prévu
une traduction en français ; la première de ces extensions étant \textit{frenchle} 
elle-même (ou \texttt{babel}\index{extension!babel@\verb'babel'} si vous utilisez d’autres langues que l’anglais et le français ; cf
paragr.  \ref{multilin} page \pageref{multilin}).
Vous pouvez, bien sûr, à tout moment modifier la traduction des libellés de
votre choix en redéfinissant les macros-instructions mais il est préférable de le
faire avant le \texttt{{\backslash}begin\{document\}}.

\noindent
\begin{minipage}{\textwidth}
\begin{center} %=============================
\textit{Texte source}\\[.5ex]
{\large
\begin{boxedverbatim}
\fraddto\captionsfrench{\def\figurename{Figurine}}

\begin{figure}
  \begin{tabular}{|c|}
    \hline
    \centerline{Peu importe la figurine.}\\
    \hline
\end{tabular}
\caption{Une nouvelle figurine}
\end{figure}
\end{boxedverbatim}
}\\[.5em]
\end{center}
\end{minipage}
\begin{center}
\setcounter{mpfootnote}{1} %- - - - - - - - - - - - - - - - - - - - - - - - counter mpfootnote
\renewcommand{\thempfootnote}{\arabic{mpfootnote}}
\parbox{60mm}{
\begin{center}
\textit{Composition standard}
\fbox{\begin{minipage}{55mm}
\nofrenchtypography\nofrenchlayout{
\noindent 
\rule{0pt}{1pt}\hfill \fbox{
\large Peu importe la figurine}\hfill\rule{0pt}{1pt}\\
}
\rule{0pt}{1pt}\hfill\rule{0pt}{1.5em} Figure 1: Une nouvelle figurine\hfill\rule{0pt}{1pt}\\
\end{minipage}
}%
\end{center}
}%
\parbox{65mm}{
\begin{center}
\textit{Composition avec frenchle}
\fbox{
\begin{minipage}{60mm}
\hfill\fbox{\large Peu importe la figurine}\hfill\rule{0pt}{1pt}\\[1em]
\centerline{{Figurine} 1 -- \textit{Une nouvelle figurine }}\\
\end{minipage}}
\end{center}
}
%
\end{center} % ===========================

On notera que les définitions sont introduites simplement à la façon \TeX{} pour
s’appliquer dans toutes les circonstances.

Les titres des figures et des tableaux sont précisés avec l’ordre \texttt{{\backslash}caption} (au
dessus dans le cas d’un tableau ; en dessous dans le cas d’une figure) comme
dans l’exemple ci-après.
\begin{center} %=============================
\begin{minipage}{\textwidth}
\begin{center}
\textit{Texte source}\\[.5em]
{%\large
\begin{boxedverbatim}
\begin{figure}
  \begin{tabular}{|c|}
    \hline
    \centerline{Peu importe la figure.}\\
    \hline
  \end{tabular}
\caption{Une figure explicative}
\end{figure}

\begin{table}
\caption{Exemple de tableau}\medskip
  \begin{tabular}{|c|}
    \hline
    \centerline{Peu importe le tableau.}\\
    \hline
  \end{tabular}
\end{table}
\end{boxedverbatim}
}
\\[.5em]
\end{center}
\end{minipage}%{\textwidth}
\setcounter{mpfootnote}{1} %- - - - - - - - - - - - - - - - - - - - - - - - counter mpfootnote
\renewcommand{\thempfootnote}{\arabic{mpfootnote}}
\parbox{70mm}{
\begin{center}
\textit{Composition standard}
\fbox{\begin{minipage}{65mm}
\nofrenchtypography\nofrenchlayout{
\noindent 
\rule{0pt}{1pt}\hfill \fbox{
 \hspace{2em}Peu importe la figure\hspace{2em}}\hfill\rule{0pt}{1pt}\\
}
\rule{0pt}{1pt}\hfill\rule{0pt}{1.5em} Figure 1: Une figure explicative\hfill\rule{0pt}{1pt}\\[3em]
\rule{0pt}{1pt}\hfill\rule{0pt}{1.5em} Table 1: Exemple de tableau\hfill\rule{0pt}{1pt}\\
\rule{0pt}{1pt}\hfill \fbox{
 \hspace{2em}Peu importe le tableau.\hspace{2em}}\hfill\rule{0pt}{1pt}
\end{minipage}
}%
\end{center}
}%
\parbox{70mm}{
\begin{center}
\textit{Composition avec frenchle}
\fbox{
\begin{minipage}{65mm}
\hfill\fbox{\hspace{1em}Peu importe la figure\hspace{1em}}\hfill\rule{0pt}{1pt}\\[.75em]
\centerline{\textsc{Fig.} 1 -- \textit{Une figure explicative }}\\[3.65em]
\centerline{\textsc{Tab.} 1 -- \textit{Exemple de tableau }}\\[.05em]
\rule{0pt}{1pt}\hfill\fbox{\hspace{2em}Peu importe le tableau\hspace{2em}}\hfill\rule{0pt}{1pt}
\end{minipage}}
\end{center}
}
%
\end{center} % ===========================

La composition de ces titres est contrôlée avec l’aide des ordres 
\texttt{{\backslash}captionfont}\index{captionfont@\verb'\captionfont'}
et\\ \texttt{{\backslash}captionseparator}\index{captionseparator@\verb'\captionseparator'} :

\begin{center} %=============================
\textit{Texte source}\\[1ex]
\begin{boxedverbatim}
\renewcommand{\captionfont}{\upshape
                           \bfseries\selectfont}
\renewcommand{\captionseparator}{$\Rightarrow$}

\begin{figure}
\begin{tabular}{|c|}
\hline
\centerline{Peu importe la figure.}\\
\hline
\end{tabular}
\caption{Une figure explicative}
\end{figure}
\end{boxedverbatim}
\\[.5em]
\setcounter{mpfootnote}{1} %- - - - - - - - - - - - - - - - - - - - - - - - counter mpfootnote
\renewcommand{\thempfootnote}{\arabic{mpfootnote}}
\parbox{70mm}{
\begin{center}
\textit{Composition standard}
\fbox{\begin{minipage}{65mm}
\nofrenchtypography\nofrenchlayout{
\noindent 
\rule{0pt}{1pt}\hfill \fbox{
\hspace{2em}Peu importe la figure.\hspace{2em}}\hfill\rule{0pt}{1pt}\\
}
\rule{0pt}{1pt}\hfill\rule{0pt}{1.5em} Figure 1: Une figure explicative\hfill\rule{0pt}{1pt}\\
\end{minipage}
}%
\end{center}
}%
\parbox{70mm}{
\begin{center}
\textit{Composition avec frenchle}
\fbox{
\begin{minipage}{65mm}
\hfill\fbox{\hspace{2em}Peu importe la figure\hspace{2em}}\hfill\rule{0pt}{1pt}\\[1em]
\centerline{\textsc{Fig.} 1$\Rightarrow$ \textbf{Une figure explicative }}\\
\end{minipage}}
\end{center}
}
%
\end{center} % ===========================

Tous les effets précédents sont contrôlés par les ordres 
\texttt{{\backslash}frenchtypography}\index{frenchtypography@\verb'\frenchtypography'} 
et \texttt{{\backslash}frenchtrans\-lation} \index{frenchtranslation@\verb'\frenchtranslation'}
( et leurs inverses \texttt{{\backslash}nofrenchtypography}
\index{nofrenchtypography@\verb'\nofrenchtypography'} 
et \texttt{{\backslash}nofrench\-translation)} ; \index{nofrenchtranslation@\verb'\nofrenchtranslation'}
faisons un essai avec \texttt{{\backslash}nofrenchtranslation} :
\begin{center} %=============================
\begin{minipage}{\textwidth}
\begin{center} %=============================
\textit{Texte source}\\
\begin{boxedverbatim}
{\nofrenchtranslation
 \begin{table}
 \caption{Titres divers}\medskip
  \begin{tabular}{|c|c|}
   \hline
    \verb|\tablename|    & \tablename \\
    \verb|\appendixname| & \appendixname \\
    \verb|\figurename|   & \figurename \\
    \verb|\abstractname| & \abstractname \\
    \verb|\contentsname| & \contentsname \\
    \hline
  \end{tabular}
 \end{table}
}
\end{boxedverbatim}
\end{center}
\end{minipage}
\\[.5em]
\setcounter{mpfootnote}{1} %- - - - - - - - - - - - - - - - - - - - - - - - counter mpfootnote
\renewcommand{\thempfootnote}{\arabic{mpfootnote}}

\noindent
\parbox{70mm}{
\begin{center}
\textit{Composition standard} \small \nofrenchtranslation
\begin{tabular}{|@{}c|c|c|@{}c|}\hline
\multicolumn{4}{|c|} {\rule{0pt}{1.2em}Table 1: Titres divers}\\[.2em]\cline{2 - 3}
\rule{0pt}{1.2em}&\tt{\backslash}tablename & \tablename&\\
&\tt{\backslash}appendixname & \appendixname& \\
&\tt{\backslash}figurename& \figurename&\\
&\tt{\backslash}abstractname&\abstractname&\\
&\tt{\backslash}contentsname&\contentsname&\\\cline{2 - 3}
\multicolumn{4}{|c|}{}\\\hline
\end{tabular}
\end{center}
}%
\parbox{70mm}{
\begin{center}
\textit{Composition avec frenchle} \small \nofrenchtranslation
\begin{tabular}{|@{}c|c|c|@{}c|}\hline
\multicolumn{4}{|c|} {\rule{0pt}{1.2em}Table 1:  \textit{Titres divers}}\\[.2em]\cline{2 - 3}
\rule{0pt}{1.2em}&\tt{\backslash}tablename & \tablename&\\
&\tt{\backslash}appendixname & \appendixname& \\
&\tt{\backslash}figurename& \figurename&\\
&\tt{\backslash}abstractname&\abstractname&\\
&\tt{\backslash}contentsname&\contentsname&\\\cline{2 - 3}
\multicolumn{4}{|c|}{}\\\hline
\end{tabular}\end{center}
}
%
\end{center} % ===========================

et maintenant mettons simplement \verb'\nofrenchtypography'\index{nofrenchtypography@\verb'\nofrenchtypography'} 
pour voir quel effet
cela produit sur le même tableau ; il va falloir ouvrir l’œil !
\begin{center} %=============================
\begin{minipage}{\textwidth}
\begin{center}
\textit{Texte source}\\
\begin{boxedverbatim}
{\nofrenchtypography
 \begin{table}
 \caption{Titres divers}\medskip
  \begin{tabular}{|c|c|}
   \hline
    \verb|\tablename|    & \tablename \\
    \verb|\appendixname| & \appendixname \\
    \verb|\figurename|   & \figurename \\
    \verb|\abstractname| & \abstractname \\
    \verb|\contentsname| & \contentsname \\
    \hline
  \end{tabular}
 \end{table}
}
\end{boxedverbatim}
\\[.5em]
\end{center}
\end{minipage}%{\textwith}
\setcounter{mpfootnote}{1} %- - - - - - - - - - - - - - - - - - - - - - - - counter mpfootnote
\renewcommand{\thempfootnote}{\arabic{mpfootnote}}

\noindent
\parbox{70mm}{
\begin{center} \nofrenchtypography
\textit{Composition standard} \small \nofrenchtranslation
\begin{tabular}{|@{}c|c|c|c@{}|}\hline
\multicolumn{4}{|c|} {\rule{0pt}{1.2em}Table 1: Titres divers}\\[.2em]\cline{2 - 3}
\rule{0pt}{1.2em}&\tt{\backslash}tablename & \tablename&\\
&\tt{\backslash}appendixname & \appendixname& \\
&\tt{\backslash}figurename& \figurename&\\
&\tt{\backslash}abstractname&\abstractname&\\
&\tt{\backslash}contentsname&\contentsname&\\\cline{2 - 3}
\multicolumn{4}{|c|}{}\\\hline
\end{tabular}
\end{center}
}%
\parbox{70mm}{
\begin{center} \nofrenchtypography
\textit{Composition avec frenchle} \small 
\begin{tabular}{|@{}c|c|c|c@{}|}\hline
\multicolumn{4}{|c|} {\rule{0pt}{1.2em}\textsc{Tab.} %Table 
1: {Titres divers}}\\[.2em]\cline{2 - 3}
\rule{0pt}{1.2em}&\tt{\backslash}tablename & \tablename&\\
&\tt{\backslash}appendixname & \appendixname& \\
&\tt{\backslash}figurename& \figurename&\\
&\tt{\backslash}abstractname&\abstractname&\\
&\tt{\backslash}contentsname&\contentsname&\\\cline{2 - 3}
\multicolumn{4}{|c|}{}\\\hline
\end{tabular}\end{center}
}
%
\end{center} % ===========================
%\subparagraph
{\textbf{ Remarque de B. \textsc{Gaulle}}} Avez-vous trouvé où la 
\texttt{{\backslash}frenchtypography}\index{frenchtypography@\verb'\frenchtypography'} 
n’est pas appliquée ici ?
 Bravo !
\subsection{La mise en page}
L’extension \textit{frenchle} modifie la mise en page de \LaTeX.\index{mise en page} Premier exemple :
avec \LaTeX, par défaut, tous les débuts de paragraphes sont mis en retrait de
la marge, sauf pour les premiers paragraphes de partie, chapitre, section, etc.
L’extension \textit{frenchle} corrige ce problème :\\[1em]
\begin{center} %=============================
\begin{minipage}{\textwidth}
\begin{center}
\textit{Texte source}\\
\begin{boxedverbatim}
\section{Nouvelle section}
Le premier paragraphe débute ainsi.
Ceci est un texte de remplissage sans
intérêt.

Le second paragraphe débute ainsi.
Ceci est un texte de remplissage sans
intérêt.
\end{boxedverbatim}
\\[.5em]
\end{center}
\end{minipage}
\setcounter{mpfootnote}{1} %- - - - - - - - - - - - - - - - - - - - - - - - counter mpfootnote
\renewcommand{\thempfootnote}{\arabic{mpfootnote}}
\noindent
\parbox{70mm}{
\begin{center} \nofrenchtypography
\textit{Composition standard} \small 
\end{center}
\fbox{
\begin{minipage}{60mm}
{\Large \bf 1\hspace{.5em} Nouvelle section}\\
\rule{0pt}{2.1em}Le premier paragraphe débute ainsi.
Ceci est un texte de remplissage sans
intérêt.

\hspace*{1.5em}Le second paragraphe débute ainsi.
Ceci est un texte de remplissage sans
intérêt.
\end{minipage}
}
}%
\parbox{70mm}{
\begin{center} \frenchtypography \frenchlayout
\textit{Composition avec frenchle} \small 
\end{center}
\fbox{
\begin{minipage}{60mm}
{\Large \bf 1\hspace{.5em} Nouvelle section}\\
\hspace*{1.5em\rule{0pt}{2.1em}}Le premier paragraphe débute ainsi.
Ceci est un texte de remplissage sans
intérêt.

\hspace*{1.5em}Le second paragraphe débute ainsi.
Ceci est un texte de remplissage sans
intérêt.
\end{minipage}
}
}\\[1.5em]
%
\end{center} % ===========================

Si vous souhaitez, pour d’autres besoins, utiliser l’extension \texttt{titlesec}, alors
\textit{frenchle} adoptera les options (\texttt{indentfirst}\index{indentfirst@\verb'\indentfirst'} 
ou \texttt{nonindentfirst}\index{nonindentfirst@\verb'\nonindentfirst'}) par défaut ou
précisées au chargement de l’extension.

Avec l’extension \textit{frenchle} la mise en page des listes est différente au niveau
des marqueurs et des espacements verticaux :
\begin{center} %=============================
\begin{minipage}{\textwidth}
\begin{center} %=============================
\textit{Texte source}\\[.5ex]
\begin{boxedverbatim}
La liste se compose de :
\begin{itemize}
\item premier élément de la liste ;
\item deuxième élément ;
  \begin{itemize}
  \item sous-élément ...
  \end{itemize}
\item dernier élément.
\end{itemize}
\end{boxedverbatim}
\\[.5em]
\end{center}
\end{minipage}
\setcounter{mpfootnote}{1} %- - - - - - - - - - - - - - - - - - - - - - - - counter mpfootnote
\renewcommand{\thempfootnote}{\arabic{mpfootnote}}
\noindent
\frlabelitems{\renewcommand{\labelitemi}{--}%
\renewcommand{\labelitemii}{--}%
\renewcommand{\labelitemiii}{--}%
}
\parbox[t]{65mm}{
\begin{center} \nofrenchtypography \nofrenchlayout \english
\textit{Composition standard} \small 
\end{center}
\fbox{ \nofrenchtrivsep
\begin{minipage}{55mm}
La liste se compose de :
\begin{itemize}
\item [\textbullet]premier élément de la liste ;
\item [$\bullet$]deuxième élément ;
\begin{itemize}
\item [--]sous-élément ...
\end{itemize}
\item [$\bullet$]dernier élément.
\end{itemize}
\end{minipage}
}
}%
%\renewcommand{\FrenchLabelItem}{\textemdash}
%\frlabelitems{\renewcommand{\labelitemi}{--}%
%\renewcommand{\labelitemii}{--}%
%\renewcommand{\labelitemiii}{--}}%
\parbox[t]{65mm}{
\begin{center} \frenchtypography \frenchlayout
\textit{Composition avec frenchle} \small 
\end{center}
\fbox{
\begin{minipage}{55mm}
La liste se compose de :
\begin{itemize}
\item premier élément de la liste ;
\item deuxième élément ;
\begin{itemize}
\item sous-élément ...
\end{itemize}
\item dernier élément.
\end{itemize}

\end{minipage}
}
}\\[1.5em]
%
\end{center} % ===========================

Les espacements verticaux sont gérés par les ordres \\
\texttt{\backslash{}frenchtrivsep}\index{frenchtrivsep@\verb'\frenchtrivsep'} et
\texttt{\backslash{}nofrenchtrivsep}\index{nofrenchtrivsep@\verb'\nofrenchtrivsep'} :

\begin{center} %=============================
\textit{Texte source}\\[1ex]
\begin{boxedverbatim}
{\nofrenchtrivsep

La liste se compose de :
\begin{itemize}
\item premier élément de la liste ;
\item deuxième élément ;
\item dernier élément.
\end{itemize}
}
\end{boxedverbatim}
\\[.5em]\setcounter{mpfootnote}{1} %- - - - - - - - - - - - - - - - - - - - - - - - counter mpfootnote
\renewcommand{\thempfootnote}{\arabic{mpfootnote}}
\noindent
\frlabelitems{\renewcommand{\labelitemi}{--}%
\renewcommand{\labelitemii}{--}%
\renewcommand{\labelitemiii}{--}%
}
\parbox[t]{65mm}{
\begin{center} \nofrenchtypography \nofrenchlayout \english
\textit{Composition standard} \small 
\end{center}
\fbox{ \nofrenchtrivsep
\begin{minipage}{55mm}
La liste se compose de :
\begin{itemize}
\item [$\bullet$]premier élément de la liste ;
\item [$\bullet$]deuxième élément ;
%\begin{itemize}
%\item [--]sous-élément ...
%\end{itemize}
\item [$\bullet$]dernier élément.
\end{itemize}
\end{minipage}
}
}%
%\renewcommand{\FrenchLabelItem}{\textemdash}
%\frlabelitems{\renewcommand{\labelitemi}{--}%
%\renewcommand{\labelitemii}{--}%
%\renewcommand{\labelitemiii}{--}}%
\parbox[t]{65mm}{
\begin{center} \frenchtypography \frenchlayout
\textit{Composition avec frenchle} \small
\end{center}
\fbox{\nofrenchtrivsep
\begin{minipage}{55mm}
La liste se compose de :
\begin{itemize}
\item premier élément de la liste ;
\item deuxième élément ;
%\begin{itemize}
%\item sous-élément ...
%\end{itemize}
\item dernier élément.
\end{itemize}

\end{minipage}
}
}\\[1.5em]
%
\end{center} % ===========================

Les valeurs d’espacement sont appliqués par \textit{frenchle} à tous les environnements
de liste, de façon homogène. À chaque fois que vous essaierez de modifier
la valeur de ces dimensions (ou qu’une extension le fera à votre place) vous serez
pévenu par \textit{frenchle} que celà n’est pas possible de la sorte. L’affichage de ces
messages\index{messages} d’avertissement est contôlé par les ordres 
\verb|\frenchtrivsepwarnings|\index{frenchtrivsepwarnings@\verb'\frenchtrivsepwarnings'} 
et \verb|\nofrenchtrivsepwarnings|\index{nofrenchtrivsepwarnings@\verb'\nofrenchtrivsepwarnings'}.

Vous pouvez choisir vos propres valeurs d’espacement dans les listes en
utilisant l’ordre \\\verb|\frtrivseplengths| et dans ce cas les messages d’avertisssement
précités ne sont plus émis (sauf à 
rajouter \verb|\frenchtrivsepwarnings|\index{frenchtrivsepwarnings@\verb'\frenchtrivsepwarnings'}
dans \verb|\frtrivseplengths|)\index{frtrivseplengths@\verb'\frtrivseplengths'}
 ; l’exemple suivant vous le démontre.

\begin{center} %=============================
\textit{Texte source}\\[1ex]
\begin{boxedverbatim}
\frtrivseplengths{%
       \setlength{\parsep}{0.1ex plus 0.1ex minus 0.1ex}%
       \setlength{\itemsep}{0.1ex plus 0.1ex minus 0.1ex}%
       \setlength{\topsep}{0.2ex plus 0.2ex minus 0.2ex}%
       \setlength{\partopsep}{0.8ex plus 0.8ex minus 0.8ex}%
                }

La liste se compose de :
\begin{itemize}
\item premier élément de la liste ;
\item deuxième élément ;
\item dernier élément.
\end{itemize}
\end{boxedverbatim}
\setcounter{mpfootnote}{1} %- - - - - - - - - - - - - - - - - - - - - - - - counter mpfootnote
\renewcommand{\thempfootnote}{\arabic{mpfootnote}}
\noindent
\frlabelitems{\renewcommand{\labelitemi}{--}%
\renewcommand{\labelitemii}{--}%
\renewcommand{\labelitemiii}{--}%
}
\parbox[t]{65mm}{
\begin{center} \nofrenchtypography \nofrenchlayout \english
\textit{Composition standard} \small 
\end{center}
\fbox{ \nofrenchtrivsep
\begin{minipage}{55mm}
La liste se compose de :
\begin{itemize}
\item [$\bullet$]premier élément de la liste ;
\item [$\bullet$]deuxième élément ;
%\begin{itemize}
%\item [--]sous-élément ...
%\end{itemize}
\item [$\bullet$]dernier élément.
\end{itemize}
\end{minipage}
}
}%
%\renewcommand{\FrenchLabelItem}{\textemdash}
%\frlabelitems{\renewcommand{\labelitemi}{--}%
%\renewcommand{\labelitemii}{--}%
%\renewcommand{\labelitemiii}{--}}%
\parbox[t]{65mm}{
\begin{center} \frenchtypography \frenchlayout
\textit{Composition avec frenchle} \small 
\end{center}
\fbox{%\nofrenchtrivsep
\begin{minipage}{55mm}
\frtrivseplengths{%
\setlength{\parsep}{0.1ex plus 0.1ex minus 0.1ex}%
\setlength{\itemsep}{0.1ex plus 0.1ex minus 0.1ex}%
\setlength{\topsep}{0.2ex plus 0.2ex minus 0.2ex}%
\setlength{\partopsep}{0.8ex plus 0.8ex minus 0.8ex}%
}

La liste se compose de :
\begin{itemize}
\item premier élément de la liste ;
\item deuxième élément ;
%\begin{itemize}
%\item sous-élément ...
%\end{itemize}
\item dernier élément.
\end{itemize}

\end{minipage}
}
}\\[1.5em]
%
\end{center} % ===========================

Vous pouvez aussi changer les marqueurs de liste en faisant appel à la
commande spéciale \verb|\frlabelitems|\index{frlabelitems@\verb'\frlabelitems'} comme dans l’exemple (sans aucun intérêt typographique)
suivant :

\begin{center} %=============================
\begin{minipage}{\textwidth}
\begin{center} %=============================
\textit{Texte source}\\\index{labelitemi@\verb'\labelitemi'}\index{labelitemii@\verb'\labelitemii'}\index{labelitemiii@\verb'\labelitemiii'}
\begin{boxedverbatim}
\frlabelitems{\renewcommand{\labelitemi}{*}%                                 
              \renewcommand{\labelitemii}{**}%
              \renewcommand{\labelitemiii}{***}%
             }
La liste se compose de :
\begin{itemize}
\item premier élément de la liste ;
  \begin{itemize}
  \item sous-élément ...
  \end{itemize}
\item dernier élément.
\end{itemize}
\end{boxedverbatim}
\end{center}
\end{minipage}
\setcounter{mpfootnote}{1} %- - - - - - - - - - - - - - - - - - - - - - - - counter mpfootnote
\renewcommand{\thempfootnote}{\arabic{mpfootnote}}
\noindent
\parbox[t]{65mm}{
\begin{center} \nofrenchtypography \nofrenchlayout \english
\textit{Composition standard} \small 
\end{center}
\fbox{ \nofrenchtrivsep
\begin{minipage}{55mm}
\frlabelitems{\renewcommand{\labelitemi}{*}%                                 
\renewcommand{\labelitemii}{**}%
\renewcommand{\labelitemiii}{***}%
              }
La liste se compose de :
\begin{itemize}
\item [\textbullet]premier élément de la liste ;
\item [$\bullet$]deuxième élément ;
\begin{itemize}
\item [--]sous-élément ...
\end{itemize}
\item [$\bullet$]dernier élément.
\end{itemize}
\end{minipage}
}
}%
\parbox[t]{65mm}{
\begin{center} \frenchtypography \frenchlayout
\textit{Composition avec frenchle} \small 
\end{center}
\fbox{
\begin{minipage}{55mm}
\frlabelitems{\renewcommand{\labelitemi}{*}%                                 
\renewcommand{\labelitemii}{**}%
\renewcommand{\labelitemiii}{***}%
}
La liste se compose de :
\begin{itemize}
\item premier élément de la liste ;
\item deuxième élément ;
\begin{itemize}
\item sous-élément ...
\end{itemize}
\item dernier élément.
\end{itemize}

\end{minipage}
}
}\\[1.5em]
%
\end{center} % ===========================
\frlabelitems{\renewcommand{\labelitemi}{--}%
\renewcommand{\labelitemii}{--}%
\renewcommand{\labelitemiii}{--}%
}


L’extension \textit{frenchle} s’adapte à la classe de document demandée ; la mise
en page peut donc être extrêmement différente, voici un exemple avec la classe
\texttt{letter}\index{classe!letter@\texttt{letter}} :
\begin{center} %=============================
\textit{Texte source}\\[1ex]
\begin{boxedverbatim}
\address{B. \textsc{Gaulle}\\ Mon adresse}
\begin{letter}
       {M. Dupond\\
       17, rue St. \’Eloi\\
       24140 La Monzie Montastruc}

  \opening{Cher Monsieur X,}

  La documentation est là !

  \closing{Bonne lecture.}
  \cc{Monsieur le directeur Y}
  \encl{lettre du ministre Z}
\end{letter}
\end{boxedverbatim}

\noindent
\parbox[t]{66mm}{
\begin{center} \nofrenchtypography \nofrenchlayout \english
\textit{Composition standard}
\end{center}
\fbox{ \nofrenchtrivsep
\begin{minipage}{61mm} \scriptsize
\begin{flushright}
\begin{tabular}{l@{}}
B. \textsc{Gaulle}\\
  Mon adresse\\[2ex]
April 28, 2012
\end{tabular}
\end{flushright}
{M. Dupond\\
17, rue St.Éloi\\
24140 La Monzie Montastruc}\\[1.5ex]

{Cher Monsieur X,}\\[-.5ex]

La documentation est là !\\[1ex]
%\rule{0pt}{1pt}\hfill\hfill{Bonne lecture.}\hfill\rule{0pt}{1pt}\\
\centerline{\phantom{Bonne lecture.}Bonne lecture.}\\[11ex]
cc: Monsieur le directeur Y\\[.5ex]
encl: lettre du ministre Z
\end{minipage}
}
}%
\parbox[t]{66mm}{
\begin{center} \frenchtypography \frenchlayout
\textit{Composition avec frenchle}  
\end{center}
\fbox{
\begin{minipage}{61mm}\scriptsize
B. \textsc{Gaulle}\\
  Mon adresse\\[-2.5ex]
\begin{flushright}
\begin{tabular}{l@{}}
M. Dupond\\
17, rue St.Éloi\\
24140 La Monzie Montastruc\\[6ex]
Le 28 avril  2012\\[3.5ex]
\end{tabular}
\end{flushright}
\rule{3em}{0pt}Cher Monsieur X,\\[5ex]
\rule{5em}{0pt}La documentation est là !\\[2ex]
\rule{5em}{0pt}Bonne lecture.\\[15ex]
c.c. : Monsieur le directeur Y\\
P. j. : lettre du ministre Z
\end{minipage}
}
}\\[1.5em]
%
\end{center} % ===========================

Bien entendu, l’adresse de l’expéditeur doit se retrouver dans la fenêtre de
l’enveloppe.\\


Avec la classe \texttt{book}\index{classe!book@\texttt{book}} \textit{frenchle} s’efforce de maintenir la numérotation des
pages au même endroit tout au long du document (contrairement à \LaTeX) ; par
ailleurs la mise en page du titre courant et des débuts de partie sont différentes :

\begin{center} %=============================
\textit{Texte source}\\[1ex]
\begin{boxedverbatim}
\chapter{Nouveau chapitre}	

\section{Nouvelle section}

Il était une fois dans la villle de Foix un ...          
\end{boxedverbatim}

\noindent
\parbox[t]{66mm}{
\begin{center}  \frenchtypography \frenchlayout
\textit{Composition standard}
\end{center}
\fbox{ 
\begin{minipage}{61mm} %\scriptsize
\vspace{7ex}
{\large\textbf{Chapter 2\\[1ex]
Nouveau chapitre}}\\[3ex]
\textbf{2.1.\hspace{1em} Nouvelle section}\\[1ex]
{\tiny Il était une fois dans la villle de Foix un ...}\\[5.5ex]
\begin{center}
{\tiny {3}}
\end{center}
\end{minipage}
}
}%
\parbox[t]{66mm}{
\begin{center} 
\textit{Composition avec frenchle}  
\end{center}
\fbox{
\begin{minipage}{61mm}%\scriptsize
%\centerline{\tiny \textsc{CHAPITRE 2\hspace{2em}NOUVEAU CHAPITRE}}
{\tiny \rule{1em}{0pt}\hfill 3\rule{1em}{0pt}}\\[6ex]
{\large\textbf{Chapitre 2\\[1ex]
Nouveau chapitre}}\\[3ex]
\textbf{2.1.\hspace{1em} Nouvelle section}\\[1ex]
\rule{2ex}{0pt}{\tiny Il était une fois dans la villle de Foix un ...}\\[8ex]
\rule{1em}{0pt}
\end{minipage}%\\[1.5em]
}
}
\end{center} % ===========================

À chaque changement de partie la numérotation des chapitres est remise à
zéro ; si vous ne le souhaitez pas il suffit de coder \verb|\noresetatpart|\index{noresetatpart@\verb'\noresetatpart'}.

Dans le même registre, \LaTeX{} remet à zéro le compteur de notes de bas de
page à chaque chapitre ; si vous ne le souhaitez pas il suffit alors de coder l’ordre
\verb|\noresetatchapter|\index{noresetatchapter@\verb'\noresetatchapter'}.\\

L’extension \textit{frenchle} tient compte aussi des styles ou classes de documents
de l’AMS\index{classe!amsbook@\texttt{amsbook}} :

\begin{center} %=============================
\textit{Texte source}\\[1ex]
\begin{boxedverbatim}
\chapter{Nouveau chapitre avec amsbook}

\section{Nouvelle section}

Il était une fois un livre que
j’avais composé avec amour ...
\end{boxedverbatim}

\noindent
\parbox[t]{66mm}{
\begin{center}  \frenchtypography \frenchlayout
\textit{Composition standard}
\end{center}
\fbox{ 
\begin{minipage}{61mm} %\scriptsize \tinySMALL
\begin{center}
{\scriptsize \textsc CHAPTER 2}\\[5ex]
\textbf{Nouveau chapitre avec amsbook}\\[3ex]
\textbf{\small 1.\hspace{1em} Nouvelle section}\\[.5ex]
\end{center}
{\scriptsize Il était une fois un livre que
j’avais composé avec amour ...}\\[9ex]
\centerline{\tiny 3}
\end{minipage}
}
}%
\parbox[t]{66mm}{
\begin{center} 
\textit{Composition avec frenchle}  
\end{center}
\fbox{
\begin{minipage}{61mm}%\scriptsize
{\tiny \hfill 1.\hspace{1em}NOUVELLE SECTION\hfill{3}}\\[7ex]
\begin{center}
CHAPITRE 2\\[3ex]
\textbf{Nouveau chapitre avec amsbook}\\[1ex]
\textbf{\small 1.\hspace{1em} Nouvelle section}\\[1ex]
\end{center}
{\scriptsize Il était une fois un livre que
j’avais composé avec amour ...}
\end{minipage}
}
}%\\[1.5em]
%
\end{center} % ===========================
\vspace{1ex}

Toute la mise en page et différentes choses dont on a déjà parlé (comme les
notes dans les tables et les notes consécutives) sont en fait contrôlées par les
ordres \verb|\frenchlayout| (qui est l’option par défaut)\index{frenchlayout@\verb'\frenchlayout'}
 et \verb|\nofrenchlayout| ; \index{nofrenchlayout@\verb'\nofrenchlayout'}
voici à nouveau l’exemple avec la classe \texttt{letter}\index{classe!letter@\texttt{letter}} :
\begin{center} %=============================
\begin{minipage}{\textwidth}
\begin{center}
\textit{Texte source}\\[1ex]
\begin{boxedverbatim}
\nofrenchlayout
\address{B.          \textsc{Gaulle}}
\begin{letter}
{M. Dupond\\
17, rue St. \’Eloi\\
24140 La Monzie Montastruc}

\opening{Cher Monsieur X,}

La documentation est là !

\closing{Bonne lecture.}
\cc{Monsieur le directeur Y}
\encl{lettre du ministre Z}
\end{letter}
\end{boxedverbatim}
\end{center}
\end{minipage}

\noindent
\parbox[t]{66mm}{
\begin{center} \nofrenchtypography \nofrenchlayout \english
\textit{Composition standard}
\end{center}
\fbox{ \nofrenchtrivsep
\begin{minipage}{61mm} \scriptsize
\begin{flushright}
\begin{tabular}{l@{}}
B. \textsc{Gaulle}\\
  \\[2ex]
April 28, 2012
\end{tabular}
\end{flushright}
{M. Dupond\\
17, rue St.Éloi\\
24140 La Monzie Montastruc}\\[1.5ex]

{Cher Monsieur X,}\\[-.5ex]

La documentation est là !\\[1ex]
%\rule{0pt}{1pt}\hfill\hfill{Bonne lecture.}\hfill\rule{0pt}{1pt}\\
\centerline{\phantom{Bonne lecture.}Bonne lecture.}\\[11ex]
cc: Monsieur le directeur Y\\[.5ex]
encl: lettre du ministre Z
\end{minipage}
}
}%
\parbox[t]{66mm}{
\begin{center} \frenchtypography \frenchlayout
\textit{Composition avec frenchle}  
\end{center}
\fbox{
\begin{minipage}{61mm}\scriptsize
\begin{flushright}
\begin{tabular}{l@{}}
B. \textsc{Gaulle}\\
  \\[2ex]
28 April 2012
\end{tabular}
\end{flushright}
{M. Dupond\\
17, rue St.Éloi\\
24140 La Monzie Montastruc}\\[1.5ex]

{Cher Monsieur X,}\\[-.5ex]

La documentation est là !\\[1ex]
%\rule{0pt}{1pt}\hfill\hfill{Bonne lecture.}\hfill\rule{0pt}{1pt}\\
\centerline{\phantom{Bonne lecture.}Bonne lecture.}\\[11ex]
c.c. : Monsieur le directeur Y\\[.5ex]
P. j. : lettre du ministre Z
\end{minipage}
}
}\\[1.5em]
%
\end{center} % ===========================
 \frenchtypography \frenchlayout

\subsection{Le multilinguisme}\label{multilin}
Par défaut, \textit{frenchle} est bilingue français-anglais, c’est-à-dire que vous pouvez
composer des parties du document selon les habitudes typographiques
françaises et d’autres à l’anglaise (ou plutôt à l’américaine). Il existe plusieurs
méthodes pour réaliser cela :
\begin{description}
\item[1\iere méthode :] dès que vous souhaitez passer en anglais vous saisissez l’ordre
\verb|\english|\index{english@\verb'\english'} et lorsque vous désirez revenir au français vous 
saisissez \verb|\french|\index{french@\verb'\french'}.
\begin{center}
\fbox{
\begin{minipage}{65mm}
\centerline{\texttt{\backslash{}english} \textit{texte en anglais} \texttt{\backslash{}french}}
\end{minipage}
}
\end{center}
\item[2\ieme méthode :] puisque le document, par défaut, est en français vous pouvez
limiter la partie anglaise à un seul bloc \LaTeX{}:
\begin{center}
\fbox{
\begin{minipage}{50mm}
\centerline{\texttt{\{\backslash{}english} \textit{texte en anglais} \}}
\end{minipage}
}
\end{center}
%{\english texte en anglais }
\item[3\ieme méthode :] consiste à appliquer la méthode \LaTeX{} c’est-à-dire à utiliser
l’environnement \texttt{english} :
\begin{center}
\fbox{
\begin{minipage}{40mm}
\texttt{\backslash{}begin\{english\}}\\
\textit{partie à l’anglaise}\\
\texttt{\backslash{}end\{english\}}
\end{minipage}
}
\end{center}
\item[4\ieme méthode :] en utilisant l’extension \texttt{babel}\index{extension!babel@\verb'babel'}
(on lui préfèrera l’extension \texttt{babelfr}\index{extension!babelfr@\texttt{babelfr}}) :
\begin{center}
\fbox{
\begin{minipage}{65mm}
\texttt{\backslash{}usepackage[frenchle]\{babelfr\}}\\
\texttt{\backslash{}begin\{document\}}\\
\vdots\\
\texttt{\backslash{}selectlanguage\{english\}}\\
\textit{\texttt{partie à l’anglaise}}\\
\texttt{\backslash{}selectlanguage\{french\}}
\end{minipage}
}\\[1ex]
\end{center}
\end{description}

\noindent Bien entendu, tout ce que l’on vient de dire est réversible ; je veux dire que
vous pouvez remplacer \texttt{english} par \texttt{french} si votre objectif est de composer
seulement quelques morceaux en français.

Pour passer du \index{bilinguisme}bilinguisme 
au multilinguisme il est nécessaire de faire appel \index{multilinguisme}\index{langues}\index{langages}
à d’autres extensions, la plus connue étant \texttt{babel} 
(qui couvre un très grand nombre \index{extension!babel@\verb'babel'}
de langages) mais aussi l’extension \textsl{mlp}\index{extension!mlp@\textsl{mlp}} de Bernard \textsc{Gaulle} qui permet aussi
l’allemand et est distribuée avec \textsl{\befr}. \index{efrench@\textsl{\befr}}
Il est possible d'étendre \textsl{mlp}\index{extension!mlp@\textsl{mlp}} à d'autres langages.

\paragraph*{ Faites donc le pas à \textsl{\befr}.}
Du moment que \textsl{\befr} est publié sous la même licence LPPL
que {\sl Frenchle}, rien ne s'oppose à ce que vous utilisiez 
%\textsl{french} et 
les autres outils de la distribution
 \textsl{\befr} plutôt que {\it frenchle}.
Vous disposerez alors de nombreuses possibilités supplémentaires.
Tout se trouve dans les archives CTAN sous \index{efrench@\textsl{\befr}}
{\sl language/french/efrench}.

\section{Ce que ne fait pas l'extension}
Les versions antérieures à \LaTeXe ne peuvent pas être utilisées avec \textit{frenchle}.
De même, le mode de compatibilité avec \LaTeX{} 2.09 n’est pas accepté par
\textit{frenchle}.

L’extension \textit{frenchle} ne gère pas la coupure des mots (ou césure) ; cela est\index{césure}
un dispositif intégré à \LaTeX{} par un dispositif dérivé de \texttt{babel}\index{extension!babel@\verb'babel'}.
Celui-ci est normalement totalement contrôlé par la gestion des langues intégrée
à votre système (\TeX{} Live ou Mik\TeX, voir aussi sous \ref{multilingue} à la page \pageref{multilingue}).

Les classes de document non standard (et a fortiori les classes « maison ») \index{classe!non standard@\textit{non standard}} %!frabbrev@\texttt{frabbrev.tex}
ne sont pas supportées par \textit{frenchle}. En dehors de ce principe il y a la pratique
qui est bien différente car toutes les classes de document sont en général
« calquées » sur les classes standard de \LaTeX{} et donc respectent les méthodes
de programmation et les structures \LaTeX{} qui sont nécessaires à \textit{frenchle}. Donc,
beaucoup de classes de document fonctionnent bien avec l’extension \textit{frenchle}.

L’extension \textit{frenchle} ne corrige pas les erreurs de \LaTeX, cela va de soi,
quoique ... quand cela est possible nous essayons d’améliorer la situation. Ainsi, par
exemple, \LaTeX{} a la fâcheuse habitude de perdre les notes de bas de page dans
certaines situations (cf page \pageref{notesfigu}). Avec \textit{frenchle} vous serez prévenu en général que
le texte de votre note n’a pas été imprimé et un message vous donne même la méthode
pour vous en sortir. Aussi vaut-il mieux ne jamais supprimer les messages\index{messages} de
\textit{frenchle}.

L’extension \textit{frenchle} ne cherche pas à remplacer les extensions ou autres dispositifs
efficaces qui existent par ailleurs ; citons, par exemple, l’extension \texttt{aeguill}\index{extension!aeguill@\verb'aeguill'}
pour obtenir des guillemets francais ou l’extension \texttt{lettrine}\index{extension!lettrine@\verb'lettrine'} pour obtenir des
grosses lettres en début de paragraphe. Notons que ces deux sujets sont traités
entièrement par \textsl{\befr} .
Raison de plus de faire le pas vers \textsl{\befr} .\index{efrench@\textsl{\befr}}
Pour faciliter le passage de  \texttt{frenchb} à \textit{frenchle} le traitement des guillemets\index{guillemets}
de  \texttt{frenchb} a été introduit dans \textit{frenchle} ; on dispose ainsi des commandes \verb|\og|
et \verb|\fg| ; voici un exemple d’utilisation :

\begin{center}
\fbox{
\begin{minipage}{60mm}
\texttt{\backslash{}usepackage[cm]\{aeguill\}}\\
\vdots\\
\texttt{\backslash{}usepackage\{frenchle\}}\\
\texttt{\backslash{}begin\{document\}}\\
\end{minipage}
}
\end{center}
\textbf{Remarque}  :
l'appel de l'extension \texttt{\backslash{}usepackage[cm]\{aeguill\}}\index{extension!aeguill@\verb'aeguill'}
 n'est \index{efrench@\textsl{\befr}} pas nécessaire si vous utilisez le codage T1\index{codage!T1} 
de fontes (\texttt{\backslash{}usepackage[T1]\{fontenc\}}).\\
\begin{center}
\begin{tabular}{l|l|}
\multicolumn{1}{c}{\textit{Texte source}}&\multicolumn{1}{c}{\textit{Composition avec frenchle}}\\\hline
\verb|frenchle est destiné à|&\rule{0pt}{2.5ex}frenchle est destiné à\\
\verb|\og franciser \fg\ les|&\og franciser \fg\ les\\
\verb|documents.|&documents\\
\verb|Les classes [...] sont|&Les classes [...] sont\\
\verb|\og calquées \fg\ |&\og calquées \fg\  \\
\verb|sur les classes|&sur les classes\\
\verb|standard. [...] Les classes|&standard. [...] Les classes\rule{1em}{0pt}\\
\verb|non-standard (et \emph{a|&non-standard (et \emph{a}\\
\texttt{fortiori\} les classes}&\emph{fortiori} les classes\\
\verb|\og \emph{maison} \fg)|&\og \emph{maison} \fg)\\
\verb|etc.|&etc.\\\cline{2-2}
\end{tabular}\\[2ex]
\end{center}
Ceci est une des mille et une façons d’accéder aux guillemets français. On
notera que dans le cas présent il me faut saisir un \texttt{\backslash\textvisiblespace}  après le \texttt{\backslash{}fg} s’il doit y
avoir effectivement un espace après le guillemet fermant.

\section{Ce que pourrait faire l'extension}\label{peutfaire}
Bien d’autres choses pourraient être ajoutées à \textit{frenchle} n’hésitez pas à faire
part de vos améliorations. Pour tester facilement ces améliorations vous pouvez
les inclure dans le fichier de configuration appelé \textit{frenchle}.cfg. Voici un
exemple permettant de saisir les guillemets (8bits, non utf-8) directement au clavier :
\begin{center}
\begin{boxedverbatim}
% Pour utiliser les guillemets 8-bits :
\catcode‘\« =\active\catcode‘\ »=\active
\def« {\og\ignorespaces}\def »{\fg}
\end{boxedverbatim}
\rule{0pt}{1ex}\\[2ex]
\begin{tabular}{l|l|}
\multicolumn{1}{c}{\textit{Texte source}}&\multicolumn{1}{c}{\textit{Composition avec frenchle}}\\\hline
\verb|frenchle est destiné à|&\rule{0pt}{2.5ex}frenchle est destiné à\\
\texttt{« franciser » les}&« franciser » les\\
\verb|documents.|&documents\\
\verb|Les classes [...] sont|&Les classes [...] sont\\
\texttt{« calquées » }&« calquées »  \\
\verb|sur les classes|&sur les classes\\
\verb|standard. [...] Les classes|&standard. [...] Les classes\rule{1em}{0pt}\\
\verb|non-standard (et \emph{a|&non-standard (et \emph{a}\\
\texttt{fortiori\} les classes}&\emph{fortiori} les classes\\
\texttt{«}\verb| \emph{maison}| \texttt{»})&\og \emph{maison} \fg)\\
\verb|etc.|&etc.\\\cline{2-2}
\end{tabular}\\[2ex]
\end{center}

L’avantage du fichier de configuration est que je n’ai rien à coder dans le
préambule du document et que ces modifications s’appliqueront automatiquement
à tous les documents utilisant \textit{frenchle} (avec une réserve dépendant de
l’emplacement exact du fichier \texttt{frenchle.cfg}\index{fichier!frenchle.cfg@\texttt{frenchle.cfg}}).

Cette action n'est pas possible dans le codage d'entrée UTF-8, mais \textsl{frenchle} et 
\textsl{french} du paquet \textsl{\befr} comprennnent correctement les guillemets 
« et » sans devoir les définir dans un fichier accesoire.

\section{Comment appeler \textit{frenchle} ?}
Plusieurs méthodes existent pour faire appel à \textit{frenchle}, en voici les trois
principales :
\begin{description}
\item[1\iere méthode :] demander le chargement de l’extension par l’ordre \verb|\usepackage| ;
dans ce cas placer cet ordre le plus près de l’ordre \verb|\begin{document}|\index{classe!de document} :\\[.5em]
\rule{0pt}{1em}\hfill\fbox{
\begin{minipage}{90mm}
 \backslash\texttt{documentclass}\{\textit{classe\_de\_documents}\}\\
 \backslash\texttt{usepackage}\{\textit{extension 1} \}\\
\rule{0.4\textwidth}{0pt} \vdots \\
 \backslash\texttt{usepackage}\{\textit{extension n-1} \}\\
 \backslash\texttt{usepackage}\{\texttt{frenchle}\}\\
 \backslash\texttt{begin\{document\}}
\end{minipage}
}\hfill\rule{0pt}{1em}\\[.5em]
\item[2\ieme méthode :] demander à ce que l’extension \texttt{babel}\index{extension!babel@\verb'babel'}
 charge \textit{frenchle} (on préfèrera
plutôt faire appel à l’extension \texttt{babelfr}) :\index{extension!babelfr@\texttt{babelfr}}\\[.5em]
\rule{0pt}{1em}\hfill\fbox{
\begin{minipage}{90mm}
 \backslash\texttt{documentclass}\{\textit{classe\_de\_documents}\}\\
 \backslash\texttt{usepackage}\{\textit{extension 1} \}\\
\rule{0.4\textwidth}{0pt} \vdots \\
 \backslash\texttt{usepackage}\{\textit{extension n-1} \}\\
 \backslash\texttt{usepackage[frenchle]}\{\texttt{babelfr}\}\\
 \backslash\texttt{begin\{document\}}
\end{minipage}
}\hfill\rule{0pt}{1em}\\[.5em]

\item[3\ieme méthode :]  demander à ce que toutes les extensions puissent éventuellement
bénéficier de \textit{frenchle}, y compris \texttt{babelfr}\index{extension!babelfr@\texttt{babelfr}} 
qui chargera automatiquement \index{classe!de document}
\textit{frenchle} :\\[.5em]
\rule{0pt}{1em}\hfill\fbox{
\begin{minipage}{100mm}
 \backslash\texttt{documentclass[french]}\{\textit{classe\_de\_documents}\}\\
 \backslash\texttt{usepackage}\{\textit{extension 1} \}\\
\rule{0.4\textwidth}{0pt} \vdots \\
 \backslash\texttt{usepackage}\{\textit{extension n-1} \}\\
 \backslash\texttt{usepackage\{babelfr}\}\\
 \backslash\texttt{begin\{document\}}
\end{minipage}
}\hfill\rule{0pt}{1em}\\[.5em]
On notera que nous codons \texttt{french} et non pas \texttt{frenchle} ; ceci est volontaire
car les extensions peuvent reconnaître des noms de langage (ici \textsl{french})
mais pas les noms des extensions (variés) qui définissent les actions. Dans
le cas présent l’option \textsl{french}\index{option french@option \verb'french'} sera interprétée au niveau de 
\texttt{babelfr}\index{extension!babelfr@\texttt{babelfr}} et \index{efrench@\textsl{\befr}}
donnera lieu soit au chargement de \textit{frenchle} soit à celui de \textsl{french} de la distribution \textsl{\befr} si celui-ci existe
dans votre installation.
\end{description}

Une quatrième méthode existe avec \textsl{\befr} en substituant \textsl{mlp}\index{extension!mlp@\textsl{mlp}} à l’extension
\texttt{babel}\index{extension!babel@\verb'babel'}, offrant alors d’autres possibilités mais cela n’est pas l’objet de
cette documentation.
%\pagebreak
\section{Messages}\label{messag}
Voici l’ensemble des messages\index{messages} émis par \textit{frenchle} avec leurs explications. Pour
aborder cette partie il est fortement conseillé d’avoir lu, au moins une fois, la
\linkandfootnote{lafaq}{FAQ}{http://www.efrench.org/distributions/faq.pdf}  
sur la francisation de \LaTeX.\\
\begin{verbatim}
-8a- \footnotetext{...} perdu.
-8b- Coder event. \protect\footnote.
\end{verbatim}
Signifie en général que vous avez utilisé une \backslash{}footnote dans un \backslash{}caption
de tableau. L’extension \textit{frenchle} ne pouvant mettre correctement le texte
de la note en bas de page (défaut actuel de \LaTeX), vous devez, vous
même, insérer la commande \texttt{\backslash{}footnotetext{...}} 
après le tableau.\\
\begin{verbatim}
-9- Corrupted/absent language.dat file.
\end{verbatim}
L’extension \textit{frenchle} vérifie à chaque exécution que le fichier
 \index{fichier!langage.dat@\texttt{language.dat}}\vers'language.dat'
 soit accessible et non-corrompu. Il est désormais chargé par \LaTeX automatiquement.\\
\begin{verbatim}
-13- le caractère "..." est déjà actif
     la double ponctuation est alors désactivée.
\end{verbatim}
Vous utilisez très probablement un style ou une extension qui fait déjà
usage de ce ou ces caractères. Pour éviter toute anomalie de fonctionnement,
l’extension \textit{frenchle} désactive alors l’effet de la double ponctuation
! : ; ? pour tout votre document. Si ce n’est pas ce que vous voulez,
essayez de charger l’extension en question, soit après le chargement de
l’extension \textit{frenchle} soit dans un environnement \texttt{english}.\\
\begin{verbatim}
-15- le langage frenchle porte le numéro ...
\end{verbatim}
L’extension \textit{frenchle} vous indique le numéro interne employé pour le langage
\textsl{french}. \\
\begin{verbatim}
-16- the English language is numbered ...
\end{verbatim}
Même explication que pour le message précédent mais portant ici sur
l’anglais.\\
\begin{verbatim}
-19- utilisation du langage interne numéro...
\end{verbatim}
Aucun langage \textsl{french} n’a été trouvé dans le fichier \texttt{language.dat}\index{fichier!langage.dat@\texttt{language.dat}}, 
dans ces conditions le
style \textit{frenchle} vous indique le numéro de langue interne qu’il va utiliser.

À vous de voir si ce numéro est acceptable pour la mise en page de votre
texte français.

Mais ceci ne devrait plus arriver avec les versions récentes de \LaTeX.\\
\begin{verbatim}
-20- WARNING: the french language is undefined in your format
     the french language is undefined (ERROR!)
\end{verbatim}
Vous ne pouvez utiliser \textit{frenchle} sans que votre moteur \LaTeX{} soit un minimum
francisé, c.-à-d. dispose par exemple de motifs de césure adaptés\index{césure}.
Le premier message n’est qu’un avertissement si le fichier de configuration
\texttt{language.dat}\index{fichier!langage.dat@\texttt{language.dat}} définit le français ; si ce n’est pas le cas le deuxième
message est émis.\\[2ex]
\texttt{-21-\backslash\textit{xxx}TeXmods n’est pas défini}.
Une demande a été faite pour travailler avec la langue \textit{xxx} mais celle-ci est
inconnue ou, tout au moins, la commande \backslash\textit{xxx}TeXmods n’est pas définie.
Probablement une erreur interne...\\
\begin{verbatim}
-23- Extension : style frenchle V... -- date -- (B.Gaulle)
\end{verbatim}
Ceci est la bannière du style \textit{frenchle}. Pensez à vous mettre à jour régulièrement.\\
\begin{verbatim}
-24- frenchle.sty utilise dans ce document le codage de fonte (O)T1.
\end{verbatim}
Ceci est un message d’information permettant de voir quel 
codage\index{OT1 (codage)@\verb'OT1' (codage)}\index{T1 (codage)@\verb'T1' (codage)} de
fonte a été détecté par \textit{frenchle} et sera utilisé pour tout le document\index{codage!utilisé}.
Vous avez toujours le loisir de changer de \index{codage!changement de}codage avant le chargement
initial de \textit{frenchle} de façon qu’il détecte celui qui convient à l’ensemble
des parties françaises du document.\\
\begin{verbatim}
-25- frenchle.sty affiche ici ses messages en 7-bits (à la TeX).
\end{verbatim}
Le format \LaTeX{} qui a été créé ne supporte pas le 8-bits en sortie. Cela
peut effectivement venir du moteur \TeX{} car tous ne disposent pas de cette
facilité. Si vous souhaitez toutefois afficher les messages en 8-bits, forcez
l’option \verb|\usualmessages|\index{usualmessages@\verb'\usualmessages'}.\\
\begin{verbatim}
-25- frenchle.sty affiche ici ses messages en 8-bits.
\end{verbatim}
Ceci est l’option normale si le moteur \TeX{} est capable de produire du 8-bits
 en sortie, à la place des caractères hexadécimaux sous la forme \chap\chap{}xx. 
Cette option peut avoir été forcée par la commande \backslash{}usualmessages.\\
\begin{verbatim}
-27- frenchle.sty language x (y) was initially
(at initex) numbered z (ERROR!)
\end{verbatim}
Signifie que l’ordre des langues a été probablement modifié dans le fichier
\texttt{language.dat} ou qu’il ne s’agit pas du bon fichier.
Ceci ne devrait plus arriver avec les versions récentes de \LaTeX,
puisqu'il n'est plus nécessaire de créer le format fr\LaTeX qui pourrrait entrer en conflit
avec les définitions déjà inscrites dans \texttt{langage.dat}.\\
\begin{verbatim}
-29a- ****Warning****: your TeX V3 engine + CM
     fonts (your format default) isn’t sufficient
     to hyphenate words containing diacritics (like in french).
\end{verbatim}
Ce que l’on appelle communément la césure des mots ne pourra jamais\index{césure}
être effective, dans ces conditions, sur des mots comportant des lettres
accentuées. Il s’agit là du plus grave défaut de francisation que vous obtiendrez
mais il y en a d’autres... Il serait peut-être bon de considérer
l’installation d’un moteur \TeX{} avec option Ml\TeX{} ou la mise en place,
par défaut dans le format\index{fontes!cm@\textit{cm}}, de polices de caractères 8-bits.
Quoique le fomat ne soit plus une solution à l'heure actuelle, voir le paragraphe \ref{lesformats}.\\
\begin{verbatim}
-34- this file and other auxiliary files require to
     use the following LaTeX packages: frenchle ...!
     check \usepackage or remove these files.
     Typesetting is aborted! 
\end{verbatim}
Vous avez dans un passage \LaTeX{} précédent utilisé une (ou plusieurs)
extension qui n’est plus demandée actuellement. Peut-être est-ce volontaire?
Dans ce cas il est préférable d’effacer les fichiers auxiliaires pouvant
contenir des informations relatives à cette extension. Sinon, il suffit de demander
le chargement de l’extension ad hoc.\\
\begin{verbatim}
-41- Format is out of date, please run initex again.
\end{verbatim}
Ce message ne devrait pas apparaître avec les {\em récentes} versions de \LaTeX.

L’extension \textit{frenchle} s’est aperçue d’une incohérence au niveau des commandes
de césure. Ces dernières correspondent à une installation ancienne\index{césure}
de la francisation qui les plaçait dans le format. Il était {\em auparavant} impératif de refaire
un format correct.\\
\begin{verbatim}
-42- The french patch file (frlpatch.sty) is not suitable
     for this version of the "frenchle"  package dated YY/MM/DD.
\end{verbatim}
Signifie qu’un fichier de \textit{patch}\index{fichier!frlpatch.sty@\texttt{frlpatch.sty}} a été trouvé dans le système mais qu’il ne
convient pas à la version de l’extension \textit{frenchle} que vous utilisez. Il est
nécessaire d’accorder l’un avec l’autre. Dans le doute vous pouvez toujours
renommer le fichier de patch pour qu’il ne soit pas trouvé.\\
\begin{verbatim}
-48- Lecture du fichier de configuration de frenchle.
\end{verbatim}
Un fichier de configuration \texttt{frenchle.cfg} a été détecté sur l’installation.
Il est lu et les ordres exécutables sont appliqués.\\
\begin{verbatim}
-51- ERREUR : ce document n’a pas été converti en 8-bits.
\end{verbatim}
Certains documents doivent être convertis en 8-bits avant utilisation. Selon
le type de document, la composition est arrêtée temporairement ou
définitivement. Ce message est réservé à des tests ; il ne devrait jamais
être émis à votre intention. Dans l'état de cette version et sous codage UTF8, ce test doit être supprimé.\\
\begin{verbatim}
-52- Error: the frenchle package doesn’t run in 
      such minimal document class, sorry!
\end{verbatim}
L’extension \textit{frenchle} ne peut fonctionner avec une classe de document
réduite, et en particulier avec la classe\index{classe!minimal@\verb'minimal'}%
\index{minimal@\verb'minimal'} \texttt{minimal}.\\
\begin{verbatim}
-58- Valeur de ... ignorée.
\end{verbatim}%\tthyphenation
Un environnement de liste est utilisé avec modification d’espacement, 
(\verb|\topsep, \partosep,| \verb|\itemsep|\index{itemsep@\verb'\itemsep'} ou \verb|\parsep|) alors que les espacement
verticaux sont imposés par \textit{frenchle}. Vous avez plusieurs solutions : 
soit vous abandonnez l’idée de les modifier (si cela vient de
vous) soit vous pouvez revenir aux espacements standard de \LaTeX{} avec\index{nofrenchtrivsep@\verb'\nofrenchtrivsep'}
\texttt{\backslash{}nofrenchtrivsep} soit enfin vous pouvez supprimer ce message avec
l’ordre \texttt{\backslash{}nofrenchtrivsepwarnings}\index{nofrenchtrivsepwarnings@\verb'\nofrenchtrivsepwarnings'}.\\
\begin{verbatim}
-68- ERROR: frenchle is no more running with 2.09 emulation, sorry!
\end{verbatim}
Vous utilisez probablement un très vieux document qui n’a pas été entièrement
reconverti pour \LaTeXe ; cette version de \textit{frenchle} ne peut être utilisée
dans ce cas.\\
\begin{verbatim}
-71- ATTENTION : si babel est utilisé, mettre french en option
\end{verbatim}
Vous avez probablement fait appel à babel\index{babel@\textit{babel}!french en@frenchle en option de}
 par une commande du genre\index{extension!babel@\verb'babel'}
\backslash{}usepackage[...]\{babel\} puis vous avez demandé à charger une extension
\textsl{french} (\texttt{efrench.ldf} ou \texttt{frenchle.ldf}), ce qui est incompatible. Soit
vous utilisez babel avec l’option \textit{efrench} ou \textit{frenchle} soit vous utilisez
une extension \textsl{french} toute seule. Il est probable que la composition du
document n’ira pas bien loin...\\
\begin{verbatim}
-73- ERREUR avec AmS\TeX{} : frenchle.sty a été chargé trop tôt.
\end{verbatim}
Il est indispensable de charger l’extension de francisation après AmS\TeX{}
de façon à ce qu’elle s’adapte au contexte.\\
\begin{verbatim}
-93- you are using two french options for
Babel, please choose one...
\end{verbatim}
Il se peut que vous ayez demandé le chargement de \textit{frenchle} par Babel
et en même temps que vous ayez donné à Babel l’option \textsl{french} (dans le
\backslash{}documentclass par exemple). Il vous faut alors soit éliminer une option
soit utiliser l’extension \texttt{babelfr}\index{extension!babelfr@\texttt{babelfr}} 
à la place de \texttt{babel}\index{extension!babel@\verb'babel'}.
\pagebreak
\section{Remerciements et contacts}
\paragraph*{Remerciements de Bernard \fsc{gaulle}}Ce document serait incomplet sans les 
remerciements\index{remerciements} aux nombreux contributeurs
(dont une liste exhaustive serait bien impossible à réaliser) qui ont
apporté leurs problèmes, leurs solutions ou leurs corrections ; c’est aussi grâce
à eux que \textit{frenchle} est maintenu à jour et enrichi régulièrement, je les en remercie
vivement et avec moi toute la communauté francophone des utilisateurs de
\LaTeX.

\paragraph*{Contacts}\label{contact}
En cas de problème, prière de \linkandfootnote{rayj}{me} 
{mailto:raymond@juillerat-marly-ch.net} contacter\index{contacts} (qui ai mis à jour ces pages)
ou \linkandfootnote{noustux}{<l'équipe eFrench>}{mailto:efrench@lists.tuxfamily.org}
Nous ferons de notre mieux pour vous aider.

\bibliographystyle{plain}
\ifx\Section\undefined\else\let\section\Section\fi% Restore \section.
\def\refname{Références bibliographiques}\index{références}
\begin{drapeaufg}
\begin{thebibliography}{9}\index{bibliographie}
\bibitem{ hautsdecasse}J. André \& J. Grimault, \textit{\linkandfootnote{rhautsdecasse}
{Emploi des capitales (première partie)}{http://cahiers.gutenberg.eu.org/cg-bin/article/CG_1990___6_42_0.pdf}}, 
\textit{in} Les Cahiers GUTenberg \Numero{} 6, 1990.
\bibitem{ codetypo}\textit{Code typographique}, Fédération de la communication , 17\ieme édition, 1995.
\bibitem{ division} J. Désarménien, \textit{La division par ordinateur des mots français : application
à \TeX}, {\it in} TSI vol. 5 {\Numero} 4, 1986.
%[4] D. Flipo, B. Gaulle et K. Vancauwenberghe,Motifs de césure français,
%in Les Cahiers GUTenberg N18, 1994.
%(http://www.gutenberg.eu.org/pub/GUTenberg/publicationsPDF/
%18-motifs.pdf)
\bibitem{cesures} D. Flipo, B. Gaulle et K. Vancauwenberghe,\textit{ 
\linkandfootnote{ rcesures}{Motifs de césure français}{http://cahiers.gutenberg.eu.org/cg-bin/article/CG_1994___18_35_0.pdf}}, 
{\it in} Les Cahiers GUTenberg \Numero 18, 1994\index{césure}.
%[5] B. Gaulle, FAQ \textsl{french}, foire aux questions à propos de \LaTeX{} en français,
%2007
%(http://frenchpro.free.fr/screen.pdf/FAQscreen.pdf)
\bibitem{faq} B. Gaulle, eFrench
\textit{\linkandfootnote{lafaq}{%
Foire aux questions à propos de \LaTeX{} en français}%
{http://www.efrench.org/distributions/faq.pdf}}, 2007
\bibitem{compagnon} M. Goossens, F. Mittelbach et A. Samarin, \textit{The \LaTeX{} companion},
Addison-Wesley, 1993.
\bibitem{guidrom} \textit{Guide du typographe romand}, Association suisse des compositeurs à la machine,
6\ieme~édition, 2000.
\bibitem{lamptex} L. Lamport, \textit{\LaTeX, A document preparation system}, Addison-Wesley, 1994.
\bibitem{lampndx} L. Lamport, \textit{An Index Processor For \LaTeX}, 1987.
\bibitem{imprnat} \textit{Lexique des règles typographiques en usage à l’Imprimerie nationale},
4\ieme~édition, 1991, ISBN 2-11-081075-0.
\bibitem{perrousmanu} Y. Perrousseaux, \textit{Manuel de typographie française élémentaire}, Atelier
Perrousseaux, 4\ieme~édition,1995, ISBN 2-911220-00-5.
\bibitem{perrousmepa} Y. Perrousseaux, \textit{Mise en page et impression}, Atelier Perrousseaux,
1996, ISBN 2-911220-01-3.
%[13] E. Saudrais, Le petit typographe rationnel, document électronique, 2007
%(http://perso.wanadoo.fr/eddie.saudrais/prepa/typo.pdf)
\bibitem{lepetity} E. Saudrais,
 \textit{\linkandfootnote{rlepetity}{Le petit typographe rationel}{http://tex.loria.fr/typographie/saudrais-typo.pdf}} 
\end{thebibliography}
\end{drapeaufg}
\vfill
\newpage
\nooverfullhboxmark% Pourquoi faut-il le repréciser pour la versions screen?
\tableofcontents
\def\footnote#1{}% nullify any footnote here.
\printindex
\end{document}
