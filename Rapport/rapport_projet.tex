\documentclass[12pt,a4paper]{article}
\usepackage[utf8]{inputenc}
\usepackage[french]{babel}
\usepackage[T1]{fontenc}
\usepackage{graphicx}
\usepackage{geometry}
\geometry{margin=2.5cm}

\title{Rapport de Projet - CY-Fighters}
\date{\today}
\author{Mattéo, Tom \& Myriam}

\begin{document}
\maketitle

\section{Introduction}
Ce rapport présente le travail réalisé par notre équipe dans le cadre du projet CY-Fighters. Notre équipe est composée de trois membres, chacun ayant des responsabilités spécifiques dans le développement du projet.

\section{Description du projet}
CY-Fighters est un jeu de combat au tour par tour permettant à un ou deux joueurs de faire s'affronter des équipes de combattants. Le projet consiste à développer un programme complet qui gère:

\begin{itemize}
    \item La création et la gestion d'équipes de combattants
    \item Un système de combat au tour par tour avec des mécaniques stratégiques
    \item Des combattants possédant diverses caractéristiques (points de vie, attaque, défense, agilité, vitesse)
    \item Des techniques spéciales avec effets variés (dégâts, soins, modifications de statistiques)
    \item Une interface utilisateur permettant de visualiser l'état du jeu et de prendre des décisions
    \item Un système de chargement des combattants et techniques à partir de fichiers
    \item Une intelligence artificielle avec différents niveaux de difficulté (optionnel)
\end{itemize}

Le combat se déroule jusqu'à ce qu'une équipe soit vaincue, c'est-à-dire lorsque tous ses combattants sont mis K.O. (points de vie à zéro ou moins). Chaque combattant peut effectuer une attaque classique ou utiliser une technique spéciale avec un temps de recharge, ajoutant ainsi une dimension stratégique au jeu.

\section{Présentation de l'équipe et répartition des tâches}
Notre équipe est composée de :
\begin{itemize}
    \item \textbf{Mattéo} : Responsable de l'interface graphique SDL.
    \item \textbf{Tom} : En charge de la gestion des combattants et de leurs techniques.
    \item \textbf{Myriam} : Responsable du système de combat et de l'interface terminale.
\end{itemize}

\section{Organisation et méthodologie de travail}
L'équipe s'est organisée en répartissant les tâches selon les compétences et les préférences de chacun. Cette répartition nous a permis de travailler en parallèle sur différents aspects du projet.

Nous avons utilisé Git comme système de gestion de versions pour faciliter la collaboration et le suivi des modifications. Des réunions régulières ont été organisées pour synchroniser notre travail et résoudre les problèmes rencontrés.

\section{Implémentation technique}
\subsection{Structure des données}
Nous avons implémenté les structures suivantes pour représenter les éléments du jeu:
\begin{itemize}
    \item \textbf{Combattant}: stocke les propriétés d'un personnage (points de vie, attaque, défense, agilité, vitesse) ainsi que ses techniques spéciales et effets actifs
    \item \textbf{Technique spéciale}: contient le nom, la valeur, la description, le nombre de tours actifs et le temps de recharge
    \item \textbf{Équipe}: regroupe plusieurs combattants sous un nom commun
    \item \textbf{Effet}: représente un effet temporaire appliqué à un combattant
\end{itemize}

\subsection{Système de combat}
Le système de combat implémente un mécanisme de tour par tour basé sur la vitesse des combattants. À chaque tour, le joueur peut choisir entre une attaque normale ou l'utilisation d'une technique spéciale disponible.

\subsection{Interface utilisateur}
Deux interfaces ont été développées:
\begin{itemize}
    \item Une interface terminale fonctionnelle permettant de jouer au jeu complet
    \item Une interface graphique en SDL offrant une expérience visuelle plus riche (non intégrée au système de combat)
\end{itemize}

\section{Problèmes rencontrés et solutions}
Au cours du développement, nous avons rencontré plusieurs difficultés :

\subsection{Problèmes de compilation}
Nous avons fait face à des problèmes de compilation qui ont ralenti notre progression. Ces difficultés ont nécessité une révision approfondie du code et des dépendances du projet.

\subsection{Intégration SDL et système de combat}
L'un des défis majeurs a été l'intégration de l'interface SDL avec le système de combat. Malgré nos efforts, nous n'avons pas réussi à faire fonctionner ensemble ces deux composantes du projet. Cette difficulté met en évidence l'importance d'une meilleure planification de l'architecture logicielle dès le début du projet.

\subsection{Gestion des effets temporaires}
La gestion des effets temporaires sur les combattants s'est avérée plus complexe que prévu, notamment pour appliquer et retirer correctement les modifications de statistiques au bon moment.

\section{Résultats obtenus}
Malgré les difficultés rencontrées, nous avons réussi à développer :
\begin{itemize}
    \item Une interface graphique en SDL fonctionnelle (mais qui ne fait pas tourner les combats)
    \item Un système de gestion des combattants et de leurs techniques.
    \item Un système de combat opérationnel avec interface terminale.
    \item Une sélection de huit combattants avec des caractéristiques et techniques variées.
    \item Un système de chargement des données depuis des fichiers.
\end{itemize}

\section{Perspectives d'amélioration}
Si nous avions plus de temps pour développer ce projet, nous aurions souhaité:
\begin{itemize}
    \item Finaliser l'intégration de l'interface SDL avec le système de combat
    \item Implémenter les différents niveaux d'intelligence artificielle pour le mode un joueur
    \item Ajouter des animations et effets visuels pour les attaques et techniques spéciales
    \item Développer un système de sauvegarde et chargement de parties
    \item Équilibrer davantage les combattants et leurs techniques
\end{itemize}

\section{Conclusion}
Ce projet nous a permis d'acquérir une expérience précieuse en développement collaboratif et en gestion de projet. Les difficultés rencontrées, notamment dans l'intégration des différents composants, nous ont fourni des enseignements importants pour nos futurs projets.

Le développement de CY-Fighters nous a également permis de mettre en pratique plusieurs concepts importants de programmation: structures de données, gestion de fichiers, interfaces utilisateur, et logique de jeu. Malgré les défis techniques, nous avons réussi à créer un jeu fonctionnel qui répond aux principales exigences du cahier des charges.

\end{document}
