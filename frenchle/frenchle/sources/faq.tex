% !TeX encoding = utf8
%This is FaQ.tex  : the LaTeX F A Q of the french style
%                       LPPL Copyright by eFrench
%
%
\documentclass[a4paper,12pt,openright]{article}
\usepackage[utf8]{inputenc}
\usepackage[T1]{fontenc}
\usepackage{times}
\usepackage{url}

\makeatletter\def\tempa{\let\if@screen\iffalse}\makeatother%
\tempa%
\def\href#1#2{#2}
\let\titcolor\relax
\usepackage[print]{pdfscreen}


\usepackage{french, mymaj, fancybox}

\usersfrenchoptions{%
  \disallowuchyph
  \overfullhboxmark
}

  \definecolor{section0}{rgb}{.722,.525,.431}
  \definecolor{section1}{rgb}{.650,.350, 0  }%{.937,.561,.123}
  \definecolor{section2}{rgb}{.750,.250, 0}%{.855,.647,.123}
  \definecolor{section3}{rgb}{.950,.050,.0}
  \definecolor{section4}{rgb}{.545,.271,.750}
  \definecolor{section5}{rgb}{.627,.322,.176}
  \definecolor{titcolor}{rgb}{.922,.725,.831}
  \definecolor{bleu}{rgb}{.0,.0,.650}
%\def\titcolor{\color{section0}}

\hypersetup{
  pdfwindowui=true, pdfnewwindow=true,
  colorlinks=true, pdfmenubar=true, pdffitwindow=true,
  anchorcolor=bleu, urlcolor=bleu,
}

\makeatletter
\newcommand{\linkandfootnote}[3]{%
             \href{#3}{#2}\bgroup%
      \expandafter\ifx\csname r@tlf.#1\endcsname\relax%
       \expandafter\global%
       \expandafter\let\csname r@tlf.#1\endcsname\empty%
                  \footnote{\texttt{<}\url{#3}\texttt{>}.\label{lf.#1}}%
                  \else\refmark{lf.#1}%
                  \fi\egroup%
                           }%
\makeatother

\newif\ifnoDOCinstall
\newif\ifnoFINALS

\textwidth=152mm
\textheight=240mm
\voffset -1in
%\hoffset 0in%-1in
\oddsidemargin=11mm
\evensidemargin=11mm
\topmargin=1.0cm
\footskip=1.3cm
\renewcommand{\ttdefault}{cmtt}
\hypersetup{
pdftitle={Foire aux questions },
pdfauthor={Bernard Gaulle, Raymond Juillerat},
pdfcreator={Bernard Gaulle},
pdfsubject={TeX, LaTeX et la langue française},
pdfkeywords={TeX, LaTeX, pdfLaTeX, efrench, frenchle.sty, french.sty},
}
%\makeindex
\begin{document}
\newlength{\oriparindent}
\setlength{\oriparindent}{\parindent}
\newcommand{\indneg}{\setlength{\parindent}{-2mm}}
\newcommand{\indpos}{\setlength{\parindent}{\oriparindent}}
\newcommand{\pgLapdTeX}
{(L\hspace{-.8ex}\raisebox{.4ex}{\scriptsize A}\hspace{-.3ex})\TeX}
%\disallowuchyph% C'est ma devise... (inclus dans myfroptn)

% Page de titre :
\makeatletter
\def\@fnsymbol#1{\ensuremath{\ifcase#1\or \star \or \diamond 
                              \or \heartsuit \else\@ctrerr\fi}}
\makeatother
\title{\hbox{}\vspace*{-3cm}
       \huge\bf
       FAQ \textit{eFrench}\\
       Foire aux questions\\ à propos de \LaTeX{ }%\textsl{eFrench}
%\footnote{Les marques de mise à jour dans la marge sont en rapport
%          avec la version V5,00 du 22 décembre 2000.}
% \\ pour \LaTeX
 en français}
\author{Bernard \fsc{Gaulle}, repris par Raymond \fsc{Juillerat}\\ 
et le groupe \textit{eFrench} \href{mailto:efrench(at)lists.tuxfamily.org}{\texttt{<efrench.org>}}%
\thanks{Attention, l'adresse m.él. fournie via les butineurs est
piégée ; il faut la corriger à la main.}\\%
LPPL ©  {\sc eFrench} (voir le fichier 
\linkandfootnote{copyright}{\texttt{Copyright.pdf}}
{http://efrench.org/}%
% ou \texttt{copyrigh.tex}
).\\
}
\date{version pro \frenchstyleid}\label{prempage}
\maketitle\thispagestyle{empty}

%\begin{resume} Cette notice décrit comment installer et utiliser avec \LaTeX{} 
%  l'extension \textsl{} (appelée autrefois style \texttt{french} puis \texttt{ French Pro}). 
%  Cette extension a été créée pour imprimer des documents typographiquement 
%  plus conformes à l'usage français que ce que produit \LaTeX{} par défaut.
%  Un grand nombre de commandes peuvent être utilisées mais
%  l'emploi courant de cette extension ne nécessite 
%  {\em a priori\/} aucune connaissance particulière
%  ni une utilisation forcenée de commandes spécifiques. Toutefois cet
%  emploi n'est pas toujours {\em transparent}. 
%\maj
%  Une version allégée est fournie (\texttt{frenchle}).
%Une version appauvrie (\texttt{pmfrench})
%  est aussi utilisable sur tous les sites et dans toutes les
%  configurations. L'installation de \textsl{eFrench} s'accompagne d'une
%  création de {\em format} pour l'introduction, notamment, des
%  fichiers de motifs de césure français. D'autres extensions l'accompagnent.
%\end{resume}
\vfill\vfill
\section*{Préface}
Cette publication est une version modifiée sur la base d'un document \textit{faq.pdf}
créé par Bernard \textsc{Gaulle} et mis à jour par lui une dernière fois le 28 juin 2007.
Cette FAQ était à l'origine un des nombreux documents qui accompagnaient 
cette création complète de typographie française appelée \textit{FrenchPro}.

Son auteur, Bernard \textsc{Gaulle}, est décédé en août 2007. En novembre 2007, pour
ne pas laisser tomber dans l'oubli ce grand-œuvre et 
laisser à disposition des auteurs francophones ces outils incomparables, Laurent \textsc{Bloch}
 a lancé le projet nommé \textit{eFrench}\cite{lblochefr}.

Parallèlement, les environnements \TeX{} et \LaTeX{} ont évolué jusqu'à aujourd'hui vers une simplification pour
l'utilisateur \textit{lambda}, 
avec le résultat que certaines explications ne sont plus du tout d'actualité,
les manipulations évoquées ayant été automatisées depuis, 
en particulier en ce qui concerne les moteurs
les plus courants (voir § \ref{Qpdftex} et \ref{motfrahup}).

C'est surtout pour la mettre à jour que cette FAQ a été revue et corrigée. Les modifications sont marquées par une barre dans la marge gauche.
\vfill
\newpage
\sommaire[3]
\noindent

{\sc eFrench}  © LPPL (voir le fichier 
\linkandfootnote{copyright}{\texttt{Copyright.pdf}}
{http://efrench.org/}%
% ou \texttt{copyrigh.tex}
).\\
%% \NNL sert a éliminer les saut de page dans la toc ou le sommaire
%% mais cela génère de messages « Token not allowed » avec hyperref.
\newcommand{\NNL}{\protect\let\\\relax}
\makeatletter\def\HyPsd@CatcodeWarning#1{}\makeatother%
%%
\let\Subsubsection\subsubsection
\def\subsubsection#1{\Subsubsection[\NNL#1]{\protect\color{section3}#1}}
\let\Subsection\subsection
\def\subsection#1{\Subsection[\NNL#1]{\protect\color{section2}#1}}
\let\Section\section
\def\section#1{\Section[\NNL#1]{\protect\color{section1}#1}}

\section{Quelques notions de base}

Pour mieux répondre à vos interrogations les réponses sont éclatées en
deux parties : d'une part, l'explication textuelle voulue la plus brève
possible et d'autre part, l'application \TeX{}nique qui illustre le propos ou le
complète.
%? Copyright 1996-2007
\subsection {Qu'est-ce que {\TeX} ?}
\indneg

{\sl \small En bref :} Si vous ne connaissez pas 
\linkandfootnote{TeX}{\TeX}
{http://www.tug.org/whatis.html}
 (prononcez « \textit{tek} ») sachez que c'est
un programme génial dont l'auteur est 
\linkandfootnote{DKnuth}{Donald \sc Knuth}
{http://www-cs-faculty.stanford.edu/~knuth}
%Donald Knuth2 
et qui permet de
composer, mettre en page des documents à imprimer ou à visualiser. Le
document source est fourni sous forme de texte balisé. Celui que vous lisez
contient, par exemple, la balise \texttt{$\backslash$title\{FAQ} ...\texttt{\}}. Ces balises ont un rôle
important au moment de la composition par \TeX{} car, suivant la classe de
document employée, la typographie appliquée sera différente.\indpos

On emploie souvent, par erreur, le mot \TeX{} à la place de celui de \LaTeX{}
(voir ci-après) ou \textit{vice versa}.

\indneg{\TeX\sl nique :} \indpos  {\TeX} produit en sortie un fichier « \texttt{dvi} » indépendant de l'unité de
restitution.

\begin{MAJ}Mais aujourd'hui, le format « \texttt{dvi} » fait de plus en plus place au 
format  « \texttt{pdf} » et c'est plutôt {pdf\TeX} que {\TeX} ou mieux encore
plutôt {pdf\LaTeX} que {\LaTeX} qui sont utilisés.
\end{MAJ}

\subsection {Qu'est-ce que \LaTeX{} ?}
\indneg{\sl \small En bref :} 
\linkandfootnote{LaTeX}{\LaTeX}
{http://www.latex-project.org/}
%LATEX3
 (prononcez « \textit{la tek} ») est un langage de balisage qui étend
les possibilités de \TeX; il propose des balises extrêmement puissantes. C'est
\LaTeX{} qui est le plus utilisé dans le monde ; il s'agit d'un standard de fait,
\maj aujourd'hui supplanté par sa version nettement plus étendue {pdf\LaTeX}.

\indneg{\TeX\sl nique :} \indpos La classe de document employée pour ce document a été la classe
\texttt{article} ; elle a été précisée au moyen de la balise
\begin{center}
\fbox{\makebox[0.7\textwidth][l]{$\backslash$\texttt{documentclass} \it\{classe\_de\_documents\} }}%\usebox{\mybox}}}
\end{center}\indpos
dans ce document-ci donc \{classe\_de\_documents\} = \texttt{article}.

Une balise \LaTeX{} est en fait une macro-instruction faisant appel à des commandes
\TeX.\indpos

La plupart des compilateurs {\TeX} sont désormais basés sur la distribution
générique (sous forme source)
\linkandfootnote{ web2c}{web2c}
{http://www.tug.org/web2c}
% web2c4
\maj et la base multilingue \textit{babel}.
On emploie souvent, par erreur, le mot \LaTeX{} à la place de celui de \TeX{}
ou \textit{vice versa}.
\subsection {Qu'est-ce qu'un format {\TeX} ?}\label{quoiFormat}
\indneg{\sl \small En bref :} Le programme \TeX, lorsqu'il démarre, a besoin d'un ensemble de
données qui ont été préparées à l'installation du produit ; c'est ce que l'on
appelle le « \textit{format} ». Il contient notamment la définition de toutes les balises
mises à disposition de l'utilisateur. Le format par défaut de {\TeX} s'appelle
\texttt{plain} mais il est souvent appelé sous le nom plus normal de \texttt{tex} ; celui de
\LaTeX{} s'appelle \texttt{latex}. \begin{MAJ}
Par contre les formats francisés étaient généralement
appelés, respectivement, \texttt{frtex} et \texttt{frlatex}, mais ils ne sont plus 
aujourd'hui nécessaires, ce qui simplifie l'installation.%lasse_de_documents
\end{MAJ}

\indneg{\TeX\sl nique :}   Pour créer un format on exécute un « initex » :
%\usepackage{mltex}
\begin{center}
\fbox{\makebox[0.7\textwidth][l]{\texttt{tex -ini} \it nom\_du\_format . . .}}
\end{center}
\setlength{\parindent}{\oriparindent}

%\TeX -ini nom\_du\_format . . .
\subsection{ Qu'est-ce que {pdf\TeX} ; {pdf\LaTeX}  ?}\label{Qpdftex}
\setlength{\parindent}{-2mm}{\sl \small En bref :} Le moteur 
\linkandfootnote{pdfTeX}{pdf\TeX}
{http://www.tug.org/applications/pdftex/}
%pdfTEX5
 est un moteur {\TeX} standard produisant, par
défaut, un fichier de type Adobe PDF. Il peut aussi fonctionner à l'identique
de \TeX. Son nom de format commence en général par \texttt{pdf}.  \\
\begin{MAJ}
%Pour les formats
%francisés on utilisait en général les noms \texttt{frpdftex} et \texttt{frpdflatex},\begin{MAJ}
%mais ce formatage n'est plus nécessaire aujourd'hui,
%n'apportant aucun avantage.

La différence entre pdf\TeX{} et pdf\LaTeX{} est la même qu'entre 
\TeX{} et \LaTeX{} au niveau de la saisie. 
Les moteurs pdf\TeX et pdf\LaTeX{} peuvent produire en sortie un fichier de type « \texttt{pdf} » ou « \texttt{dvi} ».
\end{MAJ}

\subsection{ Qu'est-ce que Ml\TeX{} ?}\label{qmltex}
\indneg{\sl \small En bref :} 
\linkandfootnote{MlTeX}{Ml\TeX}
{http://www.ctan.org/tex-archive/systems/generic/mltex}
%Ml\TeX6 
ou MultiLingual {\TeX} est une option (\texttt{-mltex}), utilisable
avec tous les moteurs {\TeX} basés sur 
\linkandfootnote{ web2c}{web2c}%
{http://www.tug.org/web2c}%
%web2c4
, qui avait été réalisée à l'origine
par \textsc{Michael Ferguson} pour résoudre divers problèmes linguistiques dont
celui de la césure de mots accentués.
\linkandfootnote{DKnuth}{Donald \sc KNUTH}%
{http://www-cs-faculty.stanford.edu/~knuth}
% Donald Knuth2 
a repris un certain
nombre d'idées dans la version 3 de \TeX{} mais pas toutes. Le suivi et la
mise à jour de {Ml\TeX} était assurée par 
%\linkandfootnote{BReichle}{\sc Bernd Raichle}
%{http://www.informatik.uni-stuttgart.de/ifi/is/Personen/raichle.html}
Bernd {\sc Raichle}.
Pour
des raisons de Copyright, les moteurs {\TeX} modifiés pour offrir cette option
ne peuvent être appelés \TeX. On trouvera ainsi, pour \textsc{Unix} :
\linkandfootnote{ teTEX}{te\TeX}
{http://www.tug.org/tetex/}%
%teTEX8
, pour MacOs :
\linkandfootnote{ CMacTEX}{CMac\TeX}
{http://www.kiffe.com/cmactex.html}%
%CMacTEX9
, pour Windows (win32) : 
\linkandfootnote{ fpTEX}{fp\TeX}
{http://www.fptex.org/}%\bf
%fpTEX10
, etc.
Cette option est choisie à l'installation et plus précisément à la création du
format ; elle permet donc ensuite la césure automatique des mots accentués
en environnement standard.

\indneg\TeX{\sl nique} : {Ml\TeX} étant devenu une option standard d'exécution des compilateurs
\TeX{} basés sur 
\linkandfootnote{ web2c}{web2c}{http://www.tug.org/web2c}%
%web2c4 
, il suffit de coder cette option à la création du
format :
\begin{center}
\fbox{\makebox[0.6\textwidth][l]{\texttt{tex -ini -mltex} \it nom\_du\_format . . .}}
\end{center}\indpos
Il existe aussi une extension de \LaTeX, 
\linkandfootnote{MlTeX}{\tt mltex}{http://www.ctan.org/tex-archive/systems/generic/mltex}
%mltex6 , 
, que l'on peut charger lorsque
le format (fabriqué avec l'option « \texttt{-mltex} ») ne contient pas de descriptif
des caractères accentués contrairement à ce qui était conseillé par \textit{FrenchPro}.
\begin{center}
\fbox{\makebox[0.4\textwidth][l]{$\backslash$\texttt{usepackage\{mltex\}}}}
\end{center}

\begin{MAJ}
Somme toute, {Ml\TeX} était une façon de parer aux défauts des fontes de type \texttt{OT1}.
Avec \textit {eFrench} on conseille plutôt de passer aux fontes de type \texttt{T1} sans Ml\TeX{}.
\end{MAJ}
%%\usepackage{mltex}
%5. <>.
%6. <>.
%7. <>.
%8. <>.
%9. <>.
%10. <>.
%3
\subsection{ Qu'est-ce que TeX--XeT ? }\label{qtexxet}
\indneg{\sl \small En bref :} TeX--XeT ou « {\TeX} de droite-à-gauche » est une option utilisable avec
quelques moteurs \TeX{} qui a été réalisée par Peter {\sc Bretenlohner} pour résoudre
le problème de la mise en page dans les langues écrites de droite à
gauche. Donald \textsc{Knuth} avait étudié la question avec Pierre \textsc{Mackay} mais
était arrivé à une version incompatible avec \TeX. La version proposée par
\textsc{Bretenlohner} n'a pas d'impact sur le compilateur \TeX{} et est donc optionnelle.
Pour des raisons de Copyright, les moteurs \TeX{} modifiés pour offrir
cette option ne peuvent être appelé \TeX. On trouvera ainsi, pour MacOs : 
\linkandfootnote{DirectTEX}{Direct\TeX}{http://www.tex.ac.uk/tex-archive/nonfree/systems/mac/directtex/dtmanual.pdf}%
%DirectTEX11
, pour Windows : l'ancien G{\TeX} qui était proposé par l'association
\linkandfootnote{GUTenberg}{GUTenberg}{http://www.gutenberg.eu.org}
%GUTenberg 12 
avec la distribution WinGUT, etc.\indpos

\begin{MAJ}
Aujourd'hui, TeX-XeT est intégré à pdf\LaTeX{}. 
Pour avoir l'option « \TeX{} de droite-à-gauche », il faut l'activer en définissant
\texttt{$\backslash$TeXXeTstate=1}.
L'effet est de rendre ainsi actives les balises 
\texttt{$\backslash$beginL} \textit{pour le texte de gauche à droite} \texttt{$\backslash$endL}, ainsi que 
\texttt{$\backslash$beginR} \textit{pour le texte de droite à gauche} \texttt{$\backslash$endR}.
Ces quatre balises sont celles qui ont été définies dans TeX-XeT.
\end{MAJ}
\subsection{ Fontes et codages}\label{fontcoda}
\indneg{\sl \small En bref :} D'une fonte à l'autre, d'un producteur à l'autre, 
les caractères ne sont
que rarement à la même place ; ce que l'on exprime en disant : les codages de
fontes sont différents. C'est d'ailleurs le cas pour les fontes par défaut dans
\TeX, les {\it cm}, avec lesquelles on obtient des images de caractères ({\it glyphes})
différentes en commutant, par exemple, au style \texttt{tt}. Le codage très particulier
des {\it cm} est appelé \texttt{OT1} dans \LaTeX. Par défaut  \LaTeX{}  fonctionne avec ce
codage ; il n'est donc pas nécessaire de lui préciser quoi que ce soit si on
l'utilise ainsi.
Depuis la création de la norme dite de Cork, de nouvelles fontes ont été
créées, telles les \textit{ec} dont le codage est appelé \texttt{T1} dans \LaTeX. Dans ce codage
figurent les glyphes de tous les caractères composés français. Quelques polices
 de caractères (de plus en plus) \maj sont codées selon cette norme.

\indneg{\TeX\sl nique :} 
Si on fait appel à des fontes codées \texttt{T1} il faut le préciser à \LaTeX{} :
\begin{center}
\fbox{\makebox[0.5\textwidth][l] {\texttt{$\backslash$usepackage[T1]\{fontenc\}}}}
\end{center}%%\usepackage[T1]{fontenc}
Dans ce cas \LaTeX{} fera appel, par défaut, aux fontes \textit{ec} pour le mode texte
et aux fontes \textit{cm} pour le mode mathématique.
Les fontes PostScript peuvent être recodées en \texttt{T1} (voir notamment les
logiciels
\linkandfootnote{fontinst}{fontinst}{http://www.tug.org/applications/fontinst/}% 
%fontinst13 1.3
ou 
\linkandfootnote{dvi2ps}{dvi2ps}{http://www.tug.orgpermet d'uti/dvipsk/}%
%dvi2ps14 1.3
ou 
\linkandfootnote{afm2tfm}{afm2tfm}{http://www.tug.org/texinfohtml/dvips.html\#Makink-a-font-available}%
%afm2tfm15
) 
grâce  au système de fontes virtuelles.
\subsubsection{Fontes virtuelles}
\indneg{\sl \small En bref :} Certaines fontes sont, en effet, virtuelles parce qu'elles n'existent pas.
Dans la pratique cela veut dire que l'on fait appel, sous ce nom et à la place
%11. <>.
%12. <>.
%13. <>.
%14. <>.
%15. <>.
%4
d'une fonte virtuelle, à d'autres fontes, bien réelles celles-ci. Pendant toute
une période où les fontes \emph{ec} n'existaient pas, les français ont utilisé des fontes
virtuelles (\textit{dm}) qui permettaient d'accéder à des caractères accentués et ainsi
de disposer d'une césure française correcte. Malheureusement ces fontes \textit{dm}
ne répondent à aucun standard et sont bien incomplètes, il n'est donc plus
très conseillé de les utiliser. Il existe aussi les fontes virtuelles \textit{ae} qui respectent
le standard mais n'offrent pas la panoplie complète des glyphes.\indpos

La fonte virtuelle 
\linkandfootnote{fourier-GUT}{fourier-GUT}{http://ctan.tug.org/tex-archive/fonts/}
%fourier-GUT16
 fournit un ensemble complet pour le
français :
\begin{center}
\fbox{\makebox[0.5\textwidth][l]{\texttt{$\backslash$usepackage\{fourier\}}}}
\end{center}\indpos
%%\usepackage{fourier}

La société 
\linkandfootnote{YandY}{Y\&Y}{http://tug.org/yandy/}
%Y\&Y17
a développé des fontes {\it em} qui adoptent encore un autre codage : LY1.\indpos

L'extension 
\linkandfootnote{MlTeX}{\tt mltex}{http://www.ctan.org/tex-archive/systems/generic/mltex}
%mltex6 
pour \LaTeX{} utilise, de façon interne, un autre codage : LO1.
\subsection{ Qu'est-ce que TDS ?}
\indneg{\sl \small En bref :} 
\linkandfootnote{TDS}{TDS}{http://www.tug.org/tds}
%TDS18 
est une norme pour l'arborescence des fichiers utilisés dans
une installation \TeX. La structure proposée est commune à toutes les plate-formes
matérielles. On peut ainsi partager une même hiérarchie de fichiers de
données (styles, fontes, etc.) ou plusieurs entre différents systèmes d'exploitation,
tels \textsc{Unix}, Windows ou MacOS. Les formats (voir paragr. \ref{quoiFormat}) générés
à partir de compilateurs {\TeX} basés sur 
\linkandfootnote{ web2c}{web2c}{http://www.tug.org/web2c}
%web2c4 
sont même utilisables, en
principe, depuis n'importe lequel des systèmes d'exploitations ; ils sont donc
indépendants de toute plate-forme logicielle.

\indneg{\TeX\sl nique :} la hiérarchie de fichiers est appelée « \texttt{texmf} ». Il en existe au moins
une que l'on ne modifie jamais \texttt{(TEXMFMAIN=texmf)} car elle correspond à la
distribution officielle et une autre, locale \texttt{(TEXMFLOCAL=texmf-local)}  que
l'on peut modifier à volonté. En cas de difficulté de gestion de ces hiérarchies
{\TeX} vous pouvez consulter avec intérêt le manuel d'%
\linkandfootnote{administration}{administration}
{http://daniel.flipo.free.fr/doc/tex-admin/TeX-admin.pdf}{ }
%administration19 
de Daniel Flipo.
\section{(La)\TeX{} et le français}
\indneg{\sl \small En bref :} qu'appelle-t-on  {(La)\TeX} ou \pgLapdTeX{} ?
Ce logo à mi-chemin entre 
\linkandfootnote{TeX}{\TeX}
{http://www.tug.org/whatis.html}{ }
et \linkandfootnote{LaTeX}{\LaTeX}
{http://www.latex-project.org/}
précise que l'on parle de \TeX{} comme de \LaTeX{} ou de tout 
\linkandfootnote{autres}{autre}
{http://www.ctan.org/tex-archive/systems/}{ }
%autre 20 
ensemble de macro-instructions faisant appel à un moteur \TeX.
%16. </>.
%17. <>.
%18. <>.
%19. <>.
%20. <>.
%5
\subsection{Est-ce que ça fonctionne bien en français?}
\indneg{\sl \small En bref :} Quel que soit le logiciel de traitement de texte, divers dispositifs sont
nécessaires et doivent correspondre à l'usage français : la césure des mots, les
caractères composés, la typographie, les usages des éditeurs, etc. Ces dispositifs
existent potentiellement dans \hbox{\pgLapdTeX} mais d'autres, tels que la correction
orthographique, ne peuvent pas en faire partie car \TeX{} est juste un compilateur
(de programme informatique) ; il n'agit pas au moment de la saisie.
Les fonctions de ce genre font partie des éditeurs (de fichiers ou de textes)
ou des traitements de textes usuels. Bien que {\pgLapdTeX} intègre l'essentiel des
dispositifs, certaines précautions d'usage ou certains préalables d'installation
(décrits dans ce document) sont nécessaires. Moyennant quoi {\pgLapdTeX} fonctionne
admirablement bien en français.
\subsection{ Césure et caractères composés}
\indneg{\sl \small En bref :} {\TeX} a été conçu avec une petite faiblesse : son incapacité à couper
(en bout de ligne) les mots contenant la primitive 
\verb!\accent! 
qui est utilisée
à chaque fois que l'on désire imprimer une lettre accentuée. Heureusement
divers contournements existent ; les deux principaux sont les suivants :

\begin{itemize}
\item l'utilisation de {Ml\TeX} (voir paragr. \ref{qmltex}) ;
\item l'utilisation de polices de caractères contenant tous les caractères composés 
nécessaires au français : les fontes \textit{ec} ou les fontes \textit{lmodern}, au
codage \texttt{T1}, répondent à ce critère (mais il en existe bien d'autres).
\end{itemize}\indpos

À condition d'avoir choisi l'une de ces deux possibilités, \LaTeX{} offre alors,
à ce niveau, tous les dispositifs nécessaires au français ; il faut éventuellement
lui préciser aussi le codage du texte d'entrée (voir paragr. \ref{inenco}) et celui des
fontes (voir paragr. \ref{soncodent}).

\indneg{\TeX\sl nique :} rappelons que ces codages sont précisés comme suit :
\begin{center}
\fbox{\parbox{0.6\textwidth}
{\texttt{$\backslash$usepackage[}\textit{codage\_d'entrée}\texttt{]\{inputenc\}}\\[.5ex]
\texttt{$\backslash$usepackage[}\textit{codage\_des\_fontes}\texttt{]\{fontenc\}}}}
\end{center}

\subsection{ Caractères français}
\indneg{\sl \small En bref :} Celui qui publie en français utilise un jeu de caractères adapté à la langue. Chaque
caractère a une position alphabétique. Lorsque l'on réalise des index et des
bibliographies, il est d'usage de respecter l'ordre alphabétique français.
On pourra consulter le 
 \linkandfootnote{docja}{document de Jacques André}
{http://jacques-andre.fr/faqtypo/unicode/alpha-fr.pdf}  sur le sujet.
%21. <>.
%6

\indneg{\TeX\sl nique :} Voici donc le jeu de caractères exact dans l'ordre alphabétique :\\
\rule{0pt}{1ex}\\
\fbox{\begin{tabular}{*{21}{c@{\hspace{1.4ex}}}}
a&A&à&À&â&Â&æ&Æ&b&B&c&C&ç&Ç&d&D&e&E&é&É&è\\
È&ê&Ê&ë&Ë&f&F&g&G&h&H&i&I&î&Î&ï&Ï&j&J&k&K\\
l&L&m&M&n&N&o&O&ö&Ö&œ&Œ&p&P&q&Q&r&R&s&S&t\\
T&u&U&ù&Ù&û&Û&ü&Ü&v&V&w&W&x&X&y&Y&ÿ&Ÿ&z&Z\\
\end{tabular}}
\subsection{Motifs français de césure}\label{motfrahup}
\indneg{\sl \small En bref :} {\TeX} peut couper les mots français selon l'usage si son format (voir
paragr. \ref{quoiFormat}) qui a été créé avec \texttt{initex} 
ou \texttt{virtex} contient une table des
motifs français de césure. L'association GUTenberg est le garant de la validité
de cette table.\indpos

Le fichier correspondant 
 \linkandfootnote{frhyphGut}{\tt frhyph.tex}
{http://ftp.gutenberg.eu.org/pub/GUTenberg/hyphenations/}
%(frhyph.tex) 22 
est utilisé dans toutes les distributions
francisées ; il est aussi accessible sur les serveurs
\linkandfootnote{frhyphCtan}{\tt ctan}
{http://ctan.tug.org/tex-archive/language/hyphenation/frhyph.tex}
% ctan23 
 Mais dans la distribution {\it eFrench} est aussi incluse
une extension en $\beta$test de \texttt{frhyph.tex},
nommée \linkandfootnote{frhyph1}{\texttt{frhyph1.tex}}
{http://www.efrench.org/hyphen/frhyph1.tex}
   et proposée par  Bernard Gaulle.

Ces motifs français de césure ont été mis au point tout au long des années ;
ils sont actuellement le meilleur compromis pour appliquer une coupure correcte
de mots dans un document français. Il est donc fortement conseillé de
faire confiance à ces motifs de césure et de n'appliquer qu’exceptionnellement
des modifications de césure ponctuelles dans le document lui-même ou dans
un fichier séparé, mais jamais dans les fichiers officiels précités.

Pour mieux connaître et comprendre ces motifs de césure on consultera
l'article \cite{cesures} publié dans les Cahiers GUTenberg.


\indneg{\TeX\sl nique :} Le fichier des motifs français de césure est chargé automatiquement,\begin{MAJ}
à l'installation de la langue \texttt{french} dans le moteur (Mik\TeX{} ou \TeX Live, par exemple).
Ceci se fait en deux phases :
\begin{enumerate}
\item le chargement dans le système (Mik\TeX{} , \TeX Live, ...) d'une langue, que ce soit 
\texttt{french} ou une autre, adapte le fichier \texttt{language.dat} en y incorporant
les dénominations les fichiers de césure correspondants.
\item la partie de \textit{ babel} (voir paragr. \ref{forbabel}) qui est est incorporée
au moteur {\LaTeX} charge les fichier de césure selon le contenu de \texttt{language.dat},
sans que \textit{ babel} soit invoqué explicitement.
\end{enumerate}
Remarquons que c'est aussi à cause de ce mécanisme qu'il n'est plus 
nécessaire de créer les formats \texttt{fr}...\texttt{tex} .
\end{MAJ}
%la création du format (voir paragr. 1.3) soit par hyconfig.tex25 de la
%distribution FrenchPro 24 ou 
%e fichier de configuration language.dat26 :
%22. <>.
%23. <>.
%24. <http://frenchpro6.com/frenchpro/french/initex/>.
%25. <http://frenchpro6.com/frenchpro/french/initex/>.
%7
%le fichier de configuration language.dat26 :
Voici à titre documentaire la partie de \texttt{language.dat} définissant les césures de la langue française :\\[0.5ex]
\begin{minipage}{12cm}
ici précédée par l'anglais \\[0.5ex]
\fbox{
\begin{minipage}{11cm}
\tt
\%==============================================/ \\
\% Name~~~| patterns ~| exceptions | ~ encoding / \\
\%==============================================/ \\
usenglish ushyph.tex \% this is the original ~ ~/ \\
\rule{0pt}{1pt} ~ ~ ~ ~ ~ ~ ~ ~ ~ ~ \% Don. Knuth's hyphenation/ \\
=english ~enhyph.tex ~enhyphex.tex \%  ~ ~ ~ ~ ~ / \\
french ~~ frhyph.tex ~frhyphex.tex \%  ~ ~ ~ ~  ~ / \\
=patois ~ \% ~ ~ ~ ~ ~\%  ~ ~ ~ ~ ~ ~ ~ ~ ~ ~ ~ ~ / \\
=acadien ~\% ~ ~ ~ ~ ~\%  ~ ~ ~ ~ ~ ~ ~ ~ ~ ~ ~  ~ / \\
 german ~ ~dehypht.tex\% traditional german ~ ~ ~/\\
\vdots
\end{minipage}\\[1ex]
}\\[1ex]
et suivie de l'allemand traditionnel et d'autres langues ...
\end{minipage}
\indpos
%...celui de

Il est à noter que la césure d'un mot est fortement liée à la fonte utilisée
et aussi au codage (voir paragr. \ref{fontcoda}) de cette fonte. Heureusement le fichier
des motifs français est indépendant des codages de fontes usuels, tels que \texttt{T1}
 (préférable) et \texttt{OT1}.
\subsection{ Faut-il saisir avec un clavier français ?}
\indneg{\sl \small En bref :} Le clavier doit, avant tout, vous être familier. 
Que la disposition des touches  soit française
(\texttt{AZERTY}) ou autre (en général \texttt{QWERTY}, en Suisse \texttt{QWERTZ}) 
n'est pas vraiment un problème pour
\pgLapdTeX. Il est souhaitable que l'éditeur de document utilisé facilite la saisie
des éléments français répétitifs dans vos textes. C'est le cas des caractères
composés. Si vous ne disposez pas de tous les caractères composés au clavier,
il faut alors que l'éditeur accepte des combinaisons de touches permettant de
les afficher à l'écran sous leur forme normale. 
%Si ce n'est pas le cas vous pouvez
%configurer cela vous même, facilement, grâce à l'extension keyboard (voir
%paragr. 2.7) de FrenchPro.

\indpos Avec \LaTeX, par défaut, vous êtes supposé saisir de l'{\sc ascii} (7-bits), ce qui
correspond aux possibilités du clavier \texttt{QWERTY} des anglo-saxons. Comme il y a
de fortes chances pour que ce ne soit pas votre cas ou du moins que vous ne
le souhaitiez pas, vous devrez le préciser en nommant votre codage d'entrée
(voir ci-après).
\subsection{Quel codage d'entrée faut-il préciser à \LaTeX?}
\label{inenco}
\indneg{\sl \small En bref :} Le codage d'entrée de votre texte saisi ({\it input encoding} en anglais)
peut être spécifié au moyen de l'extension  \textit{inputenc}
(standard \LaTeX).
% ou keyboard (voir paragr. 2.7) de FrenchPro. L'intérêt de
%l'un ou de l'autre est différent (voir paragraphe suivant).
%26. <http://frenchpro6.com/frenchpro/french/inputs/french/>.
%8

\indneg{\TeX\sl nique :} Par défaut, \LaTeX{} est chargé avec le codage d'entrée {\sc ascii} de la façon
suivante :\indpos
\begin{center}
\fbox{\parbox{0.5\textwidth}
{\texttt{$\backslash$usepackage[ascii]\{inputenc\}}}}
\end{center}
%\usepackage{inputenc}
\begin{MAJ}

Le codage \textsc{utf-8} est aujourd'hui présent pratiquement sur tous les systèmes.
Afin de rendre vos fichiers compatibles sur plusieurs machines,
c'est la meilleure solution.
Vous choisissez alors pour vos sources \pgLapdTeX ce code :
\begin{center}
\fbox{\parbox{0.5\textwidth}
{\texttt{$\backslash$usepackage[utf8]\{inputenc\}}}}
\end{center}
\end{MAJ}

Si vous êtes sous Windows le codage par défaut est \texttt{ansinew} ; il vous
faudra alors coder :
\begin{center}
\fbox{\parbox{0.5\textwidth}
{\texttt{$\backslash$usepackage[ansinew]\{inputenc\}}}}
\end{center}
%\usepackage[ansinew]{inputenc}
%%ou %\usepackage[ansinew]{keyboard}
\begin{MAJ}
Il faut aussi savoir que quelques caractères bien utiles au français
ont été omis dans le codage \textsc{iso-latin-1}, comme le e dans l'o (œ, Œ),
ce qui a été corrigé dans \textsc{iso-latin-9}. 
Malgré tout \textsc{utf-8} est préférable, maintenant que Windows
le supporte aussi (de mieux en mieux). L'éditeur gratuit \textit{Notepad++} 
permet sous Windows de faire passer un texte d'un codage à l'autre très facilement.
\end{MAJ}

Si vous êtes sur un Macintosh, le codage Apple est acceptable en français ;
codez alors :
\begin{center}
\fbox{\parbox{0.5\textwidth}
{\texttt{$\backslash$usepackage[applemac]\{inputenc\}}}}
\end{center}
%\usepackage[applemac]{inputenc}
%%ou %\usepackage[applemac]{keyboard}
\begin{MAJ}
Si enfin vous êtes sous \textsc{Unix} (ou \textsc{Linux}), le codage le plus adéquat est \textsc{utf-8}
qui est le codage par défaut de ces plateformes.
\end{MAJ}

%celui qui répond
%au nom de iso-latin-9 ; si vous ne l'avez pas avec \LaTeX{} vous pouvez utiliser
%le codage decmulti qui est très proche ou sinon l'iso-latin1. Vous pouvez
%alors coder comme suit :
%soit %\usepackage[latin9]{inputenc}
%soit %\usepackage[decmulti]{inputenc}
%soit %\usepackage[latin1]{inputenc}
%soit enfin %\usepackage[latin9]{keyboard}
%Le codage
%iso-latin-9 est proposé en standard avec l'extension keyboard (voir
%paragr. 2.7) de FrenchPro.r
%Un autre codage se généralise actuellement ; il devient progressivement le
%codage par défaut dans les systèmes d'exploitation, il s'agit du codage utf8
%issu d'unicode. Vous pouvez le coder comme suit :
%soit %\usepackage[utf8]{inputenc}
%soit enfin %\usepackage[utf8]{keyboard}
%D'autres codages existent, reportez-vous à la documentation \LaTeX{} ou à
%celle de FrenchPro (voir paragr. 2.11).
%2.7 Faut-il utiliser l'extension keyboard à la place de
%inputenc ?
%\indneg{\sl \small En bref :} Il y a de nombreux avantages (et certainement des inconvénients) à
%utiliser l'extension keyboard 27, cela est expliqué dans la documentation de
%l'extension FrenchPro (voir paragr. 2.11). Vous pouvez sans crainte particuli`
%ere remplacer inputenc par keyboard. Dans un environnement mltex6 elle
%remplace à la fois inputenc et mltex. Cela étant dit, si votre format (voir
%27. <http://frenchpro6.com/frenchpro/french/inputs/>.
%9
%paragr. 1.3) est entièrement francisé avec FrenchPro vous n'avez, a priori,
%pas besoin de faire appel à cette extension tant que vous ne changez pasr de
%codage d'entrée dans vos documents car FrenchPro fait déjà appel à l'extension
%keyboard ou plutôt au configurateur kbconfig.tex28 à la création du
%format (voir paragr. 1.3).
\subsection{Peut-on avoir son propre codage d'entrée ?}
\label{soncodent}
\indneg{\sl \small En bref :} oui, mais cela est fortement déconseillé si vous devez transmettre votre
document source à des tiers inconnus.

\indneg{\TeX\sl nique :} Avec l'extension \textit{inputenc} (voir paragr. 2.6) il vous faudra définir
un fichier personnel contenant la description des 256 caractères possibles de
votre codage perso ; c'est un peu fastidieux si vous voulez changer beaucoup
de caractères.
\textit{FrenchPro} proposait l'extension \textit{keyboard}. 

\begin{MAJ}
L'option \textit{keyboard} n'a pas été reprise par \textit{eFrench},
 la prolifération des codages d'entrée pour le français s'étant 
drastiquement tassée et \textsc{utf-8} offrant une palette de caractères dépassant largement
la limite de 256 imposée par un codage en 8 bits.
\end{MAJ}
%Grâce à l'extension keyboard (voir paragr. 2.7) de la distribution French-r
%Pro (voir paragr. 2.11) vous pouvez déclarer votre propre codage d'entrée ; il
%suffit d'éditer un fichier de configuration et d'entrer les caractères souhaités,
%par exemple :
%%------ Tableau des caracteres accentues de mo/
%% \______________________________________/
%% |___1__|___2__|___3__|___4__|___5__|___/
%\ac[ \‘ | \' | \" | \^ | {\c} | ] /
%\noms| gr | ac | um | hat | c | /
%%---------------------------------------------/
%[a | à | . | ¨a | â | . | ] /
%\hex| ^^e0 | . | ^^e4 | ^^e2 | . | /
%[A | à | . | ¨A | â | . | ] /
%\hex| ^^c0 | . | ^^c4 | ^^c2 | . | /
%%---------------------------------------------/
%Cette extension keyboard est utilisable soit au chargement d'un document
%soit au moment de la création du format (voir paragr. 1.3) ce qui revient à
%appliquer, par défaut, un même codage à tous les documents. Il est, cependant,
%toujours possible de changer de codage « à la volée » à l'intérieur d'unr
%document :
%%\usepackage{inputenc}1.3
%ou %\usepackage{keyboard}
%...
%\inputencoding{nouveau_codage }
%ou \kbencoding{nouveau_codage }
%28. <http://frenchpro6.com/frenchpro/french/inputs/keyboard/>.
%10
\subsection{Quel codage de police employer avec \LaTeX{} ?}
\indneg{\sl \small En bref :} Pour simplifier, il y a essentiellement deux codages de police utilisables
en français avec \LaTeX: \texttt{T1} ou \texttt{OT1} (voir paragr. \ref{fontcoda}).\indpos

Le codage \texttt{OT1} étant la valeur par défaut en \LaTeX, rien n'est à préciser
dans le document. Mais il faut savoir que ce codage doit être employé
conjointement avec un format Ml\TeX (voir paragr. \ref{qmltex}) pour bénéficier de la
césure des mots accentués.
{\em Mais cette méthode est aujourd'hui dépassée.}

Lorsqu'on dispose de fontes codées \texttt{T1}, comme les \textit{ec} ou les \textit{lmodern}, il
faut alors le préciser à \LaTeX, cf ci-dessous.

\indneg{\TeX\sl nique :} Pour choisir un codage de fontes \texttt{T1} et utiliser les fontes associées par
défaut on codera :

\begin{center}
\fbox{\parbox{0.5\textwidth}
{\texttt{$\backslash$usepackage[T1]\{fontenc\}}}}
\end{center}\indpos
%%\usepackage[T1]{fontenc}

On notera que toutes les polices PostScript Adobe de base sont utilisables
en codage \texttt{T1} bien que des glyphes manquent pour certaines langues. Pour
utiliser une fonte PostScript de base de chez Adobe (\textit{Times} par exemple)
on fait appel, en général, à l'extension correspondante :

\begin{center}
\fbox{\parbox{0.4\textwidth}
{\texttt{$\backslash$usepackage\{times\}}}}
\end{center}\indpos

selon Bernard \textsc{Gaulle} toutefois, il est préférable de coder :
\begin{center}
\fbox{\parbox{0.4\textwidth}
{\texttt{$\backslash$usepackage\{pslatex\}}}} .
\end{center}\indpos
%%\usepackage{pslatex}

Certaines extensions forcent un codage de fontes ; c'est, par exemple, le cas
de l'extension 
\linkandfootnote{fourier-GUT}{fourier}{http://ctan.tug.org/tex-archive/fonts/}
%fourier 16 
qui utilise les fontes fourier-GUT et exige le codage
\texttt{T1}.

\maj
%LO1 est le codage employé, en interne, lorsque l'on fait appel à l'extension
%MlTEX (voir paragr. 1.5) :
%%\usepackage{mltex}
%Si l'on dispose des fontes em (codées LY1) on fera appel à l'extension em :
%%\usepackage{em}

\subsection{{\TeX} parle anglais... irrémédiablement  !}
\indneg{\sl \small En bref :} L'essentiel des messages émis par \TeX{} réside dans un fichier spécifique
et rien n'empêche l'installateur de traduire ces messages en français,
même s'il est ensuite difficile de s'y retrouver dans la documentation de référence 
qui est en anglais. Par ailleurs, il restera toujours quelques messages
en anglais qui font partie du programme \TeX{} lui-même. Ce qu'il faut aussi
savoir, à ce niveau, c'est que \TeX{} n'est qu'un moteur fait pour véhiculer des
programmes, des styles typographiques et des extensions ; toutes sortes de
choses qui existent en grand nombre et sont toutes (ou presque) développées
avec des messages en anglais. Par ailleurs, tous les messages de \LaTeX{} sont
aussi en anglais. Dans ces conditions, une francisation du dialogue \TeX{} et de
ses extensions semble donc totalement illusoire...
%11

\indneg{\TeX\sl nique :}\indpos Toutefois, Bernard Gaulle a créé une extension \LaTeX{} qui permet de fournir des messages
dans la langue de son choix ; il s'agit de l'extension \emph{msg} (%
\linkandfootnote{doctecmsg}{ documentation technique}
{http://svn.tuxfamily.org/viewvc.cgi/efrench\_efrenchsources/trunk/doc/msg.pdf}%
).
%29. <http://svn.tuxfamily.org/viewvc.cgi/efrench_efrenchsources/trunk/doc/msg.pdf>29 
Ainsi, tout \LaTeX{} pourrait, en théorie, parler autre chose que
l'anglais. La première application de ce système de localisation s'est faite,
bien entendu, dans \textit{FrenchPro} qui ainsi « parle » plusieurs langues.
Cette extension a été reprise dans \textit{eFrench}.

\subsection{Qu'est-ce que \textit{FrenchPro (eFrench)} ?}\label{qeFrenchPro}

\indneg{\sl \small En bref :} Au fur et à mesure du temps (depuis 1989) et aidés par les utilisateurs
français de \TeX{} nous (Bernard Gaulle et al.) avons mis en œuvre divers ajustements des paramètres
de \TeX{} de façon à obtenir un aspect typographique plus conforme à l'usage,
en France et dans le monde francophone. Ces modifications, éparses et parfois
divergentes, ont été collectées et normalisées par votre serviteur 
(Bernard Gaulle, voir page \pageref{prempage})
pour en faire un style unique, robuste et de qualité, appelé à l'origine le
{\it style french}\footnote{ Le \textit{style french} qui fonctionnait aussi très bien avec plainTEX et \LaTeX{} 2.09 ne fonctionne
plus qu'avec \LaTeX{} 2$\epsilon$ depuis la version 4 de \textit{french}}
% ; voir toutefois les explications au paragr. 5.5.}
Il se décline maintenant en deux extensions :
\begin{MAJ}
 \textit{eFrench} (successeur de \textit{FrenchPro}) et
\texttt{frenchle} ; \textit{FrenchPro} était distribuée en \textit{shareware}, \textit{eFrench} 
et \texttt{frenchle} sont des logiciels libres (LPPL).\\[2ex]
\end{MAJ}

\noindent
%\subsubsection*{frenchle}
{\Large \it frenchle}\\[2ex]
\indneg{\sl \small En bref :} cette extension a pour objectif de réaliser un maximum de francisation
avec un minimum logiciel et sans connaissance particulière de la part de
l'utilisateur  (%
\linkandfootnote{docfrenchle}{voir la documentation provisoire}
{http://ctan.org/tex-archive/language/french/frenchle/frenchle.pdf}%
).
%voir la documentation 31
\maj  Cette extension  supporte le français et l'anglais ;
elle est utilisable seule ou en option de \textit{babel}. Une petite extension \textit{babelfr}
est fournie avec \textit{frenchle} pour faciliter la francisation des autres extensions
utilisées dans le document.\\[1ex]

\indneg{\TeX\sl nique :} Pour charger l'extension \textit{frenchle} il faut procéder comme suit :
\begin{center}
\fbox{\parbox{0.9\textwidth}{
\texttt{$\backslash$usepackage\{frenchle\}}
\begin{flushright}
ou ~ \texttt{$\backslash$usepackage[frenchle]\{babel\}}\\
ou plutôt ~ \texttt{$\backslash$usepackage[frenchle]\{babelfr\}}
\end{flushright}
}}\\[0.5ex]
\end{center}\indpos

ou mieux lorsqu'on utilise \textit{babel} (voir paragr. \ref{forbabel}) :
\begin{center}
\fbox{\parbox{0.9\textwidth}{
\texttt{$\backslash$documentclass[french]}\{\textit {classe\_de\_document} \}\\
\rule{0.4\textwidth}{0pt} \vdots 
 \begin{flushright}
\texttt{$\backslash$usepackage\{babelfr\}}
\end{flushright}
}}\\[4ex]
\end{center}
%\documentclass[french]{classe_de_document }
%...
%%\usepackage{babelfr}
\noindent
{\Large \it FrenchPro} ou plutôt son successeur {\Large \it eFrench}\\%[1ex]

%30.
%31. <http://ctan.org/tex-archive/language/french/frenchle/frenchle.pdf>.
%32. <http://frenchle.free.fr/>.
%33. <http://ctan.tug.org/tex-archive/language/french/>.
%12
\indneg{\sl \small En bref :} cette extension permet un francisation professionnelle de \LaTeX. Une
distribution logicielle spécifique est associée à chaque système d'exploitation
sur le serveur 
\linkandfootnote{distribefr}{efrench}
{http://www.efrench.org/distributions/}
%{http://ctan.org/tex-archive/language/french/frenchle/frenchle.pdf}%
 ; elle réside aussi sur 
\linkandfootnote{ctanefr}{\tt ctan}
{http://ctan.org/tex-archive/language/french/e-french/}.\indpos

Le fichier \linkandfootnote{ctanmanuefr}{\tt efrench.pdf}
{http://www.ctan.org/tex-archive/language/french/e-french/distributions/efrench.pdf}
 est à la fois un mode d'emploi et un manuel de référence.
Vous y trouverez de nombreuses commandes (plus de 170) permettant
une personnalisation maximum de vos documents (voir à ce sujet un descriptif
des fonctions de personnalisation dans le document 
\linkandfootnote{artET98pdf}{\tt\hbox{artET98.pdf}}
{http://svn.tuxfamily.org/viewvc.cgi/efrench_efrenchsources/trunk/doc/artET98.pdf?view=co}
).

\textit{FrenchPro} comme désormais \textit{eFrench} jouent essentiellement 7 rôles :
\begin{enumerate}
\item coupure correcte des mots français ;
\item traduction en français des différents libellés imprimés par \LaTeX;
\item possibilité d'utiliser plusieurs langues dans un même document ;
\item mise en page adaptée à l'usage français ;
\item micro-typographie (essentiellement la ponctuation) adaptée au français ;
\item apport d'un ensemble supplémentaire de macro-instructions généralement
nécessaires dans un document français (dont une liste d'abréviations
usuelles, des lettrines, etc.) ;
\item affichage personnalisé des messages à l'écran.
\end{enumerate}
Ces rôles de l'extension \textit{eFrench} %FrenchPro
 sont mis en oeuvre essentiellement par deux fichiers
(\texttt{french.sty} et \texttt{fenglish.sty}) mais la distribution comprend de nombreux
autres fichiers. Parmi ceux-ci il faut citer le fichier de césure français \texttt{frhyph.tex}
 (voir paragr. \ref{motfrahup}). D'autres encore facilitent l'installation, la mise en place
du mécanisme de césure, la définition du codage à la saisie, etc. Divers tests permettent
de s'assurer du bon fonctionnement de l'ensemble.

L'extension \textit{eFrench} est utilisable avec l'ensemble des compilateurs courants
\TeX{} ou les 
\linkandfootnote{autres}{platesformes}
{http://www.ctan.org/tex-archive/systems/}{ }
%platesformes 20 
d'exploitation mais aussi avec les options
Ml\TeX (voir paragr. \ref{qmltex}) et \texttt{TeX--XeT} (voir paragr. \ref{qtexxet}).   

\indneg{\TeX\sl nique :} \indpos Une fois installée dans le système, l'extension \textit{eFrench} peut être
appelée de telle façon, suivant qu'elle est utilisée avec l'extension multilingue
\textit{mlp} (voir paragr. \ref{latexmullin}) ou avec \textit{babel} :
\begin{center}
\fbox{\parbox{0.9\textwidth}{
\texttt{$\backslash$usepackage[french]\{mlp\}}
\begin{flushright}
ou ~ \texttt{$\backslash$usepackage[french]\{babel\}}
\end{flushright}
}}\\[0.5ex]
\end{center}

Si l'on souhaite qu'un maximum d'extensions bénéficient de la francisation
il est préférable de coder plutôt :
\begin{center}
\fbox{\parbox{0.9\textwidth}{
\texttt{$\backslash$documentclass[french]}\{\textit {classe\_de\_document} \}\\
\rule{0.4\textwidth}{0pt} \vdots \\
 \texttt{$\backslash$usepackage\{mlp\}} \hfill
ou ~ \texttt{$\backslash$usepackage\{babel\}}
}}\\[4ex]
\end{center}
En conjugaison avec \textit{babel}, seuls quelques dispositifs
sont actuellement inopérants (comme par exemple l'ordre 
\texttt{$\backslash$frhyphex}) mais
dans ce cas \textit{eFrench} n'est plus une extension autonome mais seulement
une option de l'extension \textit{babel}, ce qui témoigne de la différence de fonctionnement.
Si on désire utiliser \textit{eFrench} entièrement de façon native, dans
un environnement multilingue on fera plutôt appel à l'extension \textit{mlp}
(voir paragr.\ref{latexmullin}).

%\documentclass[french]{classe\_de\_document }
%...
%%\usepackage{mlp} ou %\usepackage{babel}
%34. <http://frenchpro6.com/french.html>.efrench.org
%35. <http://ctan.tug.org/\TeX-archive/language/french/frenchpro/>.
%36. <http://frenchpro6.com/frenchpro/french/ALIRE.pdf>.
%37. <http://frenchpro6.com/frenchpro/french/doc/artET98.pdf>.
%13
\subsection{ Y a-t-il d'autres extensions pour le français ?}
L'extension \textit{babel}, qui permet d'utiliser une grande variété de langues
dans un même document, propose une option de francisation « 
\linkandfootnote{frenchb}{\textit{french\textbf{b}}}
{http://ctan.org/tex-archive/macros/latex/required/babel/frenchb.dtx} ».
Cette option, développée par Daniel \textsc{Flipo}, apporte des facilités de microtypographie
ainsi que, bien entendu, de traduction en français. Elle est en
tout point, très comparable, à l'extension \texttt{frenchle} dont elle s'inspire\footnote{%
« I have borrowed many ideas from Bernard's file. » (frenchb v1.1 revised
1996/05/31)}.
Il faut savoir que \hbox{\textit{french\textbf{b}}} n'est pas, pour des raisons historiques et de
cohérence, l'option prédominante avec \textit{babel} dans \textit{eFrench} (ou avec \textit{babelfr}
de \textit{frenchle}) quand on précise l'option \textit{french} au lieu de \textit{frenchb}. Par ordre
de préférence, c'est \textit{eFrench} (\texttt{french.sty}) qui est chargé, à défaut ce sera \texttt{frenchle},
%\texttt{frenchb} si aucune des précédentes n'est présente.
ou \texttt{frenchb}  sinon.

\indneg{\TeX\sl nique :} Pour bien choisir l'extension de francisation que l'on désire employer
avec \textit{babel}\/  il suffit que vous précisiez la bonne option :
\begin{center}
\fbox{\parbox{0.9\textwidth}{
\begin{flushright}
soit ~ \texttt{$\backslash$usepackage[frenchle]\{babel\}}\\
mais plutôt ~ \texttt{$\backslash$usepackage[frenchle]\{babelfr\}}
\end{flushright}
soit ~ \texttt{$\backslash$usepackage[frenchb]\{babel\}}
\begin{flushright}
soit  enfin ~ \texttt{$\backslash$usepackage[efrench]\{babel\}}
\end{flushright}
}}\\[4ex]
\end{center}
%FL soit %\usepackage[frenchle]{babel}
%FL mais plutôt %\usepackage[frenchle]{babelfr}
%soit %\usepackage[frenchb]{babel}
%FL soit enfin %\usepackage[frenchpro]{babel}
\section{ Autres outils français}
\Subsubsection*{Guillemets aeguill}
\indneg{\sl \small En bref :} Cette extension 
\linkandfootnote{aeguill}{\it aeguill}
{http://www.ctan.org/tex-archive/macros/latex/contrib/}
dûe à Denis {\sc Roegel} apporte un choix de
fontes supplémentaire pour l'impression des guillemets français (fait appel à
l'extension \texttt{ae})
\Subsubsection*{Dictionnaire}
\indneg{\sl \small En bref :} Réalisé par Christophe \textsc{Pythoud}, et nommé
\linkandfootnote{dictfran}{francais-GUTenberg}
{http://cahiers.gutenberg.eu.org/cg-bin/article/CG_1998___28-29_252_0.pdf} %41
 ce dictionnaire élaboré, fonctionne avec le vérificateur d'orthographe {\sc ispell} (et donc
ne peut être utilisé que sur les systèmes pour lesquels {\sc ispell} est disponible).
\Subsubsection*{Classe lettre}
\indneg{\sl \small En bref :} La classe 
\linkandfootnote{lettreobsgen}{\it lettre}
{http://www.ctan.org/tex-archive/macros/latex/contrib/lettre/doc/letdoc.ps}%42
, de Denis \textsc{Mégevand}, est une adaptation du style
lettre développé à l'origine à l'Observatoire de Genève. Elle permet de composer
avec \LaTeX{} des lettres ou des téléfax dont l'allure correspond mieux
aux usages francophones. 

\begin{MAJ}
Elle fonctionne parfaitement avec \textit{frenchb} (et \textit{babel}).
Et très mal avec \textit{frenchle} ou \textit{eFrench}.
Un défi supplémentaire pour notre équipe !

\end{MAJ}
\Subsubsection*{Bibliographie et index}
%38. <>.
%39. 
%40. <>.
%41. <>.
%42. <>.
%14
Il existe dans \textit{eFrench} un système de traduction des libellés utilisés pour
une bibliographie. D'autres contributions à la typographie française existent sur 
\linkandfootnote{bibctan}{\tt ctan}
{http://www.ctan.org/tex-archive/biblio/bibtex/}.

\indneg{\TeX\sl nique :} pour trier correctement les entrées bibliographiques on utilisera de
préférence la version 8-bits de BiB\TeX{} : \textit{bibtex8}, et \textit{x\r{\i}ndy} pour l'index.
\section{ \LaTeX{} multilingue}\label{latexmullin}
\indneg{\sl \small En bref :} L'extension 
\linkandfootnote{babel}{\it babel}
{http://www.ctan.org/tex-archive/macros/latex/required/babel/}
 de \LaTeX{} permet de composer des documents faisant
appel à de nombreuses langues de la planète, écrites ou imprimées.
Il existe d'autres extensions multilingues, dont les extensions fournies avec \textit{eFrench}
et \texttt{frenchle} qui sont conçues pour fonctionner dans un environnement multilingue
(avec par défaut le français et l'anglais) mais aussi l'extension \textit{mlp}.\\

{\Large\textit{L'extension mlp}}\\

\indneg{\sl \small En bref :} Il s'agit d'une extension multilingue qui permet une cohabitation
harmonieuse d’extensions linguistiques autonomes ; elle
fait partie intégrante du paquet
% est distribuée avec
\hbox{\textit{eFrench}}.

\indneg{\TeX\sl nique :} On peut faire appel à 
\linkandfootnote{mlp}{\it  mlp} 
{http://svn.tuxfamily.org/viewvc.cgi/efrench\_efrenchsources/trunk/inputs/mlp/}
de la façon suivante :
\begin{center}
\fbox{\parbox{0.75\textwidth}
{\tt $\backslash$usepackage[english,german,french]\{mlp\}}}\\[2ex]
\end{center}
%%\usepackage[english,german,french]{mlp}
ou alors ainsi :
\begin{center}
\fbox{\parbox{0.75\textwidth}
{\tt $\backslash$usepackage[anglais,allemand,francais]\{mlp\}}}
\end{center}
%%\usepackage[anglais,allemand,francais]{mlp}\LaTeX
et dans les deux cas les parties en anglais, en allemand ou en français seront
composées selon ces styles. Pour passer d'une langue à une autre on code
\texttt{$\backslash$french} ou \texttt{$\backslash$english} ou \texttt{$\backslash$german} selon le cas.\indpos

\begin{MAJ}
Remarquons toutefois que \textit{mlp} est nettement plus restreint que \textit{babel} dans
la variété des langues disponibles, à moins de mettre la main à la pâte et de
compléter la collection des fichier d'extension \texttt{.mlp} et des fichiers \texttt{mlp-*.sty}.
En contrepartie, cet environnement laisse à \textit{eFrench} toutes ses possibilités.
(voir vers la fin du paragr. \ref{qeFrenchPro})
\end{MAJ}
\section{Questions relatives à \textit{eFrench}}
\subsection{Est-ce que \textit{eFrench} s'installe automatiquement ?}
\begin{MAJ}
\indneg{\sl \small En bref :} Si \textit{FrenchPro} oui, s'installait automatiquement, mais de manière différente
suivant les systèmes d'exploitation,  concernant \textit{eFrench} l'installation a été
grandement simplifiée et il suffit de placer la structure du paquet au bon endroit
dans la structure \textit{TDS}
et de lancer un \texttt{texhash} (ou son équivalent dans Mik\TeX{}, 
voir plus bas, fin du paragr. \ref{mikroot})
concernant cet endroit et le but est atteint.
%est 
%Avant toute chose, veuillez consulter
%les informations concernant le moteur TEX47 que vous utilisez, pour
%connaître les détails éventuels liés à votre compilateur.
%43. <>.
%45. <>.
%46. <>.
%47. <http://frenchpro6.com/frenchpro/french/engines/>.
%15
%\indneg{\TeX\\\sl nique :}
%Avec Windows vous disposez de deux installateurs :
%– l'un pour le moteur fpTEX48 qui est en fait un moteur teTEX49\LaTeX
%adapté à win32.\texttt{frenchb} 
%Pour lancer l'installation placez-vous dans le répertoire french et
%tapez :\LaTeX
%install
%Choisissez ensuite de préférence l'installation « à la unix » qui
%est un peu plus longue mais beaucoup plus efficace (et surtout\indneg{\TeX\\\sl nique :}
%complète).
%– pour le moteur MikTEX50 vous pouvez procéder de la même mani`
%ere ou sinon placez-vous dans le répertoire french et tapez :
%mikinstall
%Vous disposez, bien entendu, des procédures « uninstall » correspondantes.
%Avec un unix le moteur de prédilection est celui de teTEX49. Pour installer
%FrenchPro dans un environnement gnu il suffit de se placer dans le
%répertoire french et de taper :
%gmake -f GNUmakefile
%Voir le fichier GNUmakefile51 pour une éventuelle personnalisation.
%Dans un environnement non-gnu vous disposez d'un makefile.gen52
%qu'il suffit d'adapter à votre configuration et de sauvegarder sous le
%nom de makefile. Ensuite vous demandez simplement son exécution :
%make\textsc{Unix}
%Avec MacOS si vous utilisez une version de type unix (c'est-à-dire  10)\LaTeX
%il suffit de vous reporter à la description précédente, sinon veuillez lire
%la petite notice d'installation fournie qui vous expliquera ce qu'il reste
%à faire 53.
%5.2 Qu'est-ce qu'un format francisé pour FrenchPro ?
%\indneg{\sl \small En bref :} Pour FrenchPro un format est entièrement francisé lorsque les motifs
%de césure (voir paragr. 2.4) du français ont été installés et que le codage
%\indneg{\TeX\\\sl nique :}
%d'entrée (voir paragr. 2.5) du texte y a été défini.
%48. <http://frenchpro6.com/frenchpro/french/engines/fptex.pdf>.1.3
%49. <http://frenchpro6.com/frenchpro/french/engines/tetex.pdf>.
%50. <http://frenchpro6.com/frenchpro/french/engines/miktex.pdf>.
%51. <http://frenchpro6.com/frenchpro/french/GNUmakefile>.
%52. <http://frenchpro6.com/frenchpro/french/makefile.gen>.
%53. La distribution de FrenchPro pour MacOs 9 ou inférieur est définitivement gelée.
%16
%5.3 Peut-on mettre toute la francisation dans\textsc{Unix} le format?
%\indneg{\sl \small En bref :} Qu'il s'agisse de FrenchPro ou de frenchle on utilise un format francis
%é et on aurait envie de ne plus avoir à coder le %\usepackage{french...} ;
%malheureusement cela n'est pas possible dans la version actuelle.
%5.4 Pourquoi des distributions de FrenchPro pour Unix,
%Mac et Windows?
%\indneg{\sl \small En bref :} Il existe effectivement des distributions différentes, adaptées aux platesformes
%d'exploitation, tout simplement parce que d'une part les codages d'entr1.3
%ée (utilisés dans divers documents source) sont différents mais aussi parce
%que les procédures d'installation et les configurations sont différentes.
%5.5 FrenchPro fonctionne-t-il avec plainTEX
%ou \LaTeX{} 2.09?
%\indneg{\sl \small En bref :}
%Non :
%FrenchPro, depuis la version 4, n'est plus conçu pour fonctionner
%avec plainTEX ni \LaTeX{} 2.09.
%Oui :
%il existe dans la distribution de FrenchPro (dossier obsolate) un
%fichier frplain.sty gelé à la version V3,51 qui fonctionne avec\{mltex\}
%plainTEX. Par ailleurs, à l'installation de FrenchPro le format
%plain francisé est créé automatiquement.
%Oui
%Il y a aussi dans la distribution de FrenchPro (dossier obsolate)
%un fichier frltx209.sty gelé à la version V3,51 qui fonctionne
%encore avec le très, très, très ancien \LaTeX{} 2.09.
%Ces deux derniers fichiers sont définitivement figés et ne seront plus mis
%à jour. Ils sont destinés à aider les personnes qui ont encore des fichiers nonforbabel
%convertis à \LaTeX. La distribution actuelle des fichiers de francisation permet
%donc d'utiliser à la fois plainTEX, \LaTeX{} 2.09 et \LaTeX{} mais seules les extensions
%pour \LaTeX, FrenchPro et frenchle, sont maintenues régulièrement à
%jour.
\subsection{Où installer \textit{eFrench} dans la hiérarchie \textit{TDS} ?}
\indneg{\TeX\sl nique :}  Les fichiers de la distribution \textit{eFrench} sont à déposer de
préférence dans :\\
 .../texmf-local/french, où \\
 ... sera dépendant du système d'exploitation,
sous \textsc{Unix} par exemple ce pourra être {\bf$\tilde{ }$} (dossier personnel), 
sous Windows on pourra simplement choisir C:$\backslash$.
La structure finale sera : \\[1ex]
\textit{user-space}/\texttt{texmf-local}/tex/french/...\\
\textit{user-space}/\texttt{texmf-local}/tex/french/french.sty\\
\textit{user-space}/\texttt{texmf-local}/tex/french/...\\
\textit{user-space}/\texttt{texmf-local}/tex/french/french.msg\\
\textit{user-space}/\texttt{texmf-local}/tex/french/...\\
\textit{user-space}/\texttt{texmf-local}/tex/latex/frenchle.sty\\
\textit{user-space}/\texttt{texmf-local}/tex/latex/...\\
où \textit{user-space}/\texttt{texmf-local} est ~{\bf $\tilde{ }$}~/\texttt{texmf-local} sous \textsc{Unix} ; \texttt{C:$\backslash$texmf-local}
ou un choix personnel sous Windows, du genre 
\texttt{C:$\backslash$Users$\backslash$}\textit{identificateur}\texttt{$\backslash$texmf-local}.
En plus, sous Windows,  les dossiers sont introduits par $\backslash$ et non par / .
Sous Linux et \TeX Live, ne pas oublier l'annonce :
\begin{center}
\fbox{\parbox{0.4\textwidth}
{\tt \$texhash texmf-local}}\\[1ex]
\end{center}
alors que sous Windows et Mik\TeX, il faut déclarer  le dossier défini (\texttt{C:$\backslash$texmf-local} par exemple)
sous {\sc Roots}\label{mikroot} et lancer {\sc Refresh FNDB}, qui sont des options d'administration de Mik\TeX.
\end{MAJ}
%L'installation proprement
%dite de ces fichiers est automatique (avec le script adapté à votre système
%17
%d'exploitation) ; elle intervient à divers endroits de la hiérarchie tds (voir
%paragr. 1.8) :
%/texmf-local/doc/french/base/
%/texmf-local/generic/hyphen/\{mltex\}
%/texmf-local/[ml]\TeX/french/base/
%/texmf-local/mkindex/french/
%/texmf-local/bibtex/french/
%/texmf-local/[ml]\TeX/french/config/
%/texmf-local/\TeX/\LaTeX/frenchle/

\indpos Pour  \textit{eFrench}, vous effectuerez donc l'installation « à la main »,  
 en suivant scrupuleusement les emplacements que nous avons indiqués.
%dans le fichier TDSfr.pdf 54 de la distribution.
\subsection{Peut-on mettre des commandes \textit{eFrench} avant\\
le $\backslash${\tt begin\{document\}} ?}
\indneg{\sl \small En bref :} La documentation d'\textit{eFrench} précise 
que « le style \textit{eFrench} ne
devient vraiment actif qu'après le $\backslash${\tt begin\{document\}} » ; on ne peut donc pas,
en principe, exécuter de commande \textit{eFrench} avant le $\backslash${\tt begin\{document\}} ;
sauf celles, en nombre très réduit qui peuvent agir sur tout le document.

\indneg{\TeX\sl nique :} L'extension \textit{eFrench} déporte la définition même d'un grand nom\-bre
de commandes au moment du $\backslash${\tt begin\{document\}}. Les commandes ainsi définies
ne sont pas utilisables avant.\indpos

Dans le cas d'\textit{eFrench}, quelques commandes très spécifiques peuvent
être utilisées avant, car elles modifient son comportement au 
$\backslash${\tt begin\{document\}}.
 Il en est ainsi de $\backslash${\tt frhyphex} pour charger un fichier de césures
exceptionnelles ou $\backslash${\tt usersfrenchoptions} qui permet de changer les options
par défaut de \textit{eFrench} comme dans cet exemple :
\begin{center}
\fbox{\parbox{0.9\textwidth}
{\tt $\backslash$usersfrenchoptions\{$\backslash$guillemetsinallfonts\}}}\\[1ex]
\end{center}
%\usersfrenchoptions\{$\backslash$ guillemetsinallfonts\}
où on a choisi une option différente pour les guillemets à la française. Grâce
à ce mécanisme \texttt{$\backslash$usersfrenchoptions} l'activation d'options personnelles
peut se faire à chaque changement de langue et donc initialement après le
\texttt{$\backslash$begin{document}}. Toute règle ayant une exception, ceci fonctionne de façon
différente avec \textit{babel} (voir paragr. \ref{forbabel}) ; consulter à ce sujet la 
\linkandfootnote{ctanmanuefr}{documentation d'eFrench}
{http://www.ctan.org/tex-archive/language/french/e-french/distributions/efrench.pdf}. %eFrench%\textit{eFrench} 

%54. <http://frenchpro6.com/frenchpro/french/TDSfr.pdf>.1.3
%18
\subsection{ Faut-il faire un format \textit{eFrench} ou un format \textit{babel}  ?}
\label{forbabel}
\indneg{\sl \small En bref :} Que ce soit \textit{babel} 
ou \textit{eFrench} (\textit{FrenchPro}), ni l'un ni l'autre n'exige de format
particulier mais chacun d'eux conseille de créer un format avec du matériel
spécifique, d'où le dilemme tant de fois discuté sur les news, par exemple
 (\texttt{fr.comp.text.tex})
 ou sur la liste de discussion 
\linkandfootnote{GUTenberg}{GUTenberg}{http://www.gutenberg.eu.org}
 (\texttt{gut@ens.fr}).
%GUTenberg 12. 
Rappelons
pour simplifier que les deux fonctionnent bien avec le matériel de l'autre,
c.-à-d. un format construit avec les fichiers d'\textit{eFrench} n'empêchera pas
d'utiliser \textit{babel} et réciproquement. 
\begin{MAJ}
Il faut dire ici que l'insistance de Bernard \textsc{Gaulle}, 
créateur de \textit{FrenchPro}, concernant les formats 
n'est plus de mise. 
Les installations récentes telles Mik\TeX{} ou \TeX Live offrent
les formats de base et les moyens de créer les autres.
\pgLapdTeX{} évolue dans le sens d'une simplification pour l'utilisateur et c'est tant mieux !
\end{MAJ}
%La question à se poser lorsque l'on crée son
%format est donc : s'agit-il d'utiliser essentiellement babel ou FrenchPro ? et selon
%le cas on crée le format le plus adapté mais rien n'empêche d'avoir aussi
%plusieurs formats. Pour être complet, rappellons qu'il existe deux autres options
%de francisation, l'une frenchle31 et l'autre frenchb (voir paragr. 2.12)
%livrée en standard avec babel (et utilisable uniquement avec babel ).
%Cela répond à la question : FrenchPro ou babel ? maintenant il faut répondre
%à la question : FrenchPro et babel ?
%\indneg{\TeX\\\sl nique :} On peut utiliser FrenchPro avec babel de la façon suivante :
%%\usepackage[frenchpro]{babel}
%ou mieux :
%\documentclass[french]{classe_de_document }1.3
%...
%%\usepackage{babel}
%sans perdre un poumon (comme disait quelqu'un), seuls quelques dispositifs
%sont actuellement inopérants (comme par exemple l'ordre \frhyphex) mais
%dans ce cas FrenchPro n'est plus une extension autonome mais seulement
%une option de l'extension babel, ce qui témoigne de la différence de fonctionnement.
%Si on désire utiliser FrenchPro entièrement de façon native dans\{mltex\}
%un environnement multilingue on fera plutôt appel à l'extension mlp (voir
%paragr. 4) :
%\documentclass[french]{classe_de_document }
%...
%%\usepackage{mlp}% 40 
\subsection{Polices de caractères et guillemets français}
\indneg{\sl \small En bref :} les guillemets saisis sous forme ascii (<\/<\/ ... >\/>) ou selon le code approprié
 à votre clavier (« et »), sont utilisables avec l'extension \textit{eFrench}
qui sait les gérer dans de nombreuses circonstances typographiques (premier
et deuxième niveau, listes, lettrines, etc.). Concernant le choix de la police,
\textit{eFrench} regarde quelle est la police par défaut ({\it cm}, {\it ec}, ...) pour déterminer
%19
quels codes de guillemets il faut imprimer. On obtiendra ainsi toujours de
beaux guillemets français imprimés.

\indneg{\TeX\sl nique :}  Voici un descriptif rapide des méthodes employées :
\begin{itemize}\item  par défaut (c.-à-d. fontes {\it cm}), ces guillemets sont simulés à partir de
glyphes venant de la police {\it lasy} (ce qui explique leur aspect légèrement
arrondi) et, toujours par défaut, ce sont les seuls guillemets disponibles.
Rappelons que l'utilisation de 
\texttt{$\backslash$usepackage[OT1]\{fontenc\}}
 %\usepackage[OT1]{fontenc}
force l'utilisation des fontes {\it cm}.
\item avec le choix des fontes {\it ec} ou {\it lmodern} (c.-à-d. en demandant leur codage en utilisant   
\texttt{$\backslash$usepackage[T1]\{fontenc\}}) 
%\usepackage[T1]{fontenc})
les guillemets sont imprimés dans la police \texttt{$\backslash$rmdefault} et, toujours
par défaut, ce sont les seuls glyphes de guillemets disponibles. 

L'ordre \texttt{$\backslash$guillemetsinallfonts }
 permet d'utiliser les guillemets de la police
en cours d'utilisation, quelque soit cette police ; cela permet donc de
jouer avec les graisses, les styles et les familles de polices, (ce qui, 
selon le goût de Bernard Gaulle,
n'est pas toujours optimum).
\item lorsque les guillemets sont saisis en ascii (<\/< ... >\/>) les glyphes imprimés 
sont toujours des ligatures mais la manière diffère selon qu'il
s'agit d'une police {\it lasy} ou d'une police {\it ec}. Dans le premier cas c'est
une ligature fabriquée par l'extension {\it eFrench}, dans l'autre cas c'est
une ligature « fondue » dans la police. En aucun cas {\pgLapdTeX} ne peut
couper en deux un guillemet en bout de ligne.
\item  il est possible d'imprimer les guillemets avec d'autres polices ; on peut
par exemple utiliser l'extension 
\linkandfootnote{aeguill}{\it aeguill}
{http://www.ctan.org/tex-archive/macros/latex/contrib/}.
%aeguill 40.% 40 
\end{itemize}
\subsection{Comment personnaliser l'extension \textit{eFrench} ?}

\indneg{\sl \small En bref :}\indpos  L'extension \textit{eFrench} est entièrement configurable ; vous pouvez donc personnaliser vos documents à l'envi.

\indneg{\TeX\sl nique :} Un article a été dédié à ce sujet, vous y trouverez plein de renseignements très  utiles : 
\linkandfootnote{artET98pdf}{\hbox{artET98.pdf}}
{http://svn.tuxfamily.org/viewvc.cgi/efrench_efrenchsources/trunk/doc/artET98.pdf?view=co}
%artET98.pdf55 
en complément de la documentation d' 
\linkandfootnote{ctanmanuefr}{eFrench.}
{http://www.ctan.org/tex-archive/language/french/e-french/distributions/efrench.pdf}%eFrench.
%Pro 36 .
\vfill

\indpos\textbf{Si vous souhaitez améliorer ce document} ce sera avec un grand
plaisir que nous recevrions vos messages à : \indneg


\href{mailto:efrench(at)lists.tuxfamily.org}{\texttt{efrench@lists.tuxfamily.org}}\hfill
ou \hfill \href{mailto:raymond(at)juillerat-marly-ch.net}{\texttt{raymond@juillerat-marly-ch.net}}\hfill\rule{0pt}{1ex}\\
ou encore  \href{mailto:lb(at)laurentbloch.org}{\texttt{lb@laurentbloch.org}}
\footnote{Attention, l'adresse m.él. fournie via les butineurs est
piégée ; il faut la corriger à la main.}.\indpos
%?—?
%55. <http://frenchpro6.com/frenchpro/french/doc/artET98.pdf>.
%20
%\pagebreak
\vfill
\appendix



\def\AvertO{\ \vfill%
%%%     Ceci était un extrait de la documentation de l'extension
%%% \eFrench.

    \begin{center}*---*\end{center}
    \def\AvertO{}}

\ifnoDOCinstall\par\medskip\AvertO\fi%
\ifnoFINALS\par\medskip\AvertO\fi%

\vfill
%\newpage

\bibliographystyle{plain}
\ifx\Section\undefined\else\let\section\Section\fi% Restore \section.
\def\refname{Références bibliographiques}
\begin{drapeaufg}
\begin{thebibliography}{9}
\bibitem{lblochefr}{L. Bloch}
 \textit{\linkandfootnote{rlblochefr}{Lancement du projet \textit{eFrench}}{http://www.laurentbloch.org/spip.php?article166}}
\bibitem{cesures} D. Flipo, B. Gaulle et K. Vancauwenberghe,\textit{ 
\linkandfootnote{ rcesures}{Motifs de césure français}{http://cahiers.gutenberg.eu.org/cg-bin/article/CG_1994___18_35_0.pdf}}, 
{\it in} Les Cahiers GUTenberg No 18, 1994.
\bibitem{frenchle} B. Gaulle, 
\textit{\linkandfootnote{reffrenchle}{%
L'extension frenchle pour \LaTeX, notice d'utilisation 31}%
{http://ctan.org/tex-archive/language/french/frenchle/frenchle.pdf}}, 2007
\bibitem{frenchpro} B. Gaulle, 
\textit{\linkandfootnote{ rfrenchpro}{Guide d'utilisation de l'extension \textit{eFrench}}%
{http://www.efrench.org/distributions/efrench.pdf}}, 2007
\bibitem{personnalis} B. Gaulle, \textit{\linkandfootnote{commperso}{Comment peut-on personnaliser l'extension french de \LaTeX{} ?}{http://cahiers.gutenberg.eu.org/cg-bin/article/CG_1998___28-29_143_0.pdf}}
 document {\it in} Les Cahiers GUTenberg No 28-29, 1998
\bibitem{lepetity} E. Saudrais,
 \textit{\linkandfootnote{rlepetity}{Le petit typographe rationel}{http://tex.loria.fr/typographie/saudrais-typo.pdf}} 
\end{thebibliography}

\end{drapeaufg}
\vfill
\newpage
\nooverfullhboxmark% Pourquoi faut-il le repréciser pour la versions screen?

\def\footnote#1{}% nullify any footnote here.
\tableofcontents

\end{document}
